\documentclass[lecture]{csm}

% set meta information
\modulecode{MA1500}
\moduletitle{Introduction to Probability Theory}
\academicyear{2013/14}
\doctype{Lecture}
\doctitle{Quantiles}
\docnumber{20}

% local
\newcommand{\prob}{\mathbb{P}}
\newcommand{\expe}{\mathbb{E}}
\newcommand{\N}{\mathbb{N}}
\newcommand{\R}{\mathbb{R}}
\def\it{\item}
\def\bit{\begin{itemize}}
\def\eit{\end{itemize}} 
\def\ben{\begin{enumerate}}
\def\een{\end{enumerate}}
\newcommand{\lt}{<}
\newcommand{\gt}{>}

%======================================================================
\begin{document}
\maketitle
\tableofcontents
%======================================================================


%----------------------------------------------------------------------
\newpage
\section{Continuous distributions}
%----------------------------------------------------------------------
Let $X$ be a continuous random variable, and let $F$ denote its CDF.
\bit
\it Recall that CDFs are \emph{increasing}: $$x<y \ \Rightarrow\ F(x)\leq F(y).$$
\it In this lecture, we assume that $F$ is \emph{strictly increasing}: $$x<y \ \Rightarrow\ F(x)<F(y).$$
\it This ensures that the inverse function $F^{-1}:[0,1]\to\R$ is uniquely defined.
\it The ideas presented here can be extended to discrete distributions, with some modifications.
\eit

%----------------------------------------------------------------------
\newpage
\section{Median}
%----------------------------------------------------------------------
% location
The mean of a distribution can be thought of as a \emph{location parameter}
\bit
\it The expected \emph{squared deviation} of $X$ from the point $c\in\R$ is $\expe\big[(X-c)^2\big]$.
\it The mean of $X$ is the value of $c$ that minimises the expected squared deviaiton. 
\eit

\vspace{2ex}
Another location parameter is provided by a \emph{median} of a distribution.
\bit
\it The expected \emph{absolute deviation} of $X$ from the point $c\in\R$ is $\expe\big[|X-c|\big]$.
\it The median of $X$ is the value of $c$ that minimises the expected absolute deviaiton. 
\eit

% defn: median
\begin{definition}
A median of a distribution $F:\R\to[0,1]$ is a number $m\in\R$ such that 
\[
\displaystyle F(m)=\prob(X\leq x) = \frac{1}{2}.
\] 
\end{definition}

\begin{remark}
If $F$ is continuous and strictly increasing, the median is uniquely defined.
\end{remark}

%----------------------------------------------------------------------
\section{Quantiles}
%----------------------------------------------------------------------
The $q$-quantiles of a distribution divide the real line into $q$ `equal' parts:

% defn: q-quantiles
\begin{definition}
For $q\in\N$, the $q$-quantiles of $F:\R\to[0,1]$ are values $x_1 < x_2 < \ldots < x_{q-1}$ such that 
\[
F(x_k) = \frac{k}{q} \qquad\text{for}\quad k=1,2,\ldots,q-1
\]
\end{definition}

In particular,
\bit
\it The $2$-quantile is the \emph{median} of $F$.
\it The $4$-quantiles are the called the \emph{quartiles} of $F$.
\it The $100$-quantiles are called the \emph{percentiles} (or \emph{percentage points}) of $F$.
\eit

%----------------------------------------------------------------------
\newpage
\subsection{Quartiles}
%----------------------------------------------------------------------
\begin{definition}
Let $X$ be a random variable with distribution function $F$.
\ben
\it 
The \emph{lower quartile} of $F$ is a number $x_L$ such that $\displaystyle F(x_L) = \prob(X \leq x_L) = \frac{1}{4}$.
\it
The \emph{median} of $F$ is a number $m$ such that $\displaystyle F(m) = \prob(X \leq m) = \frac{1}{2}$.
\it
The \emph{upper quartile} of $F$ is a number $x_U$ such that $\displaystyle F(x_U) = \prob(X \leq x_U) = \frac{3}{4}$.
\it 
The \emph{inter-quartile range} of $F$ is the difference $x_U-x_L$ between its upper and lower quartiles.
\een
\end{definition}

% location and scale
\begin{remark}
\bit
\it The lower quartile is the median of the lower half of the distribution.
\it The upper quartile is the median of the upper half of the distribution.
\eit
\bit
\it The median quantifies the \emph{location} of a distribution.
\it The inter-quartile range quantifies the \emph{scale} of a distribution.
\eit
\end{remark}

\newpage

% example: quartiles
\begin{example}
A continuous random variable $X$ has PDF 
$
f(x) = \begin{cases}
	3x^2/8		& \text{if }\ 0\leq x\leq 2, \\
	0		& \text{otherwise}.	
\end{cases}
$
\par
Find the median and the interquartile range of this distribution.
%Find
%\ben
%\it the value of $c$,
%\it the median, and 
%\it the interquartile range.
%\een
\end{example}

\begin{solution}
\ben
%\it % << (i)
%$\displaystyle\int_{-\infty}^{\infty} f(x)\,dx = c\int_0^2 x^2\,dx = c\left[\frac{x}{3}\right]_0^2 = \frac{8c}{3}$.\par
%Since $f(x)$ is a probability mass function, its integral of $f(x)$ over $\R$ is equal to $1$, so $c=3/8$.
\it % << (ii)
The median $m$ satisfies
\[
F(m) = \frac{3}{8}\int_0^{m} x^2\,dx = \frac{1}{2}
\quad\Rightarrow\quad
\frac{3}{8}\times\frac{m^3}{3} = \frac{1}{2}
\quad\Rightarrow\quad
m = 4^{1/3} = 1.5874.
\]
\it % << (iii)
The lower quartile $x_L$ satisfies
\[
F(x_L) = \frac{3}{8}\int_0^{x_L} x^2\,dx = \frac{1}{4}
\quad\Rightarrow\quad
\frac{3}{8}\times\frac{m^3}{3} = \frac{1}{4}
\quad\Rightarrow\quad
x_U = 2^{1/3} = 1.2599.
\]
The upper quartile $x_U$ satisfies
\[
F(x_U) = \frac{3}{8}\int_0^{x_U} x^2\,dx = \frac{3}{4}
\quad\Rightarrow\quad
\frac{3}{8}\times\frac{m^3}{3} = \frac{3}{4}
\quad\Rightarrow\quad
x_L = 6^{1/3} = 1.8171.
\]
Hence the inter-quartile range is equal to $x_U-x_L = 1.8171 - 1.2599 = 0.5572$.
\een
\end{solution}

%----------------------------------------------------------------------
\newpage
\subsection{Percentiles}
%----------------------------------------------------------------------
\begin{definition}
The $k$th percentile of a distribution function $F$ is a value $x_k$ such that
\[
F(x_k) = \frac{k}{100} \qquad\text{for}\quad k=1,2,\ldots,99
\]
In particular, 
\ben
\it The $25$th percentile is the lower quartile.
\it The $50$th percentile is the median.
\it The $75$th percentile is the upper quartile.
\een
\end{definition}


%----------------------------------------------------------------------
\newpage
\section{The quantile function}
%----------------------------------------------------------------------
% quantile function
Let $F$ be a continuous CDF.
\ben
\it The \emph{quantile function} $Q:[0,1]\to\R$ yields a value of $x$ for which $F(x)=p$.
\it If $F$ is strictly increasing, the quantile function is just the inverse function $F^{-1}$.
\it For fixed $p\in[0,1]$, $x_p = Q(p)$ called the \emph{critical point} of the distribution at level $p$.
\een

% upper and lower tails
For example, the critical points at levels $p=0.05$ and $p=0.95$ satisfy
\[\begin{array}{rlll}
F(x_{0.05}) & = \prob(X\leq x_{0.05}) & = 0.05. 		& \quad\text{(lower tail)} \\
1-F(x_{0.95}) & = \prob(X\geq x_{0.95}) & = 0.05. 	& \quad\text{(upper tail)}
\end{array}\]
%Note that $x_{0.05}$ and $x_{0.95}$ are respectively the $5$th and $95$th percentiles of $F$.

% typical and extreme events
\begin{remark}
\bit
\it \emph{Typical event}: $\{x_{0.05} < X < x_{0.95}\}$ occurs with probability $0.9$. 
\it \emph{Extreme event}: $\{X\leq x_{0.05}\}\cup\{X\geq x_{0.95}\}$ occurs with probability $0.1$.
\it \emph{Confidence intervals} and \emph{statistical hypothesis tests} are constructed on this basis.
\eit
\end{remark}
% example: exponential distribution
\begin{example}
Let $X$ have (negative) exponential distribution with rate parameter $\lambda$. %This has distribution function
\[
F(x) = \prob(X\leq x) = \begin{cases}
	1 - e^{-\lambda x}	& x > 0, \\
	0					& \text{otherwise.}
\end{cases}
\]
Derive an explicit expression for the quantile function of $F$.
\end{example}

\begin{solution}
Let $x_p = Q(p)$ where $p\in[0,1]$. Then
\begin{align*}
p = F(x_p) 
	& \ \Rightarrow\ p = 1 - e^{-\lambda x_p} \\
	& \ \Rightarrow\ e^{-\lambda x_p} = 1 - p \\
	& \ \Rightarrow\ -\lambda x_p = \log(1 - p) \\
	& \ \Rightarrow\ x_p = -\frac{\log(1 - p)}{\lambda}.
\end{align*}

In particular,
\[
x_{0.05} = \frac{\log(20/19)}{\lambda}, \quad
x_{0.25} = \frac{\log(4/3)}{\lambda}, \quad
x_{0.50} = \frac{\log(2)}{\lambda}, \quad
x_{0.75} = \frac{\log(4)}{\lambda}, \quad
x_{0.95} = \frac{\log(20)}{\lambda}.
\]
%\[
%\begin{array}{lll}
%%x_{0.05}		& = \displaystyle\frac{\log(20/19)}{\lambda} 		& \text{($5$th percentile).}\\
%%x_{0.05}		& = \displaystyle\frac{\log(20/19)}{\lambda} 		& \text{($5$th percentile).}\\[1ex]
%x_{0.25}		& = \displaystyle\frac{\log(4/3)}{\lambda} 		& \text{(lower quartile).}\\
%x_{0.50}		& = \displaystyle\frac{\log(2)}{\lambda} 			& \text{(median).}\\
%x_{0.75}		& = \displaystyle\frac{\log(4)}{\lambda} 			& \text{(upper quartile).}\\
%%x_{0.95}		& = \displaystyle\frac{\log(20)}{\lambda} 		& \text{($95$th percentile).}	
%\end{array}
%\]
\end{solution}
\normalsize


% statistical tables
\begin{remark}[Statistical Tables]
\bit
\it
For many important distributions, it is not possible to derive explicit (closed-form) expressions for their quantile functions. 
\it
In such cases, critical points must be estimated using numerical approximation techniques. 
\it
For a number of standard distributions, tables of such critical points are available.
\it
These tables list critical points for various values of $p\in[0,1]$.
% are available.
\it 
In recent years, statistical tables have been supplanted by statistical software packages.
\eit
\end{remark}


%======================================================================
\end{document}
%======================================================================
