\documentclass[lecture]{csm}
%\documentclass[blanks,lecture]{csm}

% set meta information
\modulecode{MA1500}
\moduletitle{Introduction to Probability Theory}
\academicyear{2013/14}
\doctype{Lecture}
\doctitle{Events and Probability}
\docnumber{2}

% local
\usepackage{float}
\newcommand{\N}{\mathbb{N}}
\newcommand{\R}{\mathbb{R}}
\newcommand{\prob}{\mathbb{P}}
\newcommand{\expe}{\mathbb{E}}
\def\it{\item}
\def\bit{\begin{itemize}}
\def\eit{\end{itemize}} 
\def\ben{\begin{enumerate}}
\def\een{\end{enumerate}}
\newcommand{\lt}{<}
\newcommand{\gt}{>}

%======================================================================
\begin{document}
\maketitle
\tableofcontents
%======================================================================

\subsection*{History of probability}
Games of chance have been played since antiquity, but the mathematical principles of chance and uncertainty were first established only in the 17th century.

\vspace*{4ex}
\begin{tabular}{lll}
1654		& Classical principles 	& Blaise Pascal (1623--1662) \\
		&						& Pierre de Fermat (1601--1665)  \\
1657		& \textit{De Ratiociniis in Ludo Aleae} & Christiaan Huygens (1629--1695) \\
1713		& \textit{Ars Conjectandi} & Jakob Bernoulli (1654--1705) \\ 
1718		& \textit{The Doctrine of Chances} & Abraham de Moivre (1667--1754) \\
1812		& \textit{Theorie Analytique des Probabilites} & Pierre de Laplace (1749-1827) \\
1919		& Relative frequency & Richard von Mises (1883--1953) \\
1933		& Modern axiomatic theory & Andrey Kolmogorov (1903--1987)
\end{tabular}

%----------------------------------------------------------------------
\section{Random experiments}
%----------------------------------------------------------------------
Scientific investigation relies on repeated experimentation under the same conditions:

% for example:
\begin{tabbing}
\quad Medical research:\quad 	\= The effect of a new treatment on patients. \\
\quad Economics:				\> The price of a commodity at various times. \\
\quad Agriculture: 			\> The effect of a fertiliser on the yield of a crop.
\end{tabbing}

% defn: expt/outcome/sspace
\begin{definition}
\ben
\it
Any process of observation or measurement will be called an \emph{experiment} or \emph{trial}.
\it
Any experiment whose outcome is uncertain is called a \emph{random experiment}.
\it
A random experiment has a set of possible \emph{outcomes}, exactly one of which will occur.
\it
The set of all possible outcomes is called the \emph{sample space}, denoted by $\Omega$.
\it
Outcomes are also called \emph{elementary events}, denoted by $\omega\in\Omega$.
\een
\end{definition}

\break % <<<

% example: sample spaces
\begin{example}
A sample space is the set of all possible outcomes:
\begin{tabbing}
\underline{Random Experiment}\qquad\qquad\qquad\qquad\qquad\qquad\qquad\qquad\qquad \= \underline{Sample space} \\ 
A coin is tossed once.	\> $\Omega = \{H,T\}$ \\
A die is rolled once.	\> $\Omega=\{1,2,3,4,5,6\}$ \\
A coin is tossed repeatedly until a head occurs. \> $\Omega = \{1,2,3,\ldots\}$ \\
The height of a randomly chosen student is measured: \> $\Omega = [0,\infty)$
\end{tabbing}
\end{example}
%==================================================================================================
%\section{Events}
%==================================================================================================
\begin{remark}
Initially we will only consider \emph{finite} sample spaces. 
\end{remark}

% example: die
\begin{example}
A die is rolled once. The sample space can be represented by $\Omega=\{1,2,3,4,5,6\}$.\par
We may be interested in whether or not the following events occur:
\begin{tabbing}
\underline{Event}\qquad\qquad\qquad\qquad\qquad\qquad\qquad\qquad\qquad\qquad\qquad\qquad\=\underline{Subset} \\ 
The outcome is the number $1$.	\> $A = \{1\}$ \\
The outcome is an even number.	\> $A = \{2,4,6\}$ \\
The outcome is even but does not exceed $3$.	\> $A = \{2,4,6\}\cap\{1,2,3\}$ \\
The outcome is not even			\> $A = \Omega\setminus\{2,4,6\}$
\end{tabbing}
\end{example}

%----------------------------------------------------------------------
\section{Events}
%----------------------------------------------------------------------

% defn: events
\begin{definition}
\bit
\it An \emph{event} is a subset of the sample space. 
\it If the outcome $\omega\in\Omega$ occurs, and $\omega\in A$ where $A\subseteq\Omega$, we say that event $A$ \emph{occurs}.
\it Two events $A$ and $B$ with $A\cap B=\emptyset$ are called \emph{disjoint} or \emph{mutually exclusive}.
\it A collection $\{A_1,A_2,\ldots,A_n\}$ is called \emph{pairwise disjoint} if $A_i\cap A_j = \emptyset$ whenever $i\neq j$.
\it The empty set $\emptyset$ is called the \emph{impossible event}.
\it The sample space $\Omega$ is called the \emph{certain event}.
\eit
\end{definition}

\begin{remark}
\bit
\it If $A$ occurs and $A\subseteq B$, then $B$ also occurs.
\it If $A$ occurs and $A\cap B=\emptyset$, then $B$ does not occur. 
\it Not all subsets are necessarily events (see later).
\eit
\end{remark}

\begin{definition}
Let $\Omega$ be any set. The set of all subsets of $\Omega$ is called its \emph{power set}.
%, denoted by $\mathcal{P}(\Omega) = \{A:A\subseteq\Omega\}$.
\end{definition}

%----------------------------------------------------------------------
\section{Probability}
%----------------------------------------------------------------------
%Loosely speaking, the \emph{probability} of an event is a number between zero and one, that represents how likely the event is to occur when the associated random experiment is performed.
\begin{definition}
Let $\Omega$ be a finite sample space, and let $\mathcal{P}(\Omega)$ denote its power set.
\ben
\it A \emph{probability mass function} on $\Omega$ is a function $p:\Omega\to[0,1]$ with the property that
\[
\sum_{\omega\in\Omega} p(\omega) = 1.
\]
\it A \emph{probability measure} on $\Omega$ is a function $\prob:\mathcal{P}(\Omega)\to[0,1]$ defined by
\[
\prob(A) = \sum_{\omega\in A} p(\omega)
\]
where $p(\omega)$ is a probability mass function on $\Omega$.
\it The pair $(\Omega,\prob)$ is called a \emph{(finite) probability space}.
\een
\end{definition}

\begin{remark}
\bit
\it The number $p(\omega)$ is called the \emph{probability} of outcome $\omega\in\Omega$.
\it The number $\prob(A)$ is called the \emph{probability} of event $A\in\mathcal{F}$.
\eit
\end{remark}


%\it The \emph{probability} of an event $A\subseteq\Omega$ is the sum of the probabilities of the outcomes it contains,
%\[
%\prob(A) = \sum_{\omega\in A} p(\omega).
%\]

\break % <<

% example: die
\begin{example}
Consider a random experiment in which a fair six-sided die is rolled once.
\bit
\it A suitable sample space is $\Omega=\{1,2,3,4,5,6\}$.
\it Since the die is fair, the associated probability mass function is
\[\begin{array}{rcl}
p:	\Omega & \to & [0,1] \\
	\omega & \mapsto & 1/6.
\end{array}\] 
%\it The probability that various events occur can be found:
\eit
\begin{tabbing}
\underline{Event}\qquad\qquad\qquad\qquad\qquad\qquad\qquad\qquad\qquad
	\=\underline{Subset}\qquad\qquad\qquad\qquad\qquad
	\=\underline{Probability} \\
The outcome is the number $1$.	\> $A = \{1\}$ \> $\prob(A) = 1/6$\\
The outcome is an even number.	\> $A = \{2,4,6\}$  \> $\prob(A) = 3/6$\\
The outcome is even but does not exceed $3$.	\> $A = \{2,4,6\}\cap\{1,2,3\}$  \> $\prob(A) = 1/6$\\
The outcome is not even			\> $A = \Omega\setminus\{2,4,6\}$ \> $\prob(A) = 3/6$
\end{tabbing}
\end{example}

\begin{exercise}
Suppose instead that the die is \emph{biased}, with probability mass function given by $p(\omega)=1/8$ for $\omega\in\{1,2,3,4\}$, and $p(\omega)=1/4$ for $\omega\in\{5,6\}$. Check that $p$ is indeed a probability mass function, and compute the probabilites of the above events when the die is rolled once.
\end{exercise}


\break % <<
%This definition of probability has nice properties.
\begin{theorem}
\ben
\it $\prob(\Omega)=1$.
\it $\prob(A^c) = 1 - \prob(A)$.
\it If $A$ and $B$ are disjoint, then $\prob(A\cup B) = \prob(A)+\prob(B)$.
\een
\end{theorem}

\begin{proof}
\ben
\it By definition, $\displaystyle \prob(\Omega) = \sum_{\omega\in\Omega} p(\omega) = 1$.
\it Since $A^c=\Omega\setminus A$, $$\displaystyle \prob(A^c) 
	= \sum_{\omega\in A^c}p(\omega) 
	= \sum_{\omega\in\Omega}p(\omega) - \sum_{\omega\in A}p(\omega)
	= 1 - \prob(A).$$
\it Since $A\cap B=\emptyset$, $$\displaystyle \prob(A\cup B) 
	= \sum_{\omega\in A\cup B}p(\omega) 
	= \sum_{\omega\in A}p(\omega) + \sum_{\omega\in B}p(\omega)
	= \prob(A) + \prob(B).$$
\een
\end{proof}

\break % <<

%The following results are direct consequences of the previous theorem.
\begin{corollary}
\ben
\it $\prob(\emptyset)=0$.
\it If $A\subseteq B$ then $\prob(A)\leq \prob(B)$.
\it $\prob(A\cup B) = \prob(A)+\prob(B)-\prob(A\cap B)$.
\een
\end{corollary}

\begin{proof}
\ben
\it Since $\emptyset=\Omega^c$, we have that $\prob(\emptyset) = \prob(\Omega^c) = 1 - \prob(\Omega) = 1 - 1 = 0$.
\it If $A\subseteq B$ then $$\displaystyle \prob(B) 
	= \sum_{\omega\in B}p(\omega) 
	= \sum_{\omega\in A}p(\omega) + \sum_{\omega\in B\setminus A}p(\omega) 
	\geq \sum_{\omega\in A}p(\omega)
	= \prob(A).$$
\it For any two sets $A$ and $B$,
\begin{align*}
\prob(A\cup B) 
	& = \sum_{\omega\in A\setminus B}p(\omega) + \sum_{\omega\in A\cap B}p(\omega) + \sum_{\omega\in B\setminus A}p(\omega) \\
	& = 	\left(\sum_{\omega\in A}p(\omega)-\sum_{\omega\in A\cap B}p(\omega)\right)
		 + \sum_{\omega\in A\cap B}p(\omega)
		 + \left(\sum_{\omega\in B}p(\omega) - \sum_{\omega\in A\cap B}p(\omega)\right) \\%[0.5ex]
	& =  \prob(A)+\prob(B)-\prob(A\cap B).
\end{align*}
\een
\end{proof}

%\begin{corollary}
%Let $A_1,A_2,\ldots,A_n$ be a collection of pairwise disjoint events. Then
%\[
%\prob\left(\bigcup_{i=1}^n A_i\right) = \sum_{i=1}^n \prob(A_i).
%\]
%\end{corollary}

% handout
\subsection*{Correspondence between set theory and probability theory}
%\begin{table}[htb]
%\centering
\vspace*{3ex}
\begin{center}
\begin{tabular}{|c|l|l|} \hline
Notation 			& Set theory				& Probability theory \\ \hline
$\Omega$				& Universal set			& Sample space \\ 
$\omega\in\Omega$	& Element of $\Omega$	& Elementary event, outcome \\
$A\subseteq\Omega$	& Subset of $\Omega$		& Event $A$ \\
$A\subseteq B$		& Inclusion				& If $A$ occurs, then $B$ occurs \\
$A\cup B$			& Union					& $A$ or $B$ (or both) occurs \\ 
$A\cap B$			& Intersection			& $A$ and $B$ occur\\ 
$A^c$				& Complement of $A$		& $A$ does not occur \\
$A\setminus B$		& Difference				& $A$ occurs, but $B$ does not \\
$A\bigtriangleup B$	& Symmetric difference	& $A$ or $B$ occurs, but not both \\
$\emptyset$			& Empty set 				& Impossible event \\
$\Omega$				& Universal set			& Certain event \\ \hline
\end{tabular}
\end{center}
%\caption{Table of correspondence (Grimmett \& Stirzaker 2001).}
%\end{table}

%======================================================================
\end{document}
%======================================================================
