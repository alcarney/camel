\documentclass[lecture]{csm}

% set meta information
\modulecode{MA1500}
\moduletitle{Introduction to Probability Theory}
\academicyear{2013/14}
\doctype{Lecture}
\doctitle{Normal Approximation}
\docnumber{21}

% local
\newcommand{\prob}{\mathbb{P}}
\newcommand{\expe}{\mathbb{E}}
\newcommand{\var}{\text{Var}}
\newcommand{\N}{\mathbb{N}}
\newcommand{\R}{\mathbb{R}}
\def\it{\item}
\def\bit{\begin{itemize}}
\def\eit{\end{itemize}} 
\def\ben{\begin{enumerate}}
\def\een{\end{enumerate}}
\newcommand{\lt}{<}
\newcommand{\gt}{>}

\newcommand{\proofomitted}{\par[\textit{Proof omitted.}]\par}

%======================================================================
\begin{document}
\maketitle
\tableofcontents
%======================================================================


%----------------------------------------------------------------------
\section{Approximation of discrete distributions by continuous distributions}
%----------------------------------------------------------------------
Recall the PMFs of the binomial and Poisson distributions:
\bit
\it If $X\sim\text{Binomial}(n,p)$ then 
\[
\prob(X=k)=\displaystyle\frac{n!}{(n-k)!k!} p^k(1-p)^{n-k} \text{\quad for $k=0,1,2,\ldots,n$}.
\]
%\bit
%\it $\expe(X)=np$ and $\var(X)=np(1-p)$.
%\eit
\it If $X\sim\text{Poisson}(\lambda)$ then 
\[
\prob(X=k)=\displaystyle\frac{\lambda^k}{k!}e^{-\lambda} \text{\quad for $k=0,1,2,\ldots$}.
\]
%\bit
%\it $\expe(X)=\lambda$ and $\var(X)=\lambda$.
%\eit
\eit

If $k$ is large, computing $k!$ can take a long time. To avoid this, we make use of the fact that, under certain conditions, both the binomial and Poisson distributions can be approximated by the normal distribution.

\newpage
%Recall the following:
Binomial distribution:
\bit
\it If $X\sim\text{Binomial}(n,p)$ then %$\expe(X)=np$ and $\var(X)=np(1-p)$.
\[
\expe(X)=np \text{\quad and\quad} \var(X)=np(1-p).
\]
%\eit
%
%\bit
\it If both $np$ and $n(1-p)$ are sufficiently large,
\[
X\sim\text{N}\big(np,np(1-p)\big)\quad\text{approx.}
\]
\eit

\vspace*{2ex}
Poisson distribution:
\bit
\it If $X\sim\text{Poisson}(\lambda)$ then %$\expe(X)=\lambda$ and $\var(X)=\lambda$.
\[
\expe(X)=\lambda \text{\quad and\quad}\var(X)=\lambda.
\]
%\eit
%
%%\vspace*{2ex}
%\bit
\it If $\lambda$ is sufficiently large,
\[
X\sim\text{N}(\lambda,\lambda)\quad\text{approx.}
\]
\eit

%----------------------------------------------------------------------
\newpage
\subsection{The continuity correction}
%----------------------------------------------------------------------

% continuity correction
\begin{definition}[The continuity correction]
Let $X$ be a discrete random variable, taking values in the set $\{0,\pm 1,\pm 2,\ldots\}$. If the distribution of a continuous random variable $Y$ is taken as an approximation to the distribution of $X$, we set
\[
\prob(X=k) = \prob\left(k - \frac{1}{2} < Y < k + \frac{1}{2}\right).
\] 
\end{definition}

In particular,
\bit
\it $\prob(X\lt  k) = \prob(Y\leq k-1/2)$, 
\it $\prob(X\leq k) = \prob(Y\leq k+1/2)$,
\it $\prob(X\geq k) = \prob(Y\geq k-1/2)$,
\it $\prob(X\gt  k) = \prob(Y\geq k+1/2)$
\eit

%----------------------------------------------------------------------
\newpage
\section{Normal approximation of the binomial distribution}
%----------------------------------------------------------------------

% theorem: normal approx to binomial
\begin{theorem}
If $X\sim\text{Binomial}(n,p)$ where $p\in(0,1)$, then
\[
\prob(X=k)\to \frac{1}{\sqrt{2\pi np(1-p)}}\int_{k-1/2}^{k+1/2} \exp\left[-\frac{1}{2}\left(\frac{x-np}{\sqrt{np(1-p)}}\right)^2\right]\,dx 
\quad\text{as}\quad n\to\infty.
\]
\end{theorem}
\proofomitted

\newpage

% example
\begin{example}
A fair coin is tossed $500$ times. What is the probability that $270$ heads are obtained?
\end{example}

\begin{solution}
Let $X$ be the number of heads obtained. Then $X\sim\text{Binomial}(500,270)$, so 
\[
\prob(X > 270) = \sum_{k=271}^{n}\binom{500}{k}\left(\frac{1}{2}\right)^{500}
\]
This is difficult to compute (the exact value is $0.0333$).

Using the normal approximation, the associated normal variable is $Y\sim\text{N}(250,125)$, so
\begin{align*}
\prob(X\geq 271)
	& \approx\prob(Y > 270.5) \\
	& = 1 - \prob(Y \leq 270.5) \\
	& = 1 - \prob\left(Z \leq \frac{270.5 - 250}{\sqrt{125}}\right) \qquad\text{where}\quad Z\sim\text{N}(0,1), \\
	& = 1 - \Phi(1.8336) \\
	& = 0.0334 \qquad\text{(from tables)}.
\end{align*}
\end{solution}


%----------------------------------------------------------------------
\newpage
\section{Normal approximation of the Poisson distribution}
%----------------------------------------------------------------------

% theorem: normal approx to Poisson
\begin{theorem}
If $X\sim\text{Poisson}(\lambda)$, then
\[
\prob(X=k)\to \frac{1}{\sqrt{2\pi\lambda}}\int_{k-1/2}^{k+1/2} \exp\left[-\frac{1}{2}\left(\frac{x-\lambda}{\sqrt{\lambda}}\right)^2\right]\,dx 
\quad\text{as}\quad \lambda\to\infty.
\]
\end{theorem}
\proofomitted

% example
\begin{example}
A newsagent knows from past experience that the weekly demand for a certain magazine has Poisson distribution with mean $20$. How many copies of the magazine should the newsagent stock in order to satisfy the weekly demand with probability $0.95$?
\end{example}

\newpage

\begin{solution}
Suppose the newsagent stocks $N$ copies of the magazine. Let $X\sim\text{Poisson}(20)$ represent the weekly demand. We need that 
\[
\prob(\text{demand satisfied}) = \prob(X\leq N) = \sum_{k=0}^{N}\frac{20^k}{k!}e^{-20} = 0.95.
\]
Using the normal approximation, the associated normal variable is $Y\sim\text{N}(20,20)$, so 
\begin{align*}
\prob(X\leq N) = 0.95
	& \quad\Rightarrow\quad \prob(Y \leq N + 0.5) = 0.95 \\
	& \quad\Rightarrow\quad \prob\left(Z \leq \frac{N + 0.5 - 20}{\sqrt{20}}\right) = 0.95 \qquad\text{where}\quad Z\sim\text{N}(0,1), \\
	& \quad\Rightarrow\quad \Phi\left(\frac{N -19.5}{\sqrt{20}}\right) = 0.95 \\
	& \quad\Rightarrow\quad \frac{N -19.5}{\sqrt{20}} = 1.645 \qquad\text{(from tables: $z_{0.95} = 1.645$)},\\
	& \quad\Rightarrow\quad N = 27 \qquad\text{(to the nearest integer)}.
\end{align*}
Thus the newsagent should stock 27 copies of the magazine.
\end{solution}


%======================================================================
\end{document}
%======================================================================
