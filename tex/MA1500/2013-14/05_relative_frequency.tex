\documentclass[lecture]{csm}
%\documentclass[blanks,lecture]{csm}

% set meta information
\modulecode{MA1500}
\moduletitle{Introduction to Probability Theory}
\academicyear{2013/14}
\doctype{Lecture}
\doctitle{Frequentist Probability}
\docnumber{5}

% local
\newcommand{\R}{\mathbb{R}}
\newcommand{\prob}{\mathbb{P}}
\def\it{\item}
\def\bit{\begin{itemize}}
\def\eit{\end{itemize}} 
\def\ben{\begin{enumerate}}
\def\een{\end{enumerate}}
%\newcommand{\pounds}{{�}}

%======================================================================
\begin{document}
\maketitle
\tableofcontents
%======================================================================

%----------------------------------------------------------------------
\section{Relative frequency}
%----------------------------------------------------------------------
So far, we have defined probability to be a number between $0$ and $1$ that represents how likely an outcome or event is to occur.

\bit
\it The \emph{symmetry} that exists in many problems (e.g. those involving dice and cards) allows us to choose sensible values for these probabilities.
\it How can we define probability in more general scenarios?
\eit

Suppose that a random experiment is repeated many times under the same conditions.
\bit
\it Let $N$ be the number of times that the experiment is repeated. 
\it Let $N(A)$ be the number of times that event $A$ occurs. 
\eit

\begin{definition}
The ratio $N(A)/N$ is called the \emph{relative frequency} of event $A$. 
\end{definition}

\bit
\it The relative frequency of event $A$ is the number of times it occurs expressed as a proportion of the total number of repetitions, 
\eit

\newpage % <<

% defn: frequentist probability
\begin{definition}[Frequentist model]
Under the \emph{frequentist model}, the probability of an event $A$ is defined to be the limit of its relative frequency as the number of trials increases without bound:
\[
P(A) = \lim_{N\to\infty} \frac{N(A)}{N}.
\]
\end{definition}

\bit
%\it This is called the \emph{frequentist} model of probability.
\it The frequentist model is dominant in many areas of science (e.g.\ clinical trials).
\it In practical applications, a large number of random experiments is performed, and the relative frequency of an event is taken as an approximation of its ``true'' probability.
\eit

\vspace{2ex}
Frequentist probability has nice properties:
\bit
\it $0\leq P(A)\leq 1$ for every event $A\in\mathcal{P}(\Omega)$.
\it $P(\emptyset)=0$, because $N(\emptyset)=0$ for any number of repeteions.
\it $P(\Omega)=1$, because $N(\Omega)=N$ for any number of repeteions.
\it The \emph{complementarity} property:
\[
P(A^c) = \frac{N(A^c)}{N}= \frac{N-N(A)}{N} = 1-\frac{N(A)}{N} = 1 - P(A).
\]

\newpage

\it The \emph{additive} property:
If $A$ and $B$ are two disjoint events, then
\[
\frac{N(A\cup B)}{N} 
	= \frac{N(A)+N(B)}{N} = \frac{N(A)}{N} + \frac{N(B)}{N}
\]
so $P(A\cup B)=P(A)+P(B)$.
\par
More generally, if $A_1,A_2,\ldots,A_n$ are pairwise disjoint events, then
\begin{align*}
\frac{N(A_1\cup A_2\cup\ldots\cup A_n)}{N} 
	& = \frac{N(A_1)+N(A_2)+\ldots+N(A_n)}{N} \\
	& = \frac{N(A_1)}{N} + \frac{N(A_2)}{N}+\ldots+\frac{N(A_n)}{N}
\end{align*}
so $P(A_1\cup A_2\cup\ldots\cup A_n)=P(A_1)+P(A_2)+\ldots+P(A_n)$.
\eit
%----------------------------------------------------------------------
\newpage
\section{Conditional probability}
%----------------------------------------------------------------------
Conditional probability is a natural consequence of the frequentist model.
\bit
\it Let $N$ be the number of times the experiment is repeated. 
\it Let $N(A)$ be the number of times that event $A$ occurs. 
\it Let $N(B)$ be the number of times that event $B$ occurs. 
\it Let $N(A,B)$ be the number of times that events $A$ and $B$ both occur. 
\eit

Intuitively, the conditional probability that $A$ occurs given that $B$ occurs, is the number of trials in which $A$ and $B$ both occur as a proportion of the number of trials in which $B$ occurs:
\[
P(A|B)	= \lim_{N\to\infty}\frac{N(A,B)}{N(B)} 
		= \lim_{N\to\infty}\frac{N(A,B)/N}{N(B)/N} 
		= \frac{P(A\cap B)}{P(B)},
\]
which agrees with the definition of conditional probability given previously.
  
%----------------------------------------------------------------------
\newpage
\section{Infinite sample spaces}
%----------------------------------------------------------------------
The frequentist model does not extend to experiments that have infinitely many outcomes.

\begin{example}\label{ex:infinitesamplespace}
Consider a random experiment in which a coin is tossed repeatedly until the first head occurs. 
\bit
\it Let the outcome of the experiment be the number of times that the coin is tossed.
\eit
\par
The sample space for this experiment is an infinite set:
\bit
\it $\Omega = \{H,TH,TTH,TTTH,TTTTH,TTTTTH,\ldots\}$, or alternatively,
\it $\Omega = \{1,2,3,4,5,\ldots\}$.
\eit
\end{example}

\vspace*{2ex}
For any fixed number of repetitions, there are plausible outcomes that will not occur. 
\bit
\it How can we define their probability in terms of relative frequeency?
\eit

\vspace*{2ex}
Later on, we will define probability as a \emph{measure} on specific collections of events.
\bit
\it Our starting point will be a minimal set of properties that such measures should possess.
\eit

%\break % <<
%
%\begin{examplecont}{\ref{ex:infinitesamplespace}}
%Let $p$ be is the probability of observing a head in any particular trial.  
%If we assume that the trials are independent, we can assign a probability to each outcome:
%\[
%P(\{H\})= p,\quad P(\{TH\})=p(1-p), \quad P(\{TTH\}) = p(1-p)^2, \quad\text{and so on.}
%\]
%When $0<p<1$, each of these probabilities is non-zero, and their sum is equal to 1:
%\begin{align*}
%P(\{H\}) + P(\{TH\}) + P(\{TTH\}) + \ldots
%	& = p + p(1-p) + p(1-p)^2 + \ldots \\
%	& = p\sum_{k=0}^{\infty}(1-p)^k \\
%	& = \frac{p}{1-(1-p)} = 1.
%\end{align*}
%\end{examplecont}
%
%This is all very well, but what does ``the probability of observing a head'' actually mean?

%%----------------------------------------------------------------------
%\section{The Bayesian model}
%%----------------------------------------------------------------------
%\subsubsection*{Frequentist model (Neymann/Pearson/Wald)}
%
%Data are variable, parameters are fixed:
%\bit
%\it Data are observed from repeatable random experiments.
%\it Unknown parameters are assumed to be constant across repetitions.
%\eit
%
%\subsubsection*{Bayesian model (Bayes/Laplace/de Finetti)}
%
%Data are fixed, parameters are variable
%\bit
%\it Data are observed from a realized sample.
%\it Unknown parameters are described probabilistically.
%\eit

%======================================================================
\end{document}
%======================================================================
