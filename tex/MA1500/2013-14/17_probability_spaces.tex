\documentclass[lecture]{csm}

% set meta information
\modulecode{MA1500}
\moduletitle{Introduction to Probability Theory}
\academicyear{2013/14}
\doctype{Lecture}
\doctitle{Probability Spaces}
\docnumber{17}

% local
\newcommand{\prob}{\mathbb{P}}
\newcommand{\expe}{\mathbb{E}}
\newcommand{\N}{\mathbb{N}}
\newcommand{\R}{\mathbb{R}}
\def\it{\item}
\def\bit{\begin{itemize}}
\def\eit{\end{itemize}} 
\def\ben{\begin{enumerate}}
\def\een{\end{enumerate}}
\newcommand{\lt}{<}
\newcommand{\gt}{>}

%======================================================================
\begin{document}
\maketitle
\tableofcontents
%======================================================================

%----------------------------------------------------------------------
\section{Probability measures}
%----------------------------------------------------------------------

% defn: probability measures, probability spaces
\begin{definition}
Let $\Omega$ be any set, and let $\mathcal{F}$ be a $\sigma$-field over $\Omega$. 
%\par
A \emph{probability measure} on $(\Omega,\mathcal{F})$ is a function $\prob:\mathcal{F}\to[0,1]$ with the following properties:
\ben
\it $\prob(\Omega) = 1$,
\it For any countable collection of pairwise disjoint events $\{A_1,A_2,\ldots\}$,
\[
\prob\left(\bigcup_{i=1}^\infty A_i\right) = \sum_{i=1}^{\infty} \prob(A_i).
\text{\qquad (Countable Additivity)}
\]
\een
The triple $(\Omega,\mathcal{F},\prob)$ is called a \emph{probability space}.
\end{definition}

%\begin{remark}
%\bit
%%\it The first property means that probability is a \emph{finite measure}.
%%\it The second property is called \emph{complementarity}.
%\it The third property is called \emph{countable additivity}.
%\eit
%\end{remark}

% remark: measure theory
\begin{remark}
In the more general setting of measure theory,
\bit
\it the sets $A\in\mathcal{F}$ are called \emph{measurable sets},
\it the pair $(\Omega,\mathcal{F})$ is called a \emph{measurable space},
\it the triple $(\Omega,\mathcal{F},\prob)$ is called a \emph{measure space}.
\eit
\end{remark}

% example
\begin{example}
A fair six-sided die is rolled once. 
\ben
\it A probability space $(\Omega,\mathcal{F}, \prob)$ for the experiment is given by
\bit
\it $\Omega=\{1,2,3,4,5,6\}$,
\it $\mathcal{F} = \mathcal{P}(\Omega)$ is the power set of $\Omega$,
\it $\prob(A)=|A|/|\Omega|$ for every $A\in\mathcal{F}$.
\eit

%\vspace*{2ex}
\it If we are only interested in whether the outcome is an odd or even number, we can instead take
\bit
\it $\Omega=\{1,2,3,4,5,6\}$,
\it $\mathcal{F} = \big\{\emptyset,\{1,3,5\},\{2,4,6\},\Omega\big\}$
\it $\prob(\emptyset)=0$, $\prob(\{1,3,5\})=1/2$, $\prob(\{2,4,6\})=1/2$, $\prob(\Omega)=1$.
\eit
\een
\end{example}

%% exercise
%\begin{exercise}
%In each case of the the previous example, verify that $(\Omega,\mathcal{F},\prob)$ is indeed a probability space by checking the conditions required for $\mathcal{F}$ to be a $\sigma$-field and $\prob$ to be a probability measure.
%\end{exercise}

%----------------------------------------------------------------------
\newpage
\section{Null and almost-certain events}
%----------------------------------------------------------------------

% definition: null and a.s. events
\begin{definition}
\ben
\it If $\prob(A)=0$, we say that $A$ is a \emph{null event}.
\it If $\prob(A)=1$, we say that $A$ occurs \emph{almost surely} or \emph{with probability one}.
\een
\end{definition}

% remark: null and a.s. events
\begin{remark}
\bit
\it A null event is not the same as the impossible event ($\emptyset$). 
\it An event that occurs almost surely is not the same as the certain event ($\Omega$).
\eit
\end{remark}

% example
\begin{example}
A dart is thrown at a dartboard.
\bit
\it The probability that the dart hits a given point of the dartboard is $0$.
\it The probability that the dart does \emph{not} hit a given point of the dartboard is $1$.
\eit
\end{example}

%----------------------------------------------------------------------
\newpage
\section{Properties of probability measures}
%----------------------------------------------------------------------

% theorem: properties of probability measures
\begin{theorem}%[Properties of probability measures]\label{thm:properties_of_probability_measures}
Let $(\Omega,\mathcal{F},\prob)$ be a probability space, and let $A,B\in\mathcal{F}$. 
\ben
\it $\prob(\emptyset) = 0$,
\it Complementarity: $\prob(A^c) = 1 - \prob(A)$.
\it Monotonicity: if $A\subseteq B$ then $\prob(A)\leq \prob(B)$.
\it Addition rule: $\prob(A\cup B) = \prob(A) + \prob(B) - \prob(A\cap B)$.
%\it The inclusion-exclusion principle:
%\[
%\prob(A\cup B\cup C) = \prob(A) + \prob(B) + \prob(C) - \prob(A\cap B) - \prob(A\cap C) - \prob(B\cap C) + \prob(A\cap B\cap C)
%\]
\een
\end{theorem}

% proof
\begin{proof}
\ben

\it % emptyset
$\emptyset$ and $\Omega$ are disjoint sets, so by (countable) additivity,
\[
\prob(\emptyset) + \prob(\Omega) = \prob(\emptyset\cup\Omega) = \prob(\Omega),
\]
which can only be true if $\prob(\emptyset) = 0$.

\newpage

\it % complementarity
Complementarity: $A$ and $A^c$ are disjoint sets with $A\cup A^c =\Omega$. By the (countable) additivity of probability measures,
\[
\prob(A) + \prob(A^c) = \prob(A\cup A^c) = \prob(\Omega) = 1,
\]
so $\prob(A^c) = 1 - \prob(A)$,

\it % monotonicity
Monotonicity: Let $A\subseteq B$ and consider the disjoint union $B = A\cup (B\setminus A)$. By the (countable) additivity of probability measures,
\[
\prob(B) = \prob\big[A\cup (B\setminus A)\big] = \prob(A) + \prob(B\setminus A),
\]
and since $\prob(B\setminus A)\geq 0$, it follows that $\prob(B) \geq \prob(A)$.

\it % addition rule
Addition rule: Consider the disjoint union $A\cup B = (A\setminus B) \cup (A\cap B) \cup (B\setminus A)$. By the (countable) additivity of probability measures,
\begin{align*}
\prob(A\cup B) 	& = \prob(A\setminus B) + \prob(B\setminus A) + \prob(A\cap B), \text{\ and similarly} \\
\prob(A) 		& = \prob(A\setminus B) + \prob(A\cap B), \text{\ and} \\
\prob(B)	 		& = \prob(B\setminus A) + \prob(A\cap B).
\end{align*}
Combining these equations, we obtain $\prob(A\cup B) = \prob(A) + \prob(B) - \prob(A\cap B)$.

\een
\end{proof}

\newpage % <<

%----------------------------------------------------------------------
\newpage
%\section{The inclusion-exclusion principle}
%----------------------------------------------------------------------

% theorem: properties of probability measures
\begin{theorem}[The inclusion-exclusion principle]\label{thm:inclusion-exclusion}
Let $(\Omega,\mathcal{F},\prob)$ be a probability space, and let $A,B,C\in\mathcal{F}$. Then
%\[
%\prob(A\cup B\cup C) = \prob(A) + \prob(B) + \prob(C) - \prob(A\cap B) - \prob(A\cap C) - \prob(B\cap C) + \prob(A\cap B\cap C)
%\]
\begin{align*}
\prob(A\cup B\cup C) 
	& = \prob(A) + \prob(B) + \prob(C) \\
	& \qquad - \prob(A\cap B) - \prob(A\cap C) - \prob(B\cap C) + \prob(A\cap B\cap C)
\end{align*}
\end{theorem}

\begin{proof}
Set union is associative: $A\cup B\cup C = (A\cup B)\cup C$, so by the addition rule,
\begin{align}
\prob(A\cup B\cup C) 	
	& = \prob\big((A\cup B)\cup C\big) \nonumber \\
	& = \prob(A\cup B) + \prob(C) - \prob\big((A\cup B)\cap C\big). \tag{$\ast$}
\end{align}

Set intersection is distributive over set union: $(A\cup B)\cap C = (A\cap C)\cup (B\cap C)$, so by the addition rule,
\begin{align}	
\prob\big[(A\cup B)\cap C\big]
	& = \prob\big[(A\cap C)\cup (B\cap C)\big] \nonumber \\
	& = \prob(A\cap C) + \prob(B\cap C) - \prob\big[(A\cap C)\cap (B\cap C)\big] \nonumber \\
	& = \prob(A\cap C) + \prob(B\cap C) - \prob(A\cap B\cap C). \tag{$\ast\ast$}
\end{align}
The result then follows by ($\ast$) and ($\ast\ast$).
\qed
\end{proof}

%% remark
%\begin{remark}
%The general form of the inclusion-exclusion principle is 
%\begin{align*}
%\prob\left(\bigcup_{i=1}^n A_i\right)	
%	& = \sum_{i} \prob(A_i) - \sum_{i\lt j} \prob(A_i\cap A_j) + \sum_{i\lt j\lt k} \prob(A_i\cap A_j\cap A_k) + \ldots \\
%	& \qquad	 + (-1)^k\sum_{i_1\lt i_2\lt\ldots\lt i_k} \prob(A_{i_1}\cap A_{i_2}\cap\ldots\cap A_{i_k}) + \ldots \\
%	& \qquad\qquad + (-1)^n \prob(A_1\cap A_2\cap\ldots\cap A_n).
%\end{align*}				
%\end{remark}
%

%----------------------------------------------------------------------
\newpage
\section{Subadditivity}
%----------------------------------------------------------------------

%\begin{remark}[Subadditivity]
\bit
\it
For any two events $A,B\in\mathcal{F}$ we have $\prob(A\cap B)\geq 0$.
Hence by the addition rule,
\[
\prob(A\cup B) \leq \prob(A) + \prob(B) \text{\quad for all\quad} A,B\in\mathcal{F}.
\]
This property is called \emph{subadditivity}.
\it
For any countable collection $\{A_1,A_2,\ldots\}$ of events in $\mathcal{F}$, it can be shown that 
\[
\prob\left(\bigcup_{i=1}^{\infty}A_i\right)\leq \sum_{i=1}^{\infty} \prob(A_i).
\]
This property is called \emph{countable subadditivity}.
\it
Note that countable subadditivity holds for \textbf{any} collection of events $\{A_1,A_2,\ldots\}$.
\it
Subadditivity is a powerful property, and is used extensively in probability theory.
\eit

%\end{remark}

%======================================================================
\end{document}
%======================================================================
