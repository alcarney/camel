% !TEX root = main.tex
%----------------------------------------------------------------------
\begin{exercise}
\begin{questions}
%----------------------------------------

%%----------------------------------------
%\question
%Show that $e^x \leq x + e^{x^2}$ for all $x\in\R$.
%
%\begin{answer} 
%For all $k>1$, note that $2k! \leq (2k-2)! < (2k-1)! < (2k)! < (2k+1)!$ 
%\ben
%% case 1
%\it For $x\in (-1,+1)$, we have $x^{2k+1}<x^{2k}$ so %use the inequality %the result follows by the inequality
%\[
%\frac{x^{2k}}{(2k)!} + \frac{x^{2k+1}}{(2k+1)!} \leq \frac{x^{2k}}{k!}\qquad\text{for}\quad k\geq 1.
%\]
%Hence,
%\begin{align*}
%e^x		
%	& = 1 + x + \left(\frac{x^2}{2!} +\frac{x^3}{3!}\right) + \left(\frac{x^4}{4!} + \frac{x^5}{5!}\right) + \ldots \\
%	& \leq 1 + x + \frac{x^2}{1!} + \frac{x^4}{2!} + \ldots \\
%	& = x + \left(1 + x^2 + \frac{x^4}{2!} + \ldots\right) \\
%	& = x + e^{x^2}.
%\end{align*}
%% case 2
%\it For $x\notin (-1,+1)$ we have $x^{2k-2}<x^{2k-1}<x^{2k}$, so
%\[
%\frac{x^{2k-2}}{(2k-2)!} + \frac{x^{2k-1}}{(2k-1)!} \leq \frac{x^{2k}}{k!}\qquad\text{for}\quad k\geq 2.
%\]
%Hence,
%\begin{align*}
%e^x		
%	& = 1 + x + \left(\frac{x^2}{2!} + \frac{x^3}{3!}\right) + \left(\frac{x^4}{4!} + \frac{x^5}{5!}\right) + \ldots \\
%	& \leq 1 + x + \frac{x^4}{2!} + \frac{x^6}{3!} + \ldots \\
%	& \leq x + \left(1 + x^2 + \frac{x^4}{2!} + \frac{x^6}{3!} + \ldots\right) \\
%	& = x + e^{x^2}.
%\end{align*}
%as required.
%\een
%\end{answer}

%----------------------------------------
\question
Let $c$ be a constant, and let $X_1,X_2,\ldots$ be a sequence of random variables with $\expe(X_n) = c$ and $\var(X_n) = 1/\sqrt{n}$ for each $n$. Show that the sequence converges to $c$ in probability as $n\to\infty$.
\begin{answer} % <<<
Let $\epsilon>0$. By Chebyshev's inequality,
\[
\prob\big(|X_n-c|\geq\epsilon\big) \leq \frac{\var(X_n)}{\epsilon^2} = \frac{1}{\epsilon^2\sqrt{n}}
\]
for all $n\in\N$. Thus $\displaystyle \lim_{n\to\infty} \prob\big(|X_n-c|\geq\epsilon\big) = 0$, so the sequence converges to $c$ in probability.
\end{answer}

%----------------------------------------
\question
A fair coin is tossed $n$ times. Does the law of large numbers ensure that the observed number of heads will not deviate from $n/2$ by more than $100$ with probability of at least $0.99$, provided that $n$ is sufficiently large?
\begin{answer} % <<<
Yes, because the indicator variable has finite mean and variance.
\end{answer}

%----------------------------------------
\end{questions}
\end{exercise}
%----------------------------------------------------------------------

%======================================================================
\endinput
%======================================================================
