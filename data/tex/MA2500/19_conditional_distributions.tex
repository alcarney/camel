% !TEX root = main.tex
%----------------------------------------------------------------------
\chapter{Conditional Distributions}\label{chap:cond_dist}
%----------------------------------------------------------------------
%----------------------------------------------------------------------
\section{Conditional distributions}
%----------------------------------------------------------------------
Let $X,Y:\Omega\to\R$ be random variables on a probability space $(\Omega,\mathcal{F},\prob)$.

% defn: conditional distributions
\begin{definition}
\ben
\it % (i)
The \emph{conditional distribution} of $Y$ given $X$ is the function
\[
\begin{array}{cccl}
\prob_{Y|X}:		& \mathcal{B}^2	& \to		& [0,1] \\
				& (A,B)			& \mapsto	& \prob(Y\in B\,|\, X\in A).
\end{array}
\]
\it The \emph{conditional CDF} of $Y$ given $X$ is the function
\[
\begin{array}{cccl}
F_{Y|X}:		& \R^2	& \to		& [0,1] \\
			& (x,y)	& \mapsto	& \prob(Y\leq y\,|\, X\leq x).
\end{array}
\]
\een
\end{definition}

The following lemma is easily proved.
% lemma: quotient
\begin{lemma} 
The conditional CDF of $Y$ given $X$ satisfies
\[
F_{Y|X}(x,y) = \frac{F_{X,Y}(x,y)}{F_X(x)},
\]
where $F_{X,Y}$ is the joint CDF of $X$ and $Y$, and $F_X$ is the marginal CDF of $X$.
\end{lemma}

%\begin{proof}
%\[
%F_{Y|X}(x,y) 
%	= \prob(Y\leq y\,|\,X\leq x)
%	= \frac{\prob(\{Y\leq y\}\cap\{X\leq x\})}{\prob(X\leq x)}
%	= \frac{F_{X,Y}(x,y)}{F_X(x)}.
%\]
%\end{proof}

%----------------------------------------------------------------------
\subsection{Discrete case}
%----------------------------------------------------------------------
% conditional CDF and PMF
\begin{definition}
Let $X$ and $Y$ be jointly discrete random variables, and let $x$ be such that $\prob(X=x)>0$.
%\ben
%\it The \emph{conditional CDF} of $Y$ given $X=x$ is 
%\[
%F_{Y|X}(y|x) = \prob(Y\leq y\,|\,X=x).
%\]
%\it 
The \emph{conditional PMF} of $Y$ given $X=x$ is the function
\[
\begin{array}{cccc}
f_{Y|X}	:	& \R		& \to		& [0,1] \\[1ex]
			& y 		& \mapsto	& \prob(Y=y \,|\, X=x).
\end{array}
\]
%\]The \emph{conditional PMF} of $Y$ given $X=x$ is 
%\[
%f_{Y|X}(y|x) = \prob(Y=y\,|\,X=x).
%\]
%\een
\end{definition}

The following lemma is easily proved.
% lemma
\begin{lemma}
The conditional PMF of $Y$ given $X=x$ satisfies
\[
f_{Y|X}(y|x) = \frac{f_{X,Y}(x,y)}{f_X(x)}.
\]
\end{lemma}

% theorem: law of total probability
\begin{theorem}%[Law of total probability for discrete variables]
If $X$ and $Y$ are jointly discrete random variables, then 
\[
\displaystyle f_Y(y) = \sum_x f_{Y|X}(y|x) f_X(x).
\]
where the sum is taken over the range of $X$.
\end{theorem}
\begin{proof}
\[
\sum_x f_{Y|X}(y|x) f_X(x) 
	= \sum_x \left(\frac{f_{X,Y}(x,y)}{f_X(x)}\right) f_X(x)
	= \sum_x f_{X,Y}(x,y)
	= f_Y(y).
\]
\end{proof}

%----------------------------------------------------------------------
\subsection{Continuous case}
%----------------------------------------------------------------------
% conditional CDF and PDF
Let $X$ and $Y$ be jointly continuous random variables.
\bit
\it Suppose we observe that $X$ takes the value $x$.
\it Since $\prob(X=x)=0$, we cannot condition on the event $\{X=x\}$.
\eit

% definition: conditional density function
\begin{definition}\label{def:conditional_density}
Let $X$ and $Y$ be jointly continuous random variables. The \emph{conditional PDF} of $Y$ given $X=x$ is 
\[
f_{Y|X}(y|x) = \frac{f_{X,Y}(x,y)}{f_X(x)}
\]
\end{definition}

%\newpage

% theorem: LTP
\begin{theorem}%[Law of total probability for continuous variables]
If $X$ and $Y$ are jointly continuous random variables, then 
\[
\displaystyle f_Y(y) = \int_{-\infty}^{\infty} f_{Y|X}(y|x) f_X(x)\,dx.
\]
\end{theorem}

% proof
\begin{proof}
\[
\int f_{Y|X}(y|x) f_X(x)\,dx 
	= \int \left(\frac{f_{X,Y}(x,y)}{f_X(x)}\right) f_X(x)\,dx
	= \int f_{X,Y}(x,y)\,dx
	= f_Y(y).
\]
\end{proof}

%----------------------------------------------------------------------
\section{Conditional expectation}
%----------------------------------------------------------------------
\begin{definition}
\ben
\it The \emph{conditional expectation of $Y$ given $X=x$} is a number,% (which depends on $x$),
\[
\expe(Y|X=x) \ =\  
	\begin{cases}
	\displaystyle\sum_y y\,f_{Y|X}(y|x)								& \text{(discrete case),} \\[4ex]
	\displaystyle\int_{-\infty}^{\infty} y\,f_{Y|X}(y|x)\,dy\quad		& \text{(continuous case).}
	\end{cases}
\]
\it The \emph{conditional expectation of $Y$ given $X$} is a random variable,
\[\begin{array}{llll}
\expe(Y|X):	& \Omega 	& \to 		& \R \\
				& \omega	& \mapsto 	& \expe\big(Y|X=X(\omega)\big).
\end{array}\]
\een
\end{definition}

% remark
\begin{remark}
Let $g(x) = \expe(Y|X=x)$. The distribution of the random variable $g(X)=\expe(Y|X)$ depends only on the distribution of $X$. Its expectation is given by
%\[
%\expe\big[\expe(Y|X)\big] \equiv \expe\big[g(X)\big] = \left\{
%	\begin{array}{lll}
%	\displaystyle\sum_x g(x)f_X(x)							& \equiv \sum_x \expe(Y|X=x)f_X(x)							& \text{discrete case,} \\[4ex]
%	\displaystyle\int_{-\infty}^{\infty} g(x)f_X(x)\,dx	& \equiv \int_{-\infty}^{\infty} \expe(Y|X=x)f_X(x)\,dx	& \text{continuous case.}
%	\end{array}
%	\right.
%\]
\[
\expe\big[\expe(Y|X)\big] \ =\  
	\begin{cases}
	\displaystyle\sum_x \expe(Y|X=x)f_X(x)								& \text{(discrete case),} \\[4ex]
	\displaystyle\int_{-\infty}^{\infty} \expe(Y|X=x)f_X(x)\,dx\quad		& \text{(continuous case).}
	\end{cases}
\]

where $f_X$ is the marginal PMF or PDF of $X$.
\end{remark}



%For discrete and continuous distributions, we can condition on the event $\{X=x\}$ by using the conditional PMF and conditional PDF respectively:
%\ben
%\it % disc
%If $X$ and $Y$ are jointly discrete random variables,
%\[
%\expe(Y|X=x) = \sum y\,f_{Y|X}(y|x)
%\]
%where $f_{Y|X}(\cdot|x)$ is the conditional PMF of $Y$ given $X=x$.
%\it % cts
%If $X$ and $Y$ are jointly continuous random variables,
%\[
%\expe(Y|X=x) = \int y\,f_{Y|X}(y|x)\,dy
%\]
%where $f_{Y|X}(\cdot|x)$ is the conditional PDF of $Y$ given $X=x$.
%\een

%----------------------------------------------------------------------
\section{Law of total expectation}
%----------------------------------------------------------------------

\begin{theorem}[Law of total expectation]
Let $X$ and $Y$ be random variables on the same probability space. Then
\[
\expe\big[\expe(Y|X)\big] = \expe(Y).
\]
\end{theorem}
We prove the theorem for continuous random variables (the discrete case follows similarly).
% proof
\begin{proof}
Let $X$ and $Y$ be jointly continuous random variables.%, and let $g(x) = \expe(Y|X=x)$. Then
%\small
\begin{align*}
\expe\big[\expe(Y|X)\big] 
%	= \int g(x) f_X(x)\,dx
	& = \int_{-\infty}^{\infty} \expe(Y|X=x) f_X(x)\,dx \\
	& = \int_{-\infty}^{\infty}\left(\int_{-\infty}^{\infty} y\,f_{Y|X}(y|x)\,dy\right) f_X(x)\,dx \\
	& = \int_{-\infty}^{\infty}\left(\int_{-\infty}^{\infty} y\,\frac{f_{X,Y}(x,y)}{f_X(x)}\,dy\right) f_X(x)\,dx \\
	& = \int_{-\infty}^{\infty}\left(\int_{-\infty}^{\infty} y f_{X,Y}(x,y)\,dy\right)\,dx \\
	& = \int_{-\infty}^{\infty} y \left(\int_{-\infty}^{\infty} f_{X,Y}(x,y)\,dx\right)\,dy \\
	& = \int_{-\infty}^{\infty} y f_Y(y) \,dy \\
	& = \expe(Y).
\end{align*}
%\normalsize
\end{proof}

%----------------------------------------------------------------------
\section{Law of total variance} 
%----------------------------------------------------------------------
\begin{theorem}[Law of total variance]
Let $X$ and $Y$ be random variables on the same probability space. Then
\[
\var(Y) = \expe\big[\var(Y|X)\big] + \var\big[\expe(Y|X)\big] 
\]
\end{theorem}
This is sometimes called the \emph{variance decomposition formula}.
% proof
\begin{proof}
\bit
\it By the law of total expectation,
\begin{align*}
\var(Y)
	& = \expe(Y^2) - \expe(Y)^2 \\
	& = \expe\big[\expe(Y^2|X)\big] - \expe\big[\expe(Y|X)\big]^2
\end{align*}
\it Because $\var(Y|X)=\expe(Y^2|X) - \expe(Y|X)^2$,
\[
\var(Y) = \expe\big[\var(Y|X) + \expe(Y|X)^2\big] - \expe\big[\expe(Y|X)\big]^2
\]
\it Hence, by the linearity of expectation,
\begin{align*}
\var(Y)
	& = \expe\big[\var(Y|X)\big] + \Big(\expe\big[\expe(Y|X)^2\big] - \expe\big[\expe(Y|X)\big]^2\Big) \\
	& = \expe\big[\var(Y|X)\big] + \var[\expe(Y|X)\big].
\end{align*}
\eit
\vspace*{-2ex}
\end{proof}

%----------------------------------------------------------------------
\subsection{Variance decomposition} 
%----------------------------------------------------------------------
$\expe(Y|X)$ can be thought of as a \emph{model} of $Y$ in terms of $X$.
\bit
\it $\var\big[\expe(Y|X)\big]$ is the variance of the model. This is called the \emph{explained variance}.
\it $Y - \expe(Y|X)$ is called the \emph{residual}, representing that part of $Y$ not explained by the model $\expe(Y|X)$.
\it $ \var(Y|X) = \expe\big(\,[Y-\expe(Y|X)]^2\,\big|\,X\,\big)$ is called the \emph{residual variance} at $X$.
\it $\expe\big[\var(Y|X)\big]$ is the expected residual variance. This is called the \emph{unexplained variance}.
\eit

\vspace*{2ex}
The law of total variance divides the variance into \emph{unexplained} and \emph{explained} components:
\[
\var(Y) = \expe\big[\var(Y|X)\big] + \var\big[\expe(Y|X)\big] 
\]
or
\[
\frac{\expe\big[\var(Y|X)\big]}{\var(Y)} + \frac{\var\big[\expe(Y|X)\big]}{\var(Y)} = 1.
\]
This idea is important in statistics.

%----------------------------------------------------------------------
\subsection{Linear models} 
%----------------------------------------------------------------------
Suppose we adopt a \emph{linear model} of $Y$ against $X$:
\[
\expe(Y|X) = a + bX.
\]
It can be shown that the residual variance is minimised when
\[
a = \expe(Y) - \left[\frac{\cov(X,Y)}{\var(X)}\right] \expe(X)
\text{\qquad and\qquad} 
b = \frac{\cov(X,Y)}{\var(X)}.
\]
The proportion of the total variance explained by the model is the square of the correlation coefficient:
\[
\frac{\var\big[\expe(Y|X)\big]}{\var(Y)} = \rho(X,Y)^2.
\]
This is known as the \emph{coefficient of determination}, usually denoted by $R^2$, which quantifies the extent to which a linear model $Y=a+bX$ captures the relationship (if any) between $X$ and $Y$.

%----------------------------------------------------------------------
\section{Example}
%----------------------------------------------------------------------
\begin{example}
The jointly continuous random variables $X$ and $Y$ have following joint PDF:
\[
f(x,y) = 
\begin{cases}
	\frac{21}{4}x^2y		& \text{for } x^2<y<1 \\
	0									& \text{otherwise}
\end{cases}
\]
\ben
\it Find the marginal PDFs of $X$ and $Y$.
\it Find the mean and variance of $Y$.
\it Find the conditional PDF of $X$ given $Y=y$.
\it Find the conditional PDF of $Y$ given $X=x$. 
\it Are $X$ and $Y$ independent? 
\it Verify that $\expe(Y)=\expe\big[\expe(Y|X)\big]$.
%\it Verify that $\var(Y)=\expe\big[\var(Y|X)\big] + \var\big[\expe(Y|X)\big]$.
\een
\end{example}

\begin{solution}
The support of the joint PDF $f(x,y)$ is the set $\{(x,y): x^2 < y < 1\}$. 
\bit
\it This is the region between the vertical lines $x=-1$ and $x=+1$, bounded above by the horizontal line $y=1$ and below by parabola $y=x^2$.
\eit
In particular,
\bit
\it For fixed $x\in[-1,1]$, we must have $y\in[x^2,1]$.
\it For fixed $y\in[0,1]$, we must have $x\in[-\sqrt{y},+\sqrt{y}]$.
\eit
 
\ben
\it % << (a)
The marginal distributions are computed as follows:
\begin{align*}
f_X(x) 	
	& = \int_{-\infty}^{\infty} f(x,y)\,dy 
	= \int_{x^2}^1 \frac{21}{4}x^2y\,dy 
	= \frac{21}{4}x^2\left[\frac{y^2}{2}\right]_{x^2}^1 
	= \begin{cases} \frac{21}{8}x^2(1-x^4)	& -1<x<1, \\ 0 & \text{ otherwise.}\end{cases} \\
f_Y(y) 	
	& = \int_{-\infty}^{\infty} f(x,y)\,dx 
	= \int_{-\sqrt{y}}^{\sqrt{y}} \frac{21}{4}x^2y\,dx 
	= \frac{21}{4}y\left[\frac{x^3}{3}\right]_{-\sqrt{y}}^{\sqrt{y}}
	= \begin{cases} \frac{7}{2}y^{5/2} & 0< y< 1, \\ 0 & \text{ otherwise.}\end{cases} \\
\end{align*}

\it % << (b)
The expected value and variance of $Y$ are computed as follows:
\begin{align*}
\expe(Y)
	& = \int_0^1 y\left(\frac{7y^{5/2}}{2}\right)\,dy = \frac{7}{9}, \\
\expe(Y^2)
	& = \int_0^1 y^2\left(\frac{7y^{5/2}}{2}\right)\,dy = \frac{7}{11}, \\
\var(Y)
	& = \expe(Y^2) - \expe(Y)^2 = \frac{7}{11} - \frac{49}{81} = \frac{28}{891}.
\end{align*}

\it % << (c)
The conditional PDF of $X$ given that $Y=y$ is
\[
f_{X|Y}(x\,|\,y)
	= \frac{f_{X,Y}(x,y)}{f_Y(y)} 
	= \frac{(21/4)x^2y}{(7/2)y^{5/2}} 		
	= \begin{cases} 
		\displaystyle\frac{3}{2}x^2y^{-3/2}	& -\sqrt{y}\leq x\leq \sqrt{y}, \\
		0 										& \text{ otherwise.}
	\end{cases}
\]

\it % << (d)
The conditional PDF of $Y$ given that $X=x$ is
\[
f_{Y|X}(y\,|\,x) 	
	= \frac{f_{X,Y}(x,y)}{f_X(x)} 
	= \frac{(21/4)x^2y}{(21/8)x^2(1-x^4)} 	
	= \begin{cases} 
		\displaystyle\frac{2y}{1-x^4} 	& x^2\leq y\leq 1, \\
	 	0 									& \text{ otherwise}.
	 \end{cases}
\]	 

\it % << (d)
$X$ and $Y$ are not independent because the conditional PDF of $Y$ given $X=x$ depends on $x$.

\it % << (e)
The conditional expected value of $Y$ given that $X=x$ is
\begin{align*}
g(x) = \expe(Y|X=x) 
	& = \int_{-\infty}^{\infty} y f_{Y|X}(y\,|\,x)\,dy \\
	& = \int_{x^2}^1 y \left(\frac{2y}{1-x^4}\right)\,dy 
	= \frac{2}{1-x^4}\left[\frac{y^3}{3}\right]_{x^2}^1
	= \frac{2(1-x^6)}{3(1-x^4)} 
\end{align*}	
Hence the conditional expectation of $Y$ given $X$ is the random variable 
\[
\expe(Y|X)=\displaystyle\frac{2(1-X^6)}{3(1-X^4)}
\]
and its expected value is  
\begin{align*}
\expe\big[\expe(Y|X)\big] 
	& = \int_{-\infty}^{\infty} \expe(Y|X=x)f_X(x)\,dx \\
	& = \frac{2}{3}\int_{-1}^{1} \left(\frac{1-x^6}{1-x^4}\right)\left(\frac{21}{8}x^2(1-x^4)\right)\,dx \\
	& = \frac{7}{4}\int_{-1}^{1} x^2(1-x^6)\,dx 
	= \frac{7}{9}.
\end{align*}	
Thus $\expe\big[\expe(Y|X)\big]=\expe(Y)$ as required.
%\it % << (f)
%We need to show that $\var(Y) = \expe\big[\var(Y|X)\big] + \var\big[\expe(Y|X)\big]$.
%
%First we consider $\expe\big[\var(Y|X)\big]$. 
%
%Since $\var(Y|X=x) = \expe(Y^2|X=x) - \expe(Y|X=x)^2$, and
%\begin{align*}
%\expe(Y^2|X=x)
%	& = \int_{-\infty}^{\infty} y^2 f_{Y|X}(y\,|\,x)\,dy \\
%	& \int_{x^2}^1 y^2\left(\frac{2y}{1-x^4}\right)\,dy
%	= \frac{2}{1-x^4}\left[\frac{y^4}{4}\right]_{x^2}^1
%	= \frac{1-x^8}{2(1-x^4)}
%	= \frac{1+x^4}{2},
%\end{align*}
%it follows by part (4) that
%\begin{align*}
%\var(Y|X=x)
%	& = \expe(Y^2|X=x) - \expe(Y|X=x)^2 
%	= \frac{1+x^4}{2} - \frac{4(1-x^6)^2}{9(1-x^4)^2}.
%\end{align*}
%Thus  the expected value of $\var(Y|X)$ is 
%\begin{align*}
%\expe\big[\var(Y|X)\big]
%	& = \int_{-1}^{1} \var(Y|X=x)f_X(x)\,dx \nonumber \\
%	& = \int_{-1}^{1} \left(\frac{1+x^4}{2}\right)\left(\frac{21}{8}x^2(1-x^4)\right)\,dx
%		 - \int_{-1}^{1} \frac{4(1-x^6)^2}{9(1-x^4)^2}\left(\frac{21}{8}x^2(1-x^4)\right)\,dx \nonumber \\
%	& = \frac{21}{16}\int_{-1}^{1} x^2(1-x^8)\,dx 
%		 - \frac{82}{12}\int_{-1}^{1} \frac{x^2(1-x^6)^2}{(1-x^4)}\,dx. %\tag{*}
%\end{align*}
%Before evaluating these integrals, let us now consider the second term,
%\[
%\var\big[\expe(Y|X)\big]= \expe\big[\expe(Y|X)^2\big]-\expe\big[\expe(Y|X)\big]^2.
%\]
%By the law of total expectation, $\expe\big[\expe(Y|X)\big]^2 = \expe(Y)^2$, and by part (4),
%\begin{align*}
%\expe\big[\expe(Y|X)^2\big]
%	& = \int_{-1}^{1} \expe(Y|X=x)^2 f_X(x)\,dx \nonumber \\ 
%	& = \int_{-1}^{1} \frac{4(1-x^6)^2}{9(1-x^9)^2}\left(\frac{21}{8}x^2(1-x^4)\right)\,dx \\
%	& = \frac{82}{12}\int_{-1}^{1} \frac{x^2(1-x^6)^2}{(1-x^4)}\,dx \nonumber
%\end{align*}
%
%Hence the second term is
%%\begin{align}
%$\displaystyle\var\big[\expe(Y|X)\big] 
%%	& = \expe\big(\expe(Y|X)^2\big)-\expe\big(\expe(Y|X)\big)^2 \nonumber\\ 
%%	& = \int_{-1}^{1} \expe(Y|X=x)^2 f_X(x)\,dx  - \expe\big(\expe(Y|X)\big)^2 \nonumber\\ 
%%	& = \int_{-1}^{1} \frac{4(1-x^6)^2}{9(1-x^9)^2}\left(\frac{21}{8}x^2(1-x^4)\right)\,dx -\expe\big(\expe(Y|X)\big)^2 \nonumber\\
%	= \frac{82}{12}\int_{-1}^{1} \frac{x^2(1-x^6)^2}{(1-x^4)}\,dx - \expe(Y)^2 %\tag{**}
%%\end{align}
%$
%so
%%Finally, by (*) and (**) we have 
%\begin{align*}
%\expe\big[\var(Y|X)\big] + \var\big[\expe(Y|X)\big]
%	& = 	\frac{21}{16}\int_{-1}^{1} x^2(1-x^8)\,dx  - \expe(Y)^2 \\
%%	& = \frac{21}{16}\int_{-1}^{1} x^2-x^{10}\,dx - \frac{49}{81} \\
%	& = \frac{21}{16}\left[\frac{x^3}{3}-\frac{x^{11}}{11}\right]_{-1}^{1} - \frac{49}{81} \\
%%	& = \frac{7}{11} - \frac{49}{81} \\
%	& = \frac{28}{891}
%	= \var(Y).
%\end{align*}
%%as required.
\een
\end{solution}

%----------------------------------------------------------------------
\section{Exercises}
% !TEX root = main.tex
%----------------------------------------------------------------------
\begin{exercise}
\begin{questions}
%----------------------------------------
\question
A fair coin is tossed three times. Let $I_j$ be the indicator variable of the event that a head occurs on the $j$th toss. Compute the conditional expectation $E(Y|X)$ and verify the identity $E(E(Y|X)) = E(Y)$ in each of the following cases:
\ben
\it $X =\max\{I_1, I_2, I_3\}$ and $Y = \min\{I_1, I_2, I_3\}$,
\it $X = I_1 + I_2$ and $Y = I_2 + I_3$.
\een

\begin{answer}
\[
\begin{array}{|c|ccc|cc|cc|}\hline
 & & & & \multicolumn{2}{|c|}{\text{Part (i)}} & \multicolumn{2}{|c|}{\text{Part (ii)}} \\
\text{Outcome}	& I_1 & I_2 & I_3 & \max\{I_1,I_2,I_3\} & \min\{I_1,I_2,I_3\} & I_1+I_2 & I_2+I_3 \\ \hline
HHH	& 1 & 1 & 1 	& 1 & 1 	& 2 & 2 \\
HHT	& 1 & 1 & 0 	& 0 & 1 	& 2 & 1 \\
HTH	& 1 & 0 & 1 	& 0 & 1 	& 1 & 1 \\
THH	& 0 & 1 & 1 	& 0 & 1 	& 1 & 2 \\
HTT	& 1 & 0 & 0 	& 0 & 1 	& 1 & 0 \\
THT	& 0 & 1 & 0 	& 0 & 1 	& 1 & 1 \\
TTH	& 0 & 0 & 1 	& 0 & 1 	& 0 & 1 \\
TTT	& 0 & 0 & 0 	& 0 & 0 	& 0 & 0 \\ \hline

\end{array}
\]

\ben
\it % (a)
$X =\max\{I_1, I_2, I_3\}$ and $Y = \min\{I_1, I_2, I_3\}$.

The joint and marginal distributions of $X$ and $Y$ are:
\[
\begin{array}{|c|cc|c|}\hline
	&	Y=0		& Y=1 	& f_X	\\ \hline
X=0	&	1/8		&	0	& 1/8 	\\
X=1	&	6/8		& 1/8	& 7/8 	\\ \hline
f_Y	&	7/8		& 1/8	& 	 	\\ \hline
\end{array}
\]
The conditional distributions of $Y$ given $X=x$ are:
\[
\begin{array}{|c|cc|}\hline
			&\multicolumn{2}{|c|}{y} \\
			&	0		& 1 		\\ \hline
P(Y=y|X=0)	&	1		& 0 		\\ \hline
P(Y=y|X=1)	&	6/7		& 1/7 	\\ \hline
\end{array}
\]

The conditional expectations of $Y$ given $X=x$ are:
\bit
\it $\expe(Y|X=0) = \left(0\times 1\right) + \left(1\times 0\right) = 0$,
\it $\expe(Y|X=1) = \left(0\times\frac{6}{7}\right) + \left(1\times\frac{1}{7}\right) = 1/7$.
\eit
To verify the law of total expectation:
\[
\expe(Y) = \left(0\times\frac{7}{8}\right) + \left(1\times\frac{1}{8}\right) = \frac{1}{8},
\]
and
\begin{align*}
\expe\big(\expe(Y|X)\big)
	& = \expe(Y|X=0)P(X=0)+\expe(Y|X=1)P(X=1) \\
	& = \left(0\times \frac{1}{8}\right) + \left(\frac{1}{7}\times\frac{7}{8}\right) = \frac{1}{8} \\
	& = \expe(Y).
\end{align*}
\it % (b)
$X = I_1 + I_2$ and $Y = I_2 + I_3$.

The joint and marginal distributions of $X$ and $Y$ are:
\[
\begin{array}{|c|ccc|c|}\hline
	&	Y=0		& Y=1	& Y=2	& f_X \\ \hline
X=0	&	1/8		& 1/8	& 0		& 2/8 \\
X=1	&	1/8		& 2/8	& 1/8 	& 4/8 \\ 
X=2	&	  0		& 1/8 	& 1/8	& 2/8 \\ \hline
f_Y &	2/8		& 4/8	& 2/8	&	  \\ \hline
\end{array}
\]
The conditional distributions of $Y$ given $X=x$ are:
\[
\begin{array}{|c|ccc|}\hline
			&\multicolumn{3}{|c|}{y} 		\\
			&	0		& 1		& 2		\\ \hline
P(Y=y|X=0)	&	1/2		& 1/2 	& 0		\\ \hline
P(Y=y|X=1)	&	1/4		& 1/2 	& 1/4	\\ \hline
P(Y=y|X=2)	&	0		& 1/2 	& 1/2	\\ \hline
\end{array}
\]

The conditional expectations $Y$ given $X=x$ are:
\bit
\it $\expe(Y|X=0) = (0\times 1/2)+(1\times 1/2)+(2\times   0) = 1/2$,
\it $\expe(Y|X=1) = (0\times 1/4)+(1\times 1/2)+(2\times 1/4) = 1$,
\it $\expe(Y|X=2) = (0\times 1/2)+(1\times 0)  +(2\times 1/2) = 3/2$.
\eit

To verify the law of total expectation:
\[
\expe(Y) = \left(0\times\frac{2}{8}\right) + \left(1\times\frac{4}{8}\right) + \left(2\times\frac{2}{8}\right) = \frac{1}{2},
\]
and
\begin{align*}
\expe\big(\expe(Y|X)\big)
	& = \expe(Y|X=0)P(X=0)+\expe(Y|X=1)P(X=1)+\expe(Y|X=2)P(X=2) \\
	& = \left(\frac{1}{2}\times\frac{1}{4}\right) + \left(1\times\frac{1}{2}\right) + \left(\frac{3}{2}\times\frac{1}{4}\right) 
	= \frac{1}{8} + \frac{1}{2} + \frac{3}{8}  = \frac{1}{2} \\
	& = \expe(Y).
\end{align*}
\een
\end{answer}

%==========================================================================
\question
Let $X$ and $Y$ be continuous random variables with joint density function
\[
f(x,y) = 
\begin{cases}
	c(x+y) & 0\leq x,y\leq 1 \\
	0		& \text{otherwise}
\end{cases}
\]
\ben
\it Show that $c=1$. 
\it Compute the conditional expectation $\expe(Y|X)$.
\it Verify the identity $\expe\big(\expe(Y|X)\big) = \expe(Y)$.
\een
\begin{answer}
\ben
\it % << (a)
By the law of total probability,
\[
\int_0^1\int_0^1(x+y)\,dx\,dy = 1 \quad\text{so}\quad c = 1
\]
\it % << (b)
The marginal densities are
\begin{align*}
f_X(x)	& = \int_0^1(x+y)\,dy = (x+1/2) \qquad 0\leq x\leq 1 \\
f_Y(x)	& = \int_0^1(x+y)\,dy = (y+1/2) \qquad 0\leq y\leq 1 
\end{align*}
The conditional expectation of $Y$ given $X=x$ is
\begin{align*}
\expe(Y|X=x)
	& = \int_0^1 y\left(\frac{f_{X,Y}(x,y)}{f_X(x)}\right)\,dy \\
	& = \int_0^1 y\left(\frac{x+y}{x+1/2}\right)\,dy \\
	& = \left[\frac{3xy^2 + 2y^3}{6x+3}\right]_0^1 \\
	& = \frac{3x+2}{6x+3}
\end{align*}
so $\displaystyle\expe(Y|X) = \frac{3X+2}{6X+3}$.
\it % << (c)
Check:
\begin{align*}
\expe(Y)
	& = \int_0^1 y f_Y(y)\,dy 
	  = \int_0^1 y\left(y+\frac{1}{2}\right)\,dy = \frac{7}{12} \\
\expe\big(\expe(Y|X)\big)
	& = \int_0^1 \expe(Y|X=x)f_X(x)\,dx \\
	& = \int_0^1 \left(\frac{3x+2}{6x+3}\right)\left(x+\frac{1}{2}\right)\,dx \\
	& = \int_0^1 \left(\frac{x}{2}+\frac{1}{3}\right)\,dx
		= \frac{7}{12}
\end{align*}
\een
\end{answer} 

%==========================================================================
\question
Let the joint density of random variables $X$ and $Y$ be 
\[
f(x,y) = \begin{cases}
	cxy & \quad\text{for}\quad 0\leq x,y\leq 1\text{ where }\ x+y\leq 1, \\
	0	& \quad\text{otherwise.}
\end{cases}
\]
\ben
\it Compute the normalization constant $c$. 
\it Compute the conditional expectation $\expe(Y|X)$.
\it Verify the identity $\expe(\expe(Y|X))=\expe(Y)$.
\een 

\begin{answer}
\ben
\it % << (a)
The marginal PDF of $X$ is
\[
f_X(x) = c\int_{0}^{1-x} xy\,dy = \frac{cx(1-x)^2}{2}
\]
To find $c$,
\[
\int_{0}^{1} f_X(x)\,dx = 1, \qquad\text{so}\quad c=24.
\]
Thus
\begin{align*}
f_X(x)	& = 12x(1-x)^2 \qquad 0\leq x\leq 1 \\
f_Y(y)	& = 12y(1-y)^2 \qquad 0\leq y\leq 1 \\
\end{align*}
and
\begin{align*}
\expe(X) & = 12\int_0^1 x(1-x)^2\,dx = 2/5 \\
\expe(Y) & = 12\int_0^1 y(1-y)^2\,dy = 2/5 \\
\end{align*}
\it % << (b)
To compute $\expe(Y\,|\,X)$,
\begin{align*}
\expe(Y\,|\,X=x)
	& = \int_0^1 y\left(\frac{f_{X,Y}(x,y)}{f_X(x)}\right)\,dy \\
	& = \int_0^{1-x} y\left(\frac{24xy}{12x(1-x)^2}\right)\,dy \\
	& = \frac{24x}{12x(1-x)^2}\int_0^{1-x} y^2\,dy \\
	& =  \frac{24x}{12x(1-x)^2}\left[\frac{(1-x)^3}{3}\right] \\
	& = \frac{2}{3}(1-x)	
\end{align*}
so $\expe(Y\,|\,X) = 2(1-X)/3$.
\it % << (c)
Check:
\begin{align*}
\expe\big(\expe(Y\,|\,X)\big)
	& = \int_0^1 \expe(Y\,|\,X=x) f_X(x)\,dx \\
	& = \frac{2}{3}\int_0^1 (1-x) f_X(x)\,dx \\
	& = \frac{2}{3}\big(1-\expe(X)\big) = \frac{2}{3}\left(1-\frac{2}{5}\right) = \frac{2}{5} \\
	& = \expe(Y)
\end{align*}
\een
\end{answer}



%----------------------------------------
\end{questions}
\end{exercise}
%----------------------------------------------------------------------

%======================================================================
\endinput
%======================================================================

%----------------------------------------------------------------------

%======================================================================
\endinput
%======================================================================
