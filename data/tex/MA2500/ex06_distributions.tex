% !TEX root = main.tex
%----------------------------------------------------------------------
\begin{exercise}
\begin{questions}
%----------------------------------------
%----------------------------------------
% GS 2.1.4
\question
Let $F$ and $G$ be CDFs, and let $0<\lambda<1$ be a constant. Show that $H = \lambda F + (1-\lambda)G$ is also a CDF.
\begin{answer}
Let $H(x) = \lambda F(x) + (1-\lambda)G(x)$. It is easy to show that $H$ has the following properties:
\bit
\it if $x < y$ then $H(x) \leq H(y)$,
\it $H(x)\to 0$ as $x\to-\infty$,
\it $H(x)\to 1$ as $x\to+\infty$, and
\it $H(x+\epsilon)\to H(x)$ as $\epsilon\downarrow 0$.
\eit
Thus $H$ is a distribution function.
\end{answer}

%----------------------------------------
%%GS 2.7.9
%\question
%Let $X$ be a random variable, and let $F$ denote its CDF. Find the CDFs of the following random variables in terms of $F$:
%\begin{parts}
%\part $X^{+} = \max\{0,X\}$.
%\begin{answer}
%TODO
%\end{answer}
%\part $X^{-} = -\min\{0,X\}$.
%\begin{answer}
%TODO
%\end{answer}
%\part $|X| = X^{+} + X^{-}$.
%\begin{answer}
%TODO
%\end{answer}
%\part $-X$.
%\begin{answer}
%TODO
%\end{answer}
%----------------------------------------
\question
Let $X_1$ and $X_2$ be the numbers observed in two independent rolls of a fair die. Find the PMF of each of the following random variables:
\begin{parts}
%%--------------------
%\part $X_1$,
%\begin{answer}
%$P(X_1=k) = 1/6$ for $k=1,\ldots,6$.
%\end{answer}
%--------------------
\part $Y = 7 - X_1$,
\begin{answer}
$P(Y=k) = 1/6$ for $k=1,\ldots,6$.
\end{answer}
%--------------------
\part $U = \max(X_1,X_2)$,
\begin{answer}
Let $U=\max\{X_1,X_2\}$. Then since $\{X_1\leq k\}$ and $\{X_2\leq k\}$ are independent events,
\begin{align*}
P(U\leq k) 
	& = P(X_1\leq k \text{ and } X_2\leq k) \\
    & = P(X_1\leq k)P(X_2\leq k) \\
    & = (k/6)\cdot (k/6) = k^2/36
\end{align*}
Thus,
\[
P(U=k)  
	= P(U\leq k)-P(U\leq k-1)
    = \frac{(k^2-(k-1)^2)}{36} 
    = \frac{(2k-1)}{36}
\]    
\end{answer}
%--------------------
\part $V = X_1-X_2$.
\begin{answer}
The values of $V=X_1-X_2$ at each point of the sample space $\Omega=\{(i,j):1\leq i,j\leq 6\}$ are
\[\begin{array}{|cc|cccccc|}\hline
	& 	& & & j & & & \\
	& 	& 1	 & 2  & 3  & 4  & 5  & 6 \\\hline
	& 1 	& 0  & 1  & 2  & 3  & 4  & 5 \\
	& 2 	& -1 & 0  & 1  & 2  & 3  & 4 \\
i	& 3 	& -2 & -1 & 0  & 1  & 2  & 3 \\
	& 4 	& -3 & -2 & -1 & 0  & 1  & 2 \\
	& 5 	& -4 & -3 & -2 & -1 & 0  & 1 \\
	& 6 	& -5 & -4 & -3 & -2 & -1 & 0 \\ \hline
\end{array}\]
The required probabilities are obtained by counting the number of outcomes that give the same value of $V=X_1-X_2$:
\small
\[\begin{array}{r|rrrrrrrrrrr}
v         & -5  	& -4	 	& -3 	& -2		& -1		& 0		& 1		& 2		& 3		& 4		& 5		\\ \hline
P(V=v)    & 1/36	& 2/36	& 3/36	& 4/36	& 5/36	& 6/36	& 5/36	& 4/36	& 3/36	& 2/36	& 1/36 	\\
\end{array}\]
\normalsize
\end{answer}
%--------------------
\part $W = |X_1-X_2|$.
\begin{answer}
\[
\begin{array}{c|cccccc}
w		& 0		& 1		& 2 		& 3 		& 4 		& 5 		\\ \hline
P(W=w)	& 6/36	& 10/36	& 8/36	& 6/36	& 4/36	& 2/36 
\end{array}
\]
\end{answer}
%--------------------
\end{parts}


%----------------------------------------
\question
The PDF of a continuous random variable $X$ is given by
$
f(x) = \left\{\begin{array}{ll}
	cx^2 	& 1\leq x\leq 2,  \\
	0		& \text{otherwise.}
\end{array}\right.	
$
\begin{parts}
%--------------------
\part Find the value of the constant $c$, and sketch the PDF of $X$.
\begin{answer}
 The PDF must integrate to 1:
\[
\int_{-\infty}^{\infty}f(x)\,dx
	= \int_{1}^{2} cx^{2}\,dx 
	= \left[\frac{cx^{3}}{3} \right]_{1}^{2} 
	= \frac{7c}{3}
	 = 1
\]
so $c=3/7$. (The sketch is a quadratic curve between $x=1$ and $x=2$.)
\end{answer}
%--------------------
\part Find the value of $P(X > 3/2)$.
\begin{answer}
\[
P(X > 3/2) 	= \int_{3/2}^{2}\frac{3x^{2}}{7}\,dx 
			= \left[\frac{x^{3}}{7}\right]_{3/2}^{2} 
			= \frac{37}{56}
\]
\end{answer}
%--------------------
\part Find the CDF of $X$.
\begin{answer}
For $1\leq x\leq 2$,
\[
F(x) 	= \int_{-\infty}^{x} f(x)\,dx
		= \int_{1}^{x}\frac{3x^{2}}{7}\,dx
		= \left[ \frac{x^{3} }{7} \right] _{1}^{x} 
		= \frac{x^{3}-1}{7}
\]
so the CDF of $X$ is
\[
F(x) = \left\{\begin{array}{ll}
	0				& x < 1 \\
	\frac{1}{7}(x^{3}-1)		& 1\leq x < 2 \\
	1				& x \geq 2
\end{array}\right.	
\]	
\end{answer}
%--------------------
\end{parts}

%----------------------------------------
% GS 2.3.5(a)
\question
The PDF of a continuous random variable $X$ is given by
$
f(x) = \begin{cases}
	cx^{-d}	& \text{for } x > 1, \\
	0		& \text{otherwise.}
\end{cases}
$
\begin{parts}
%--------------------
\part Find the range of values of $d$ for which $f(x)$ is a probability density function.
\begin{answer}
The function $f(x)=cx^{-d}$ is only integrable if $d>1$, in which case
\[
\int_{-\infty}^\infty f(x)\,dx = \int_1^\infty \frac{c}{x^d}\,dx = \left[\frac{-c}{(d-1)x^{d-1}}\right]_1^{\infty} = \frac{c}{d-1}
\]
\end{answer}
%--------------------
\part If $f(x)$ is a density function, find the value of $c$, and the corresponding CDF.
\begin{answer}
If $f(x)$ is a probability density function, we require that $\int_{-\infty}^\infty f(x)\,dx = 1$, so we must have that $c = d-1$. The corresponding distribution function is
\[
F(x) = \int_{-\infty}^x f(u)\,du  
	= \int_1^\infty \frac{d-1}{u^d}\,du 
	= \left[\frac{-1}{x^{d-1}}\right]_1^x
	= 1 - \frac{1}{x^{d-1}}
\]
for $x>1$, and zero otherwise.
\end{answer}
%--------------------
\end{parts}

%----------------------------------------
% GS 2.3.5(b)
\question
Let $\displaystyle f(x) = \frac{ce^x}{(1+e^x)^2}$ be a PDF, where $c$ is a constant. Find the value of $c$, and the corresponding CDF.
\begin{answer}
By inspection, $f(x) = F'(x)$ where $F(x) = \frac{ce^x}{1+e^x}$. Writing this as $F(x) = \frac{c}{e^{-x}+1}$ it is easy to see that $F(x)\to c$ as $x\to\infty$, so we must have that $c=1$.
\end{answer}

%----------------------------------------
% GS 2.2.3
\question
Let $X_1,X_2,\ldots$ be independent and identically distributed observations, and let $F$ denote their common CDF. If $F$ is unknown, describe and justify a way of estimating $F$, based on the observations. [Hint: consider the indicator variables of the events $\{X_j\leq x\}$.]
\begin{answer}
Let $X$ be a random variable with same CDF, and let $I_j(x)$ the indicator variable of the event $\{X_j\leq x\}$. Then
\[
\prob(X\leq x) \approx \frac{1}{n}\sum_{j=1}^{n} I_j(x).
\]
The RHS yields the proportion of observations that are at most equal to $x$.
\end{answer}

%----------------------------------------
\end{questions}
\end{exercise}
%----------------------------------------------------------------------

%======================================================================
\endinput
%======================================================================
