% !TEX root = main.tex
%----------------------------------------------------------------------
\chapter{Set Theory}\label{chap:sets}
%----------------------------------------------------------------------
%----------------------------------------------------------------------
\section{Elementary set theory}
%----------------------------------------------------------------------
A set is a collection of distinct \emph{elements}.
\begin{itemize}
\item If $a$ is an element of the set $A$, we denote this by $a\in A$.
\item If $a$ is \emph{not} an element of $A$, we denote this by $a\notin A$.
\item The \emph{cardinality} of a set is the number of elements it contains.
\item The \emph{empty set} contains no elements, and is denoted by $\emptyset$.
\end{itemize}

\subsection{Set relations}
Let $A,B$ be sets. 
\begin{itemize}
\item If $a\in B$ for every $a\in A$, we say that $A$ is a \emph{subset} of $B$, denoted by $A\subseteq B$.
\item If $A\subseteq B$ and $B\subseteq A$, we say that $A$ is \emph{equal} to $B$, denoted by $A=B$, 
\item If $A\subseteq B$ and $A\neq B$, we say that $A$ is a \emph{proper subset} of $B$, denoted by $A\subset B$.
\end{itemize}

\subsection{Set operations}
Let $A$, $B$ and $\Omega$ be sets, with $A,B\subseteq \Omega$.
\begin{itemize}
\item The \emph{union} of $A$ and $B$ is the set $A\cup B = \{a\in \Omega: a\in A \text{ or }a\in B\}$.
\item The \emph{intersection} of $A$ and $B$ is the set $A\cap B = \{a\in \Omega: a\in A \text{ and }a\in B\}$.
\item The \emph{complement} of $A$ (relative to $\Omega$) is the set $A^c=\{a\in \Omega:a\notin A\}$.
\end{itemize}

\subsection{Set algebra}
\begin{tabular}{ll}
Commutative property:  	& $A\cup B = B\cup A$ \\
						& $A\cap B = B\cap A$ \\[2ex]
Associative property:	& $(A\cup B)\cup C = A\cup (B\cup C)$ \\
						& $(A\cap B)\cap C = A\cap (B\cap C)$ \\[2ex] 
Distributive property:	& $A\cup (B\cap C) = (A\cup B)\cap(A\cup C)$ \\
						& $A\cap (B\cup C) = (A\cap B)\cup(A\cap C)$ \\
\end{tabular}

%----------------------------------------------------------------------
\section{Sample space, outcomes and events}
%----------------------------------------------------------------------

\begin{definition}
\begin{enumerate}
\item Any process of observation or measurement whose outcome is uncertain is called a \emph{random experiment}.
\item A random experiment has a number of possible \emph{outcomes}. 
\item Each time a random experiment is performed, \emph{exactly one} of its outcomes will occur.
\item The set of all possible outcomes is called the \emph{sample space}, denoted by $\Omega$.
\item Outcomes are also called \emph{elementary events}, and denoted by $\omega\in\Omega$.
\end{enumerate}
\end{definition}

\begin{example}
\begin{itemize}
\item $\{1,2,\ldots,n\}$ is a finite sample space,
\item $\N=\{0,1,2,3,\ldots\}$ is a countably infinite sample space,
\item $[0,1]$ is an uncountable sample space.
\end{itemize}
\end{example}

\break % <<

\begin{definition}
\begin{enumerate}
\item An \emph{event} $A$ is a subset of the sample space, $\Omega$. 
\item If outcome $\omega$ occurs, we say that event $A$ \emph{occurs} if and only if $\omega\in A$.
\item Two events $A$ and $B$ with $A\cap B=\emptyset$ are called \emph{disjoint} or \emph{mutually exclusive}.
\item The empty set $\emptyset$ is called the \emph{impossible event}.
\item The sample space itself is called the \emph{certain event}.
\end{enumerate}
\end{definition}

\begin{remark}
\begin{itemize}
\item If $A$ occurs and $A\subseteq B$, then $B$ occurs.
\item If $A$ occurs and $A\cap B=\emptyset$, then $B$ does not occur. 
\end{itemize}
\end{remark}

%----------------------------------------------------------------------
\section{Countable unions and intersections}
%----------------------------------------------------------------------

% power set
\begin{definition}
Let $\Omega$ be any set. The set of all subsets $\Omega$ is called its \emph{power set}.
\end{definition}

\begin{itemize}
\item If $\Omega$ is a finite set, its power set is also finite.
\item If $\Omega$ is a countably infinite set, its power set is uncountable set (Cantor's Theorem).
\item If $\Omega$ is an uncountable set, its power set is also uncountable.
\end{itemize}


\begin{definition}
Let $A_1,A_2,\ldots$ be subsets of $\Omega$.
\begin{enumerate}
\item The (countable) \emph{union} of $A_1,A_2,\ldots$ is the set 
$$\displaystyle\bigcup_{i=1}^\infty A_i = \{\omega:\omega\in A_i \text{ for some }A_i\}.$$
\item The (countable) \emph{intersection} of $A_1,A_2,\ldots$ is the set
$$\displaystyle\bigcap_{i=1}^\infty A_i = \{\omega:\omega\in A_i \text{ for all } A_i\}.$$
\end{enumerate}
\end{definition}

\break % <<

% theorem: De Morgan
\begin{theorem}[De Morgan's laws]
For a countable collection of sets $\{A_1,A_2,\ldots\}$,
\begin{enumerate}
\item $\big(\bigcup_{i=1}^{\infty} A_i\big)^c = \bigcap_{i=1}^{\infty} A_i^c$,
\item $\big(\bigcap_{i=1}^{\infty} A_i\big)^c = \bigcup_{i=1}^{\infty} A_i^c$.
\end{enumerate}
\end{theorem}

% proof
\begin{proof}
\begin{enumerate}
\item % part (1) 	
Let $a\in\big(\bigcup_{i=1}^\infty A_i\big)^c$. Then $a\notin\bigcup_{i=1}^\infty A_i$, and so $a\in A_i^c$ for all $A_i$.\par
Hence $\big(\bigcup_{i=1}^\infty A_i\big)^c \subseteq \bigcap_{i=1}^\infty A_i^c$.\par
Let $a\in\bigcap_{i=1}^\infty A_i^c$. Then $a\notin A_i$ for all $A_i$, and so $a\notin\bigcup_{i=1}^\infty A_i$.\par 
Hence $\bigcap_{i=1}^\infty A_i^c \subseteq \big(\bigcup_{i=1}^\infty A_i\big)^c$.
\item % part (2)
Applying part (1) to the collection of sets $\{A_1^c,A_2^c,\ldots\}$,\par
$\big(\bigcup_{i=1}^\infty A_i^c\big)^c = \bigcap_{i=1}^\infty \big(A_i^c\big)^c = \bigcap_{i=1}^\infty A_i$.
Taking the complement of both sides,\par 
$\big(\bigcap_{i=1}^\infty A_i\big)^c = \bigcap_{i=1}^\infty A_i^c$.
\end{enumerate}
\end{proof}


%----------------------------------------------------------------------
\section{Collections of sets}
%----------------------------------------------------------------------
\begin{definition}
Let $\Omega$ be any set. Any subset of its power set is called a \emph{collection of sets over $\Omega$}. 
\end{definition}

Let $\Omega$ be the sample space of some random experiment. 
If we are interested whether the events $A$ and $B$ occur, we must also be interested in

\begin{itemize}
\item the event $A\cup B$: whether event $A$ occurs \emph{or} event $B$ occurs;
\item the event $A\cap B$: whether event $A$ occurs \emph{and} event $B$ occurs;
\item the event $A^c$: whether the event $A$ does \emph{not} occur.
\end{itemize}

\smallskip
Thus we can not use arbitrary collections of sets over $\Omega$ as the basis for investigating random experiments. Instead, we allow only collections which are \emph{closed} under certain set operations.

\begin{definition}
A collection of sets $\mathcal{C}$ over $\Omega$ is said to be
\ben
\it \emph{closed under complementation} if $A^c\in\mathcal{C}$ for every $A\in\mathcal{C}$, 
\it \emph{closed under pairwise unions} if $A\cup B\in\mathcal{C}$ for every $A,B\in\mathcal{C}$, 
\it \emph{closed under finite unions}	if $\bigcup_{i=1}^{n} A_i\in\mathcal{C}$ for every $A_1,A_2,\ldots A_n\in\mathcal{C}$,
\it \emph{closed under countable unions} if $\bigcup_{i=1}^{\infty} A_i\in\mathcal{C}$ for every $A_1,A_2,\ldots\in\mathcal{C}$.
\een
\end{definition}

% defn: fields of sets
\begin{definition}
A collection of sets $\mathcal{F}$ over $\Omega$ is called a \emph{field} over $\Omega$ if
\begin{enumerate}
\item $\Omega\in\mathcal{F}$,
\item $\mathcal{F}$ is closed under complementation, and
\item $\mathcal{F}$ is closed under pairwise unions.
\end{enumerate}
\end{definition}

% properties of fields
\begin{theorem}[Properties of fields]
Let $\mathcal{F}$ be a field over $\Omega$. Then
\begin{enumerate}
\item $\emptyset\in\mathcal{F}$,
\item $\mathcal{F}$ is closed under set differences,
\item $\mathcal{F}$ is closed under finite unions,
\item $\mathcal{F}$ is closed under finite intersections.
\end{enumerate}
\end{theorem}

\begin{proof}
See exercises.
\end{proof}

\break % <<

% defn: sigma fields
\begin{definition}
A collection of sets $\mathcal{F}$ over $\Omega$ is called a \emph{$\sigma$-field} (``sigma-field'') over $\Omega$ if
\begin{enumerate}
\item $\Omega\in\mathcal{F}$,
\item $\mathcal{F}$ is closed under complementation, and
\item $\mathcal{F}$ is closed under countable unions.
\end{enumerate}
\end{definition}

% properties of sigma fields
\begin{theorem}[Properties of $\sigma$-fields]\label{thm:properties-of-sigma-fields}
Let $\mathcal{F}$ be a $\sigma$-field over $\Omega$. Then
\begin{enumerate}
\item $\emptyset\in\mathcal{F}$,
\item $\mathcal{F}$ is closed under set differences,
\item $\mathcal{F}$ is closed under finite unions,
\item $\mathcal{F}$ is closed under finite intersections,
\item $\mathcal{F}$ is closed under countable intersections.
\end{enumerate}
\end{theorem}

\begin{proof}
See exercises.
\end{proof}

%----------------------------------------------------------------------
\section{Borel sets}
%----------------------------------------------------------------------
In many situations of interest, random experiments yield outcomes that are \emph{real numbers}.

% definition: intervals
\begin{definition}
\begin{itemize}
\item The \emph{open interval} $(a,b)$ is the set $\{x\in\R : a < x < b\}$.
\item The \emph{closed interval} $[a,b]$ is the set $\{x\in\R : a\leq x\leq b\}$.
\end{itemize}
\end{definition}

% definition: Borel $\sigma$-field
\begin{definition}
The \emph{Borel} $\sigma$-field over $\R$ is defined to be the smallest $\sigma$-field over $\R$ that contains all open intervals.
\end{definition}

\begin{remark}
\begin{itemize}
\item The Borel $\sigma$-field is usually denoted by $\mathcal{B}$, and includes all closed interval, all half-open intervals, all finite sets and all countable sets.
%	\begin{itemize}
%	\item all closed intervals, 
%	\item all half-open intervals,
%	\item all finite sets,
%	\item all countable sets.
%	\end{itemize}
\item The elements of $\mathcal{B}$ are called \emph{Borel sets} over $\R$.
\item Borel sets can be thought of as the ``nice'' subsets of $\R$.
\end{itemize}
\end{remark}

% proopsition: closed intervals
\begin{proposition}
The Borel $\sigma$-field over $\R$ contains all closed intervals.
\end{proposition}

\begin{proof}
Any closed interval $[a,b]$ can be written as a countable intersection of open intervals:
\[
[a, b] = \bigcap_{n=1}^\infty\left(a-\frac{1}{n},\ b+\frac{1}{n}\right).
\]
Hence $[a,b]\in\mathcal{B}$, because
\begin{itemize}
\item for every $n\in\N$, $\displaystyle\left(a-\frac{1}{n},b+\frac{1}{n}\right)\in\mathcal{B}$, and
\item by Theorem~\ref{thm:properties-of-sigma-fields}, $\mathcal{B}$ is closed under countable intersections.
\end{itemize}
\end{proof}


%----------------------------------------------------------------------
\section{Exercises}
% !TEX root = main.tex
%----------------------------------------------------------------------
% EXERCISE 1: PROOFS
\begin{exercise}
\begin{questions}
%----------------------------------------
% properties of fields
\question
Let $\mathcal{F}$ be a field over $\Omega$. Show that
\begin{parts}
%--------------------
\part $\emptyset\in\mathcal{F}$,
\begin{answer}
$\mathcal{F}$ is closed under complementation, and $\emptyset = \Omega^c$ where $\Omega\in\mathcal{F}$, so $\emptyset = \Omega^c$.
\end{answer}
%--------------------
\part $\mathcal{F}$ is closed under set differences,
\begin{answer}
Let $A,B\in\mathcal{F}$. Then $A\setminus B = A\cap B^c = (A^c\cup B)^c$ (De Morgan's laws). Hence $A\setminus B\in\mathcal{F}$ because $\mathcal{F}$ is closed under complementation and pairwise unions.
\end{answer}
%--------------------
\part $\mathcal{F}$ is closed under pairwise intersections,
\begin{answer}
Let $A,B\in\mathcal{F}$. Then $A\cap B = (A^c\cup B^c)^c$ (De Morgan's laws). Hence $A\cap B\in\mathcal{F}$ because $\mathcal{F}$ is closed under complementation and pairwise unions.
\end{answer}
%--------------------
\part $\mathcal{F}$ is closed under finite unions,
\begin{answer}
Proof by induction. Suppose that $\mathcal{F}$ is closed under unions of $n$ sets (where $n\geq 2$). Let $A_1,A_2,\ldots,A_{n+1}\in\mathcal{F}$. By the inductive hypothesis, $\cup_{i=1}^n\in\mathcal{F}$, so $\cup_{i=1}^{n+1} A_i = \big[\cup_{i=1}^{n} A_i\big] \cup A_{n+1} \in\mathcal{F}$ because $\mathcal{F}$ is closed under pairwise unions.
\end{answer}
%--------------------
\part $\mathcal{F}$ is closed under finite intersections.
\begin{answer}
Let $A_1,A_2,\ldots,A_n\in\mathcal{F}$. Then $\cap_{i=1}^n A_i = \big[\cup_{i=1}^n A_i^c\big]^c$ (De Morgan's laws). Hence $\cap_{i=1}^n A_i\in\mathcal{F}$ because $\mathcal{F}$ is closed under complementation and finite unions. 
\end{answer}
%--------------------
\end{parts}
%----------------------------------------
% properties of sigma fields
\question
Let $\mathcal{F}$ be a $\sigma$-field over $\Omega$. Show that
\begin{parts}
%--------------------
\part $\mathcal{F}$ is closed under finite unions,
\begin{answer}
Let $A_1,A_2,\ldots,A_n\in\mathcal{F}$. Since $\mathcal{F}$ is closed under countable unions and $\emptyset\in\mathcal{F}$, 
\[
\cup_{i=1}^n A_i = A_1\cup A_2\cup\ldots\cup A_n\cup\emptyset\cup\emptyset\ldots \in\mathcal{F}.
\]
\end{answer}
%--------------------
\part $\mathcal{F}$ is closed under finite intersections.
\begin{answer}
Let $A_1,A_2,\ldots,A_n\in\mathcal{F}$. Since $\mathcal{F}$ is closed under complementation and finite unions,
\[
\cap_{i=1}^n A_i = A_1\cap\ldots\cap A_n = (A^c_1\cup\ldots\cup A^c_n)^c \in\mathcal{F}.
\]
\end{answer}
%--------------------
\part $\mathcal{F}$ is closed under countable intersections.
\begin{answer}
Let $A_1,A_2,\ldots\in\mathcal{F}$. Since $\mathcal{F}$ is closed under complementation and countable unions,
\[
\bigcap_{n=1}^{\infty} A_n = \left(\bigcup_{n=1}^{\infty} A^c_n\right)^c \in\mathcal{F}.
\] 
\end{answer}
%--------------------
\end{parts}
\end{questions}
\end{exercise}

% EXERCISE 2: APPLICATIONS
\begin{exercise}
\begin{questions}
%----------------------------------------
% sigma fields
\question
Let $\Omega=\{1,2,3,4,5,6\}$. 
\begin{parts}
%--------------------
\part What is the smallest $\sigma$-field containing the event $A=\{1,2\}$?
\begin{answer}
A $\sigma$-field must contain $\emptyset$ and $\Omega$, and be closed under complementation and countable unions. 
\par
The smallest $\sigma$-field containing $A=\{1,2\}$ is therefore
\[
\mathcal{F} = \{\emptyset, \{1,2\}, \{3,4,5,6\}, \Omega\}
\]
\end{answer}
%--------------------
\part What is the smallest $\sigma$-field containing the events $A=\{1,2\}$, $B=\{3,4\}$ and $C=\{5,6\}$?
\begin{answer}
\[
\mathcal{F} = \{\emptyset, \{1,2\}, \{3,4\}, \{5,6\}, \{1,2,3,4\}, \{1,2,5,6\}, \{3,4,5,6\}, \Omega\}
\]
\end{answer}
\end{parts}
%----------------------------------------
% GS 1.8.3
\question
Let $\mathcal{F}$ and $\mathcal{G}$ be $\sigma$-fields over $\Omega$.
\begin{parts}
%--------------------
\part Show that $\mathcal{H}=\mathcal{F}\cap\mathcal{G}$ is a $\sigma$-field over $\Omega$.
\begin{answer}
$\mathcal{H}$ is a $\sigma$-field because:
\bit
\it $\emptyset\in\mathcal{F}$ and $\emptyset\in\mathcal{G}$ so $\emptyset\in\mathcal{H}$;
\it if $A$ belongs to both $\mathcal{F}$ and $\mathcal{G}$, then $A^c$ belongs to both $\mathcal{F}$ and $\mathcal{G}$, so $\mathcal{H}$ is closed under complementation;
\it if $A_1,A_2,\ldots$ all belong to both $\mathcal{F}$ and $\mathcal{G}$, then their union also lies in both $\mathcal{F}$ and $\mathcal{G}$, so $\mathcal{H}$ is closed under countable unions.
\eit
 \end{answer}
%--------------------
\part Find a counterexample to show that $\mathcal{H}=\mathcal{F}\cup\mathcal{G}$ is not necessarily a $\sigma$-field over $\Omega$.
\begin{answer}
Let $\Omega=\{a,b,c\}$, $\mathcal{G}=\big\{\emptyset,\{a\},\{b,c\},\Omega\big\}$ and $\mathcal{G}=\big\{\emptyset,\{a,b\},\{c\},\Omega\big\}$. Then
\[
\mathcal{H} = \mathcal{F}\cup\mathcal{G} = \big\{\emptyset,\{a\},\{c\},\{a,b\},\{b,c\},\Omega\big\}.
\]
Hence $\{a\}\in\mathcal{H}$ and $\{c\}\in\mathcal{H}$, but $\{a,c\}\notin\mathcal{H}$ so $\mathcal{H}$ is not a $\sigma$-field.
\end{answer}
%--------------------
\end{parts}

\end{questions}
\end{exercise}

%======================================================================
\endinput
%======================================================================

%----------------------------------------------------------------------

%======================================================================
\endinput
%======================================================================
