% !TEX root = main.tex
%----------------------------------------------------------------------
\begin{exercise}
\begin{questions}
%----------------------------------------
% discrete
\question
Let $X$ be a discrete random variable, with PMF $f_X(-2)=1/3$, $f_X(0)=1/3$, $f_X(2)=1/3$, and zero otherwise. Find the distribution of $Y=X+3$.
\begin{answer}
The function $g(x)=x+3$ is injective, with $g^{-1}(y) = y-3$, so
\[
f_Y(1)=1/3,\quad f_Y(3)=1/3,\quad f_Y(5)=1/3.
\]
Note that $\supp(f_X)=\{-2,0,2\}$, and $\supp(f_Y) = \{g(x):x\in\supp(f_X)\} = \{1,3,5\}$.
\end{answer}
% binomial
\question
Let $X\sim\text{Binomial}(n,p)$ and define $g(x)=n-x$. Show that $g(X)\sim\text{Binomial}(n,1-p)$.
\begin{answer}
$g(x)=n-x$ is a decreasing function on $[0,n]$: its (unique) inverse is $g^{-1}(y)=n-y$. 
\par
By Theoem~\ref{thm:transf_injective_discrete}, the PMF of $Y=g(X)$ is
\[
f_Y(y) = f_X\big[g^{-1}(y)\big] = f_X(n-y) 
	= \binom{n}{n-y} p^{n-y}(1-p)^{n-(n-y)} = \binom{n}{y}(1-p)^y (1-(1-p))^{n-y},
\]
which is the PMF of the $\text{Binomial}(n,1-p)$ distribution.
\end{answer}
% cdfs
\question
Let $X$ be a random variable, and let $F_X$ denote its CDF. Find the CDF of $Y=X^2$ in terms of $F_X$.
\begin{answer}
\begin{align*}
F_Y(y) = \prob(Y\leq y) = \prob(X^2\leq y) \\
	& = \prob(-\sqrt{y}\leq X\leq \sqrt{y}) \\[1ex]
	& = \prob(X\leq \sqrt{y}) - \prob(X < -\sqrt{y}) \\[1ex]
	& = \begin{cases}
		F_X(\sqrt{y}) - F_X(-\sqrt{y})	& y > 0, \\
		0								& \text{otherwise.}
	\end{cases}	
\end{align*}
\end{answer}

%--------------------
\question
Let $X$ be a random variable with the following CDF:
\[
F_X(x) = \left\{\begin{array}{ll}
	\displaystyle 1-\frac{1}{x^3}		& \text{for $x\geq 1$,} \\
	0								& \text{otherwise.}
\end{array}\right.
\]
Find the CDF of the random variable $Y=1/X$, and describe how a pseudo-random sample from the distribution of $Y$ can be obtained using an algorithm that generates uniformly distributed pseudo-random numbers in the range $[0,1]$.
\begin{answer}
Let $g(x) = 1/x$ denote the transformation. 
\bit
\it $\supp(f_X) = [1,\infty] \Rightarrow\ \supp(f_Y) = [0,1]$.
\it The inverse transformation: $g^{-1}(y) = 1/y$.
\eit
\par
Because $g(x)$ is a decreasing function over $\supp(f_X)$,
\[
F_Y(y) 
	= 1 - F_X\big[g^{-1}(y)\big]
	= 1 - F_X\left(\frac{1}{y}\right)
	= \left\{\begin{array}{ll}
		0	& y < 0 \\
		y^3	& 0\leq y\leq 1 \\
		1	& y > 1.
	\end{array}\right.
\]
To find a pseudo-random sample from the distribution of $Y$, we use the fact that
$F_Y(Y)\sim\text{Uniform}(0,1)$. Let $u=F_Y(y)$. Then 
\[
y = F_Y^{-1}(u) = u^{1/3}.
\]
The required sample is obtained by generating a pseudo-random sample $u_1,u_2,\ldots,u_n$ from the $\text{Uniform}(0,1)$ distribution, then computing
\[
y_i = u_i^{1/3} \text{\quad for $i=1,2,\ldots,n$.}
\]
\end{answer}
%--------------------

%----------------------------------------
\end{questions}
\end{exercise}
%----------------------------------------------------------------------

%======================================================================
\endinput
%======================================================================
