% !TEX root = main.tex
%----------------------------------------------------------------------
\begin{exercise}
\begin{questions}
%----------------------------------------
%----------------------------------------
% GS 2.5.2
\question
Let $X$ be a Bernoulli random variable with parameter $p$.% $\prob(X=0)=1-p$ and $\prob(X=1)=p$.
\begin{parts}
\part Let $Y=1-X$. Find the joint PMF of $X$ and $Y$.
\begin{answer}
\[
f_{X,Y}(x,y) = \begin{cases}
	p	& \text{if } (x,y) = (1,0) \\
	1-p	& \text{if } (x,y) = (0,1) \\
	0	& \text{otherwise.}
\end{cases}
\]
\end{answer}
\part Let $Y=1-X$ and $Z=XY$. Find the joint PMF of $X$ and $Z$.
\begin{answer}
\[
f_{X,Z}(x,z) = \begin{cases}
	1-p	& \text{if } (x,z) = (0,0) \\
	p	& \text{if } (x,z) = (1,0) \\
	0	& \text{otherwise.}
\end{cases}
\]
\end{answer}
\end{parts}


%%----------------------------------------
%% GS 2.5.3
%\question
%The random variables $X$ and $Y$ have the joint distribution function
%\[
%F(x,y) = \begin{cases}
%	(1-e^{-x})\left(\frac{1}{2} + \frac{1}{\pi}\tan^{-1} y\right)	& \text{if } x \geq 0, y\in\R\\
%	0															& \text{otherwise.}
%\end{cases}	
%\]
%Show that $X$ and $Y$ are jointly continuous.
%\begin{answer}
%Differentiating $F(x,y)$ with respect to $x$ and $y$, we obtain
%\[
%f(x,y) = \frac{\partial^2}{\partial x\partial y}F(x,y) = \frac{e^{-x}}{\pi(1+y^2)} \qquad \text{for }x\geq 0,\ y\in\R
%\]
%where we have used the fact that $\frac{d}{dy}(\tan^{-1} y) =  \frac{1}{1+y^2}$.
%%
%Thus there exists a function $f(x,y)$ such that
%\[
%F(x,y) = \int_{-\infty}^x \int_{-\infty}^y f(u,v)\,du\,dv
%\]
%so $X$ and $Y$ are jointly continuous.
%\end{answer}

%%----------------------------------------
%\question
%Let $X$ and $Y$ be two jointly discrete random variables with joint PMF given by the following table:
%\[
%\begin{array}{|cc|cccc|}\hline
%    &       &       & \multicolumn{2}{c}{y} &   \\
%    &       & 0     & 1     & 2     & 3     \\ \hline
%    & 0     & 0     & 3/56  & 6/56  & 1/56  \\
%\raisebox{-1.5ex}{$x$}   & 1     & 3/56  & 18/56 & 9/56  & 0  \\
%    & 2     & 6/56  & 9/56  & 0     & 0  \\
%    & 3     & 1/56  & 0     & 0     & 0  \\ \hline
%\end{array}
%\]
%
%\begin{parts}
%\part Find the conditional PDF of $X$ given $Y=0$.
%\begin{answer}
%\[
%\begin{array}{c|cccc}
%x          		& 0     & 1     & 2     & 3     \\ \hline
%f_{X|Y=0}(x)		& 0     & 3/10  & 6/10  & 1/10  \\
%\end{array}
%\]
%\end{answer}
%
%\part Find the conditional PMF of $Y$ given $X=1$.
%\begin{answer}
%\[
%\begin{array}{c|cccc}
%y            	& 0     & 1     & 2     & 3     \\ \hline
%f_{Y|X=1}(y)		& 1/10  & 6/10  & 3/10  & 0     \\
%\end{array}
%\]
%\end{answer}
%\end{parts}

%----------------------------------------
\question
Let $X$ and $Y$ be two independent discrete random variables with the following PMFs:
\begin{center}
\begin{minipage}{\linewidth}
\begin{minipage}{0.4\linewidth}
\[
\begin{array}{|c|cc|} \hline
x     & 1     & 2   \\ \hline
f_X(x)  & 1/3   & 2/3 \\ \hline
\end{array}
\]
\end{minipage}
%\text{\qquad\qquad}
\begin{minipage}{0.4\linewidth}
\[
\begin{array}{|c|ccc|} \hline
y     & -1    & 0     & 1     \\ \hline
f_Y(y)  & 1/4   & 1/2   & 1/4   \\ \hline
\end{array}
\]
\end{minipage}
\end{minipage}
\end{center}

\begin{parts}
%--------------------
\part Compute the joint PMF of $X$ and $Y$.
\begin{answer}
Since $X$ and $Y$ are independent we have that $f_{X,Y}(x,y)=f_X(x)f_Y(y)$, which yields the following joint distribution.
\[
\begin{array}{|cc|ccc|c|}\hline
    &       &       & y     &       &     \\
    &       & -1    & 0     & 1     & f_X(x)    \\ \hline
\raisebox{-1.0ex}{$x$}   & 1     & 1/12  & 1/6   & 1/12  & 1/3       \\
    & 2     & 1/6   & 1/3   & 1/6   & 2/3       \\ \hline
    & f_Y(y)& 1/4   & 1/2   & 1/4   & 1         \\ \hline
\end{array}
\]
\end{answer}
%--------------------
\part Compute the joint PMF of the random variables $U=1/X$ and $V=Y^2$.
\begin{answer}
Let $f_{U,V}$ denote the joint PMF of $U$ and $V$. Clearly, $U$ takes the values $0.5$ and $1$, while $V$ takes the values $0$ and $1$. We compute (for example)
\[
f_{U,V}(0.5,1) = f_{X,Y}(2,-1) + f_{X,Y}(2,1) = 1/6 + 1/6 = 1/3.
\]
Thus we obtain joint PMF shown in the following table:
\[
\begin{array}{|cc|cc|c|} \hline
    &       & \multicolumn{2}{c|}{v} &   \\
    &       & 0     & 1     & f_U(u)    \\ \hline
\raisebox{-1.0ex}{$u$}   & 1/2   & 1/3   & 1/3   & 2/3       \\
    & 2     & 1/6   & 1/6   & 1/3       \\ \hline
    & f_V(v)& 1/2   & 1/2  & 1          \\ \hline
\end{array}
\]
\end{answer}
%--------------------
\part Show that $U$ and $V$ are independent.
\begin{answer}
The marginal PMFs $f_U(u)$ and $f_V(v)$ are computed by summing the rows and columns of the joint PMF table. Thus we see that $U$ and $V$ are independent because $f_{U,V}(u,v) = f_U(u)f_V(v)$ for every pair of values $(u,v)$.
\end{answer}
%--------------------
\end{parts}


%----------------------------------------
\question
Two discrete random variables $X$ and $Y$ have the following joint PMF:
\[
f_{X,Y}(x,y) = \left\{\begin{array}{ll}
	c|x+y| 	& \text{for}\quad x,y\in\{-2,-1,0,1,2\}, \\
	0		& \text{otherwise,}
\end{array}\right.	
\]
where $c$ is a constant.
\begin{parts}
%--------------------
\part Show that $c=1/40$.
\begin{answer}
First we tabulate the values of $|x+y|$:
\begin{center}
\begin{tabular}{|cc|ccccc|}\hline
	&		&		&		& $y$	&		&		\\
	&		& -2		& -1		& 0		& 1		& 2 		\\ \hline
	& -2		& 4 		& 3		& 2 		& 1		& 0		\\
	& -1		& 3		& 2		& 1		& 0		& 1		\\
$x$	& 0		& 2		& 1		& 0		& 1		& 2		\\
	& 1		& 1		& 0		& 1		& 2		& 3		\\
	& 2		& 0		& 1		& 2		& 3		& 4		\\ \hline
\end{tabular}
\end{center}	
Because the probabilities must sum to $1$, it follows that $c=1/40$. 
\end{answer}

%--------------------
\part Find $\prob(X=0,Y=-2)$.
\begin{answer}
The joint PMF of $X$ and $Y$, along with their marginal distributions, are shown in the following table,
\begin{center}
\begin{tabular}{|cc|ccccc|c|}\hline
	&		&		&		& $y$	&		&		&			\\
	&		& -2		& -1		& 0		& 1		& 2 		&			\\ \hline
	& -2		& 4/40	& 3/40	& 2/40	& 1/40	& 0		&	10/40 	\\
	& -1		& 3/40	& 2/40	& 1/40	& 0		& 1/40	&	 7/40 	\\
$x$	& 0		& 2/40	& 1/40	& 0		& 1/40	& 2/40	&	 6/40 	\\
	& 1		& 1/40	& 0		& 1/40	& 2/40	& 3/40	&	 7/40 	\\
	& 2		& 0		& 1/40	& 2/40	& 3/40	& 4/40	&	10/40 	\\ \hline
	&		& 10/40	& 7/40	& 6/40	& 7/40	& 4/40	& \\ \hline
\end{tabular}
\end{center}	
Hence 
\[
\prob(X =0,Y=-2) = f_{X,Y}(0,-2) = 2/40 = 1/20.
\]
\end{answer}
%--------------------
\part Find $\prob(X=2)$.
\begin{answer}
$\prob(X=2) = f_X(2) = 10/40 = 1/4$
\end{answer}
%--------------------
\part Find $\prob(|X-Y|\leq 1)$.
\begin{answer}
\begin{align*}
\prob(|X-Y|\leq 1) 
	& = \prob(-1\leq X-Y\leq 1) \\
	& = \prob(X-Y = -1,0\text{ or } 1) \\
	& = \prob\big((X = Y - 1)\,{\cup}\,(X = Y)\,{\cup}\,(X = Y + 1)\big) \\
	& = 8/40 + 12/40 + 8/40 \\
	& = 7/10.
\end{align*}
\end{answer}
%--------------------

\end{parts}


%----------------------------------------
\question
Two continuous random variables $X$ and $Y$ have the joint PDF
\[
f_{X,Y}(x,y) = \left\{\begin{array}{ll}
	2x 	& \text{if}\quad 0\leq x,y\leq 1 \\
	0		& \text{otherwise.}
\end{array}\right.	
\]

\begin{parts}
%--------------------
\part Find the conditional distribution of $Y$ given that $X=x$.
\begin{answer}
The marginal PDF of $X$ is 
\[
f_{X}(x) 
	= \int_{y}f(x,y)\,dy 
	= \int_{0}^{1}2x\,dy 
	= \big[2xy\big]_{0}^{1} 
	= 2x \quad\text{for } 0\leq x\leq 1.
\]
The conditional PDF of $Y$ given $X = x$ is defined as 
\[
f_{Y|X}(y|x) 
	= \frac{f(x,y)}{f_{X}(x)} 
	= \frac{2x}{2x} 
	= 1 \quad\text{for } 0\leq y\leq 1
\]
Thus the conditional PDF of $Y$ given $X = x$ is $f_{Y|X}(y|x) = 1$ for $0\leq y\leq 1$, and because this is independent of $x$, the random variables $X$ and $Y$ are independent.

The marginal PDF of $Y$ is 
\[
f_{Y}(y) = \int_{0}^{1}2x\,dx = \big[x^{2}\big]_{0}^{1} = 1 \quad\text{for } 0\leq y\leq 1,
\]
and the conditional PDF of $X$ given $Y = y$ is
\[
f_{X|Y}(x|y) = 2x \quad\text{for } 0\leq x\leq 1
\]
\end{answer}
%--------------------
\part Find $\prob(Y\leq 0.5 | X=0.5)$ and $\prob(Y\leq 0.5 | X=0.75)$.
\begin{answer}
Because $X$ and $Y$ are independent ,
\[
\prob(Y\leq 0.5 | X=0.5)
	= \prob(Y\leq 0.5 | X=0.75)
	= \int_{0}^{0.5}1\,dy  
	= 0.5
\]	
In general, when X and Y are not independent, the probability that $Y\leq 0.5$ would depend on the value taken by $X$.	
\end{answer}

%--------------------
\part Find the marginal distribution of $Y$ and hence find $\prob(Y\leq 0.5)$.
\begin{answer}
\[
\prob(Y {\leq} 0.5) = \int_{0}^{0.5}1\,dy = \big[y\big] _{0}^{0.5} = 0.5
\]
\end{answer}
%--------------------
\end{parts}

%----------------------------------------
\question
Two continuous random variables $X$ and $Y$ have the following joint PDF:
\[
f_{X,Y}(x,y) = \left\{\begin{array}{ll}
	c(x^2+y)		& \text{when $-1\leq x\leq 1$ and $0\leq y \leq 1-x^2$}, \\
	0			& \text{otherwise.}
\end{array}\right.	
\]
where $c$ is a constant.

\begin{parts}
%--------------------
\part Show that $c=5/4$.
\begin{answer}
The joint PDF is non-zero over the region between the curve $y=1-x^2$ and the $x$-axis. 
%\par\resizebox{0.5\linewidth}{!}{\includegraphics{ex11q4a}}\par
\bit
\it For $x\in[-1,1]$ fixed, we integrate $y$ over the range $0\leq y\leq 1-x^2$.
\it For $y\in[0,1]$ fixed, we integrate $x$ over the range $-\sqrt{1-y}\leq x\leq +\sqrt{1-y}$.
\eit
To find $c$, the joint PDF must integrate to 1. 
\begin{align*}
\int\int f(x,y)\,dx\,dy = 1
	& \Rightarrow \int_{x=-1}^{1}\int_{y=0}^{1-x^{2} }c(x^{2} +y)\,dx\,dy = 1 \\ 
	& \Rightarrow \int_{-1}^{1}\left[ c\left( x^{2} y+\frac{y^{2} }{2} \right) \right]  _{0}^{1-x^{2} }\,dx = 1 \\
	& \Rightarrow \int_{-1}^{1}c\left( x^{2} (1-x^{2} )+\frac{(1-x^{2} )^{2} }{2} \right)\,dx = 1 \\
	& \Rightarrow \int_{-1}^{1}c\left( x^{2} -x^{4} +\frac{1}{2} -x^{2} +\frac{x^{4} }{2} \right)\,dx=1 \\
	& \Rightarrow \int_{-1}^{1}\frac{c}{2}  (1-x^{4} )\,dx = 1 \\
	& \Rightarrow \left[ \frac{c}{2} \left( x-\frac{x^{5} }{5} \right) \right] _{-1}^{1} = 1 \\
	& \Rightarrow \frac{c}{2} \left( 1-\frac{1}{5} +1-\frac{1}{5} \right) = 1 \\
\end{align*}	
Thus $c = 5/4$.
\end{answer}

%--------------------
\part Find $\prob(0\leq X\leq 0.5)$.
\begin{answer}
To calculate $\prob(0\leq X\leq 1/2)$, we first compute the marginal PDF of X.
\begin{align*}
f_{X}(x)
	& = \int_{0}^{1-x^{2} }\frac{5}{4}  (x^{2} +y)\,dy \\
	& = \left[ \frac{5}{4} \left( x^{2} y+\frac{y^{2} }{2} \right) \right] _{0}^{1-x^{2} } \\
	& = \frac{5}{8} (1-x^{4})
	\quad\text{for }-1\leq x\leq 1.
\end{align*}
Then we integrate this over $x$ in the range $[0,1/2]$:
\begin{align*}
\prob(0 {\leq} X {\leq} 1/2)
	& = \int_{0}^{1/2}\frac{5}{8}  (1-x^{4} )\,dx \\
	& = \left[\frac{5}{8} \left( x-\frac{x^{5} }{5}\right)\right]_{0}^{1/2} = \frac{79}{256}
\end{align*}
\end{answer}

\part Find $\prob(Y\leq X+1)$.
%--------------------
\begin{answer}
$Y\leq X+1$ is satisfied for pairs $(X,Y)$ in the region between the curves $y=1+x$ and the $x$-axis when $-1\leq x\leq 0$, and between the curve $y=1-x^2$ and the $x$-axis when $0\leq x\leq 1$. 

\bit
\it For fixed $x\in[-1,0]$, we must integrate $y$ over the range $0\leq y\leq 1+x$.
\it For fixed $x\in[0,+1]$, we must integrate $y$ over the range $0\leq y\leq 1-x^2$.
\eit
%\par\resizebox{0.5\linewidth}{!}{\includegraphics{ex11q4b}}\par
\fbox{\parbox{\linewidth}{\hfill It helps to draw a plot of the curves $y=1+x$ and $y=1-x^2$ \hfill\mbox{}}}

\begin{align*}
\prob(Y\leq X + 1)
	&  = \int_{-1}^{0}\int_{0}^{1+x} f(x,y)\,dy\,dx   +\int_{0}^{1}\int_{0}^{1-x^{2} }f(x,y)\,dy\,dx \\
	&  = \int_{-1}^{0}\int_{0}^{1+x}\frac{5}{4} (x^{2} +y)\,dy\,dx   +\int_{0}^{1}\int_{0}^{1-x^{2} }\frac{5}{4}   (x^{2} +y)\,dy\,dx \\
	& =  \int_{-1}^{0}\left[ \frac{5}{4} \left( x^{2} y+\frac{y^{2} }{2} \right) \right] _{0}^{1+x}  dx+\int_{0}^{1}\left[ \frac{5}{4} \left( x^{2} y+\frac{y^{2} }{2} \right) \right]  _{0}^{1-x^{2} } dx \\
	& = \int_{-1}^{0}\frac{5}{4} \left( x^{2} (1+x)+\frac{(1+x)^{2} }{2} \right)  dx+\int_{0}^{1}\frac{5}{4} \left( x^{2} (1-x^{2} )+\frac{(1-x^{2} )^{2} }{2} \right)  dx \\
	& = \int_{-1}^{0}\frac{5}{4} \left( x^{2} +x^{3} +\frac{1}{2} +x+\frac{x^{2} }{2} \right)  dx+\int_{0}^{1}\frac{5}{4} \left( x^{2} -x^{4} +\frac{1}{2} -x^{2} +\frac{x^{4} }{2} \right)  dx \\
	& = \left[ \frac{5}{4} \left( \frac{x}{2} +\frac{x^{2} }{2} +\frac{x^{3} }{2} +\frac{x^{4} }{4} \right) \right] _{-1}^{0} +\left[ \frac{5}{4} \left( \frac{x}{2} -\frac{x^{5} }{10} \right) \right] _{0}^{1}  \\
	& = \frac{13}{16}
\end{align*}	
\end{answer}

\part Find $\prob(Y=X^2)$.
%--------------------
\begin{answer}
$\prob(Y=X^2)=0$. This is because the region in the $(x, y)$ plane over which we integrate is a curve, $y = x^{2}$ (which has zero area). If we integrate first with respect to $y$, the range of integration is $x^{2}$ to $x^{2}$, giving zero for the definite integral. This is analagous to the calculation of $P(X = k)$ for a continuous univariate random variable (in which case the answer is again 0).\end{answer}

\end{parts}


%----------------------------------------
\end{questions}
\end{exercise}
%----------------------------------------------------------------------

%======================================================================
\endinput
%======================================================================
