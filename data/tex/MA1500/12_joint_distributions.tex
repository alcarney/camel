% !TEX root = main.tex
%======================================================================
\chapter{Joint Distributions}\label{chap:joint}
%======================================================================

Let $X$ and $Y$ be simple random variables, let $\{x_1,x_2,\ldots,x_m\}$ be the range of $X$, and let $\{y_1,y_2,\ldots,y_n\}$ be the range of $Y$.

%----------------------------------------------------------------------
\section{Joint distributions}
%----------------------------------------------------------------------
% defn: joint cdf
\begin{definition}
\ben
\it The \emph{joint CDF} of $X$ and $Y$ is the function
\[
\begin{array}{cccc}
F_{X,Y}:	& \R^2	& \to		& [0,1] \\[1ex]
			& (x,y) & \mapsto	& \prob(X\leq x, Y\leq y).
\end{array}
\]
\it The \emph{marginal CDF} of $X$ is the function
\[
\begin{array}{cccc}
F_X:	& \mathbb{R} 	& \to		& [0,1] \\[1ex]
		& x				& \mapsto	& \prob(X\leq x),
\end{array}
\]
and the marginal CDF of $Y$ is
\[
%\qquad\text{and}\qquad
\begin{array}{cccc}
F_Y:	& \mathbb{R}	& \to		& [0,1] \\[1ex]
		& y				& \mapsto	& \prob(Y\leq y).
\end{array}
%\qquad\text{respectively.}
\]

%\it The \emph{marginal} CDFs of $X$ and $Y$ are the functions
%\[
%\begin{array}{cccc}
%F_X:	& \mathbb{R} 	& \to		& [0,1] \\[1ex]
%		& x				& \mapsto	& \prob(X\leq x)
%\end{array}
%\qquad\text{and}\qquad
%\begin{array}{cccc}
%F_Y:	& \mathbb{R}	& \to		& [0,1] \\[1ex]
%		& y				& \mapsto	& \prob(Y\leq y)
%\end{array}
%\qquad\text{respectively.}
%\]
%respectively.
\een
\end{definition}

\begin{remark}%[Notation]
$\prob(X\leq x, Y\leq y)$ is the probability of the event $\{\omega: X(\omega)\leq x \text{ and } Y(\omega)\leq y\}$,
%Recall the following notation:
%\bit
%\it $\prob(X\leq x)$ is the probability of the event $\{\omega: X(\omega)\leq x\}$,
%\it $\prob(Y\leq y)$ is the probability of the event $\{\omega: Y(\omega)\leq y\}$.
%\it $\prob(X\leq x, Y\leq y)$ is the probability of the event $\{\omega: X(\omega)\leq x \text{ and } Y(\omega)\leq y\}$,
%\eit
%Note that the joint CDF is a function of two variables:
%\[
%\begin{array}{cccl}
%F_{X,Y}:	& \mathbb{R}^2	& \longrightarrow	& [0,1] \\
%		& (x,y) 			& \mapsto			& \prob(X\leq x, Y\leq y).
%\end{array}
%\]
\end{remark}

For simple random variables, it is often easier to work with joint PMFs:
% defn: joint pmf
\begin{definition}
%Let $X,Y:\Omega\to\R$ be two random variables on a finite probability space $(\Omega,\prob)$.
\ben
\it The \emph{joint PMF} of $X$ and $Y$ is the function
\[
\begin{array}{cccc}
F_{X,Y}:	& \R^2	& \to		& [0,1] \\[1ex]
			& (x,y) & \mapsto	& \prob(X=x, Y=y).
\end{array}
\]
\it The \emph{marginal PMF} of $X$ is the function
\[
\begin{array}{cccc}
f_X:	& \mathbb{R} 	& \to		& [0,1] \\[1ex]
		& x				& \mapsto	& \prob(X=x),
\end{array}
\]
and the marginal PMF of $Y$ is
\[
%\qquad\text{and}\qquad
\begin{array}{cccc}
f_Y:	& \mathbb{R}	& \to		& [0,1] \\[1ex]
		& y				& \mapsto	& \prob(Y=y).
\end{array}
%\qquad\text{respectively.}
\]
%\it The \emph{marginal} PMFs of $X$ and $Y$ are the functions
%\[
%\begin{array}{cccc}
%f_X:	& \mathbb{R} 	& \to		& [0,1] \\[1ex]
%		& x				& \mapsto	& \prob(X=x)
%\end{array}
%\qquad\text{and}\qquad
%\begin{array}{cccc}
%f_Y:	& \mathbb{R}	& \to		& [0,1] \\[1ex]
%		& y				& \mapsto	& \prob(Y=y).
%\end{array}
%\qquad\text{respectively.}
%\]
%\begin{align*}
%f_X(x)	& = \prob(X=x) = \prob(\{\omega\in\Omega : X(\omega)=x\}), \\
%f_Y(y)	& = \prob(Y=y) = \prob(\{\omega\in\Omega : Y(\omega)=y\})
%\end{align*}
%respectively.
\een
\end{definition}

% lemma
\begin{lemma}
%Let $f_{X,Y}$ be the joint PMF of two simple random variables $X$ and $Y$. 
The marginal PMFs of $X$ and $Y$ satisfy
\[
f_X(x_i) = \sum_{j=1}^n f_{X,Y}(x_i,y_j) \quad\text{and}\quad f_Y(y_j) = \sum_{i=1}^m f_{X,Y}(x_i,y_j).
\]
\end{lemma}

% proof
\begin{proof}
By definition, the joint PMF of $X$ and $Y$ can be written as $f_{X,Y}(x_i,y_j) = \prob(A_{i,j})$, where
\[
A_{i,j} = \{X=x_i, Y=y_j\} \equiv \{\omega: X(\omega)=x_i \text{ and } Y(\omega) = y_j\}.
\]
The sets $A_{i,1}, A_{i,2},\ldots,A_{i,n}$ form a partition of the event $\{X=x_i\}$, so by the partition theorem,
\[
%f_X(x_i) = \prob(X=x_i) = \sum_{j=1}^n \prob(X=x_i,Y=y_j) = \sum_{j=1}^n f_{X,Y}(x_i,y_j), \\
f_X(x_i) = \prob(X=x_i) = \sum_{j=1}^n \prob(A_{i,j}) = \sum_{j=1}^n f_{X,Y}(x_i,y_j). \\
\]
Similarly, $A_{1,j}, A_{2,j},\ldots,A_{m,j}$ form a partition of the event $\{Y=y_j\}$, so by the partition theorem,
\[
%f_Y(y_j) = \prob(Y=y_j) = \sum_{i=1}^m \prob(X=x_i,Y=y_j) = \sum_{i=1}^m f_{X,Y}(x_i,y_j).
f_Y(y_j) = \prob(Y=y_j) = \sum_{i=1}^m \prob(A_{i,j}) = \sum_{i=1}^m f_{X,Y}(x_i,y_j).
\]
\qed
\end{proof}

% example: dice
\begin{example}\label{ex:joint:dice}
A fair die is rolled once. Let $\omega$ denote the outcome, and consider the random variables
\[
X(\omega) = \left\{\begin{array}{cl}
	1 & \text{ if $\omega$ is odd,} \\
	2 & \text{ if $\omega$ is even,}
\end{array}\right. 
\quad\mbox{ and }\quad
Y(\omega) = \left\{\begin{array}{cl}
	1 & \text{ if $\omega\leq 3$,} \\
	2 & \text{ if $\omega\geq 4$.}
\end{array}\right.
\]
Find the joint PMF of $X$ and $Y$.
\end{example}

\begin{solution}
\[
\begin{array}{c|cccccc}
\omega 	& 1 & 2 & 3 & 4 & 5 & 6 \\ \hline
X(\omega) 	& 1 & 2 & 1 & 2 & 1 & 2 \\
Y(\omega) 	& 1 & 1 & 1 & 2 & 2 & 2 \\
\end{array}
\]
The joint PMF of $X$ and $Y$ is shown in the following table.
\[
\begin{array}{|c|c|c|} \hline
	& Y=1 	& Y=2 \\ \hline
X=1	& 1/3	& 1/6 \\ \hline
X=2	& 1/6 	& 1/3 \\ \hline
\end{array}
\]
The marginal PMFs are recovered by summing the rows and columns of the table.
\end{solution}

%----------------------------------------------------------------------
\section{Independent random variables}
%----------------------------------------------------------------------
%\bit
%\it 
%Recall that two events $A$ and $B$ are \emph{independent} if the fact that $B$ occurs does not affect the probability that $A$ occurs, i.e.\ if and only if $\prob(A\cap B)=\prob(A)\prob(B)$.
%\it
%Similarly, we say that two random variables $X$ and $Y$ are independent if the value taken by $X$ does not affect the distribution of $Y$.
%\eit
%Let $(\Omega,\prob)$ be a finite probability space. Two random variables $X,Y:\Omega\to\R$ 
Two random variables are said to be \emph{independent} if the value taken by one does not affect the distribution of the other.% (and vice versa).
%\bit
%\it Recall that two events $A$ and $B$ are called \emph{independent} if $\prob(A\cap B)=\prob(A)\prob(B)$.
%\eit



%% defn: independence (general rvs)
%\begin{definition}
%Two random variables $X,Y:\Omega\to\R$ are said to be \emph{independent} if the events 
%\begin{align*}
%\{X\leq x\} & = \{\omega\,:\, X(\omega)\leq x\} \\
%\{Y\leq y\} & = \{\omega\,:\, Y(\omega)\leq y\}
%\end{align*}
%are independent for all $x,y\in\R$.
%\end{definition}

% defn: independence
\begin{definition}
$X$ and $Y$ are said to be \emph{independent} if the events $\{X = x\}$ and $\{Y = y\}$ are independent for all $x,y\in\R$.
%\begin{align*}
%\{X = x\} & = \{\omega\,:\, X(\omega) = x\} \\
%\{Y = y\} & = \{\omega\,:\, Y(\omega) = y\}
%\end{align*}
\end{definition}

%The following is a trivial consequence of the definition.
% lemma: independence => joint = product of marginals.
%If $X$ and $Y$ are independent, their joint PMF is equal to the product of their marginal PMFs:
The joint PMF of two independent random variables is equal to the product of their marginal PMFs:
\begin{lemma}
$X$ and $Y$ are independent if and only if
\[
f_{X,Y}(x,y) = f_X(x)\,f_Y(y) \quad\text{for all}\quad x,y\in\R.
%\prob(X=x,Y=y) = \prob(X=x)\prob(Y=y)\quad\text{for all}\quad x,y\in\R.
\]
\end{lemma}

\begin{proof} 
Recall that two events $A$ and $B$ are said to be independent if $\prob(A\cap B)=\prob(A)\prob(B)$. Thus $X$ and $Y$ are independent if and only if $\prob(X=x, Y=y) = \prob(X=x)\prob(Y=y)$ for all $x,y\in\R$, or equivalently
\[
f_{X,Y}(x,y) = f_X(x)\,f_Y(y) \quad\text{for all}\quad x,y\in\R.
%\prob(X=x,Y=y) = \prob(X=x)\prob(Y=y)\quad\text{for all}\quad x,y\in\R.
\]
Note that if $x\notin\{x_1,x_2,\ldots,x_m\}$ or $y\notin\{y_1,y_2,\ldots,y_n\}$, both sides are equal to zero.
\end{proof} 

%% remark
%\begin{remark}
%To check whether two random variables $X$ and $Y$ are independent, we first calculate the marginal PMFs, then check whether or not 
%$\prob(X=x,Y=y) = \prob(X=x)\prob(Y=y)$ for all $x,y\in\R$.
%%$f_{X,Y}(x,y) = f_X(x)f_Y(y)$.
%\end{remark}

% example
\begin{example}\label{ex:tedious}
Let $X$ and $Y$ be random variables with a joint PMF shown in the following table. 
Find the marginal PMFs of $X$ and $Y$, and decide whether or not $X$ and $Y$ are independent.
\[\begin{array}{|cc|ccc|}\hline
	&	& 		& y 		&			\\ 
	&	& 2		& 3		& 4		\\\hline
	& 1	& 1/12	& 1/6	& 0 		\\
x	& 2	& 1/6	& 0		& 1/3 	\\
	& 3	& 1/12	& 1/6  	& 0		\\ \hline
\end{array}\]
\end{example}

\begin{solution}
The marginal distributions are obtained by summing the rows and columns of the table:
\[\begin{array}{|c|ccc|}\hline
x			& 1		& 2		& 3	\\ \hline
f_X(x)		& 1/4	& 1/2	& 1/4	\\ \hline \hline
y			& 2		& 3		& 4 	\\ \hline			
f_Y(y)		& 1/3	& 1/3	& 1/3	\\ \hline
\end{array}\]
We have (for example) that $\prob(X=2,Y=3)=0$ but $\prob(X=2)\prob(Y=3) = 1/6$, so $X$ and $Y$ are not independent.
\end{solution}

% theorem: functions of rvs
\begin{theorem}
Let $X$ and $Y$ be independent, and let $g,h:\R\to\R$ be any two functions. Then $g(X)$ and $h(Y)$ are also independent.
\end{theorem}
\begin{proof}

Let $\{c_1,c_2,\ldots,c_M\}$ be the range of $g(X)$, let $\{d_1,d_2,\ldots,d_N\}$ be the range of $h(Y)$, and consider the following partitions.
\bit
\it The sets $C_k=\{x_i:g(x_i)=c_k\}$ for $k=1,2,\ldots,M$. 
\par These partition the set $\{x_i:i=1,2,\ldots,m\}$, which is the range of $X$.
\it The sets $D_l=\{y_j:h(y_j)=d_l\}$ for $l=1,2,\ldots,N$. 
\par These partition the set $\{y_j:j=1,2,\ldots,n\}$, which is the range of $Y$.
\it The sets $E_{k,l} = \{(x_i,y_j):g(x_i)=c_k,h(y_j)=d_l\}$ for $k=1,2,\ldots,M$ and $l=1,2,\ldots,N$.
\par These partition the set of pairs $\{(x_i,y_j):i=1:2,\ldots,m ; j=1,2,\ldots,n\}$.
\eit
By construction, $g(x)=c_k$ and $h(y)=d_l$ if and only if $(x,y)\in E_{k,l}$, so by the partition theorem,
\begin{align*}
\prob\big[g(X)=c_k,h(Y)=d_l\big]
%	& = \sum_{\stackss{x,y:}{g(x)=a,h(y)=b}}\prob(X=x,Y=y) \\
	& = \sum_{(x,y)\in E_{k,l}}\prob(X=x,Y=y) \\
	& = \sum_{(x,y)\in E_{k,l}}\prob(X=x)\prob(Y=y)\qquad\text{(by independence),}\\
	& = \sum_{x\in C_k}\prob(X=x) \sum_{y\in D_l}\prob(Y=y) \\
	& = \prob\big[g(X)=c_k\big]\,\prob\big[h(Y)=d_l\big].
\end{align*}
%as required.
%\par

%
%Let $a,b\in\R$. We sum over all pairs $(x_i,y_j)$ for which $g(x_i)=a$ and $h(y_j)=b$.
%\par
%Let $\{c_1,c_2,\ldots,c_M\}$ be the range of $g(X)$, let $\{d_1,d_2,\ldots,d_N\}$ be the range of $h(Y)$, and define
%\bit
%\it $J_k=\{i:g(x_i)=c_k\}$,
%\it $J_l=\{j:h(y_j)=d_l\}$ and 
%\it $J_{k,l} = \{(j,k):g(x_i)=c_k,h(y_j)=d_l\}$.
%\eit
%Then
%\begin{align*}
%\prob\big[g(X)=c_k,h(Y)=d_l\big]
%%	& = \sum_{\stackss{x,y:}{g(x)=a,h(y)=b}}\prob(X=x,Y=y) \\
%	& = \sum_{(i,j)\in J_{k,l}}\prob(X=x_i,Y=y_j) \\
%	& = \sum_{(i,j)\in J_{k,l}}\prob(X=x_i)\prob(Y=y_j)\qquad\text{(by independence),}\\
%	& = \sum_{i\in J_k}\prob(X=x_i) \sum_{j\in J_l}\prob(Y=y_j) \\
%	& = \prob\big[g(X)=c_k\big]\,\prob\big[h(Y)=d_l\big].
%\end{align*}
%\par
%Let 
%\bit
%\it $J(a)=\{x:g(x)=a\}$,
%\it $J(b)=\{y:h(y)=b\}$ and 
%\it $J(a,b) = \{(x,y):g(x)=a,h(y)=b\}$.
%\eit
%Then
%\begin{align*}
%\prob\big[g(X)=a,h(Y)=b\big]
%%	& = \sum_{\stackss{x,y:}{g(x)=a,h(y)=b}}\prob(X=x,Y=y) \\
%	& = \sum_{(x,y)\in J(a,b)}\prob(X=x,Y=y) \\
%	& = \sum_{(x,y)\in J(a,b)}\prob(X=x)\prob(Y=y) \qquad\text{(by independence),}\\
%	& = \sum_{x\in J(a)}\prob(X=x) \sum_{y\in J(b)}\prob(Y=y) \\
%	& = \prob\big[g(X)=a\big]\,\prob\big[h(Y)=b\big].
%\end{align*}
\end{proof}

%----------------------------------------------------------------------
\newpage
\section{Exercises}
% !TEX root = main.tex
% ex12_joint_distributions.tex
\begin{exercise}
\begin{questions}
%----------------------------------------
\question
A fair coin is tossed twice. Let $X$ be the number of heads, and let $Y$ be the indicator variable of the event $\{X=2\}$. Find the joint PMF of $X$ and $Y$.
\begin{answer}
The possible outcomes, along with the associated values taken by $X$ and $Y$, are shown in the following table:
\[
\begin{array}{|c|cccc|}\hline
\omega		& TT		& TH		& HT 	& HH		\\ \hline
X(\omega)	& 0		& 1 		& 1		& 2 		\\ 
Y(\omega)	& 0		& 0		& 0		& 1		\\ \hline
\end{array}
\]
Hence the joint PMF of $X$ and $Y$ is as follows:
\[
\begin{array}{|cc|ccc|} \hline
	&		& 		& x 	& 		\\
	& 		& 0 		& 1		& 2 		\\ \hline
y	& 0		& 1/4	& 1/2	& 0		\\ 
	& 1		& 0 		& 0		& 1/4	\\ \hline
\end{array}
\]
\end{answer}


%----------------------------------------
\question % GS 2.5.2
Let $X$ be a Bernoulli random variable with parameter $p$. %Then $\prob(X=0)=1-p$ and $\prob(X=1)=p$.
\begin{parts}
%--------------------
\part Let $Y=1-X$. Find the joint PMF of $X$ and $Y$.
\begin{answer}
\[
f_{X,Y}(x,y) = \left\{\begin{array}{ll}
	p	& \text{if } (x,y) = (1,0) \\
	1-p	& \text{if } (x,y) = (0,1) \\
	0	& \text{otherwise.}
\end{array}\right.
\]
\end{answer}
%--------------------
\part Let $Z=X(1-X)$. Find the joint PMF of $X$ and $Z$.
\begin{answer}
\[
f_{X,Z}(x,z) = \begin{cases}
	p	& \text{if } (x,z) = (1,0) \\
	1-p	& \text{if } (x,z) = (0,0) \\
	0	& \text{otherwise.}
\end{cases}
\]
\end{answer}
%--------------------
\end{parts}

%%----------------------------------------
%\question % GS 2.7.7
%Airlines find that each passenger who reserves a seat fails to turn up with probability $0.1$, independently of other passengers. To avoid empty seats, EasyJet always sell 10 tickets for their 9-seater aeroplane, while Ryanair always sell 20 tickets for their 18-seater aeroplane. Which of the two airlines is most often overbooked?
%\begin{answer}
%Let $X$ and $Y$ denote the (random) number of people on an EasyJet and Ryanair flight respectively. Then $X\sim\text{Binomial}(10,0.9)$ and $Y\sim\text{Binomial}(20,0.9)$, so
%\begin{align*}
%\prob(X=k)	& = \binom{10}{k}\left(\frac{9}{10}\right)^k\left(1-\frac{9}{10}\right)^{10-k} \\
%\prob(Y=k)	& = \binom{20}{k}\left(\frac{9}{10}\right)^k\left(1-\frac{9}{10}\right)^{20-k}
%\end{align*}
%Thus
%\begin{align*}
%\prob(\text{EasyJet flight is overbooked}) 
%	& = \prob(X=10) = \left(\frac{9}{10}\right)^{10} = 0.3487 \\
%\prob(\text{Ryanair flight is overbooked}) 
%	& = \prob(Y=19)+\prob(Y=20) \\
%	& = 20\left(\frac{9}{10}\right)^{19}\left(\frac{1}{10}\right) + \left(\frac{9}{10}\right)^{20} = 0.3917
%\end{align*}	  
%so Ryanair is overbooked more often than EasyJet.
%\end{answer}

%%==========================================================================
%\question
%Let $X$ and $Y$ be two random variables with joint PMF given by the following table:
%\[
%\begin{array}{|cc|cccc|}\hline
%    &       &       & \multicolumn{2}{c}{y} &   \\
%    &       & 0     & 1     & 2     & 3     \\ \hline
%    & 0     & 0     & 3/56  & 6/56  & 1/56  \\
%\raisebox{-1.5ex}{$x$}   & 1     & 3/56  & 18/56 & 9/56  & 0  \\
%    & 2     & 6/56  & 9/56  & 0     & 0  \\
%    & 3     & 1/56  & 0     & 0     & 0  \\ \hline
%\end{array}
%\]
% 
%\begin{parts}
%%--------------------
%\part Find the conditional PMF of $X$ given $Y=0$.
%\begin{answer}
%The conditional PMF $\prob(X=x\,|\,Y=0)$ is as follows:
%\[
%\begin{array}{c|cccc}
%x          		& 0     & 1     & 2     & 3     \\ \hline
%f_{X|Y}(x|0)		& 0     & 3/10  & 6/10  & 1/10  \\
%\end{array}
%\]
%\end{answer}
%%--------------------
%\part Find the conditional PMF of $Y$ given $X=1$.
%\begin{answer}
%The conditional PMF $\prob(Y=y\,|\,X=1)$ is as follows:
%\[
%\begin{array}{c|cccc}
%y            	& 0     & 1     & 2     & 3     \\ \hline
%f_{Y|X}(y|1)		& 1/10  & 6/10  & 3/10  & 0     \\
%\end{array}
%\]
%\end{answer}
%%--------------------
%\end{parts}
%


%==========================================================================
\question
Let $X$ and $Y$ be two independent random variables with PMFs
%\begin{center}
%\begin{minipage}{\linewidth}
%\begin{minipage}{0.48\linewidth}
\[
\begin{array}{c|cc}
x     & 1     & 2   \\ \hline
f_X(x)  & 1/3   & 2/3 \\
\end{array}
%\]
%\end{minipage}
\text{\qquad and \qquad}
%\begin{minipage}{0.48\linewidth}
%\[
\begin{array}{c|ccc}
y     & -1    & 0     & 1     \\ \hline
f_Y(y)  & 1/4   & 1/2   & 1/4   \\
\end{array}
\text{\qquad respectively.}
\]
%\end{minipage}
%\end{minipage}
%\end{center}

\begin{parts}
%--------------------
\part Compute the joint PMF of $X$ and $Y$.
\begin{answer}
Since $X$ and $Y$ are independent we have that $f_{X,Y}(x,y)=f_X(x)f_Y(y)$, which yields the following joint PMF.
\[
\begin{array}{|cc|ccc|c|}\hline
    &       &       & y     &       &     \\
    &       & -1    & 0     & 1     & f_X(x)    \\ \hline
\raisebox{-1.0ex}{$x$}   & 1     & 1/12  & 1/6   & 1/12  & 1/3       \\
    & 2     & 1/6   & 1/3   & 1/6   & 2/3       \\ \hline
    & f_Y(y)& 1/4   & 1/2   & 1/4   & 1         \\ \hline
\end{array}
\]
\end{answer}
%--------------------
\part Compute the joint PMF of the random variables $U=1/X$ and $V=Y^2$. 
\begin{answer}
Let $f_{U,V}(u,v)$ denote the joint PMF of $U$ and $V$. Clearly, $U$ takes the values $0.5$ and $1$, while $V$ takes the values $0$ and $1$. We compute (for example)
\[
f_{U,V}(0.5,1) = f_{U,V}(2,-1) + f_{U,V}(2,1) = 1/6 + 1/6 = 1/3
\]
to get the joint PMF $f_{U,V}$ shown in the following table:
\[
\begin{array}{|cc|cc|c|} \hline
    &       & \multicolumn{2}{c|}{v} &   \\
    &       & 0     & 1     & f_U(u)    \\ \hline
\raisebox{-1.0ex}{$u$}   & 1/2   & 1/3   & 1/3   & 2/3       \\
    & 2     & 1/6   & 1/6   & 1/3       \\ \hline
    & f_V(v)& 1/2   & 1/2  & 1          \\ \hline
\end{array}
\]
\end{answer}
%--------------------
\part Show that $U$ and $V$ are independent.
\begin{answer}
The marginal PMFs $f_U(u)$ and $f_V(v)$ are computed by summing the rows and columns of the joint PMF table. From here, we see that $U$ and $V$ are independent because $f_{U,V}(u,v) = f_U(u)f_V(v)$ for every pair of values $(u,v)$.
\end{answer}
%--------------------
\end{parts}

%==========================================================================
\question
The random variables $X$ and $Y$ have the joint PMF
\[
f(x,y) = \left\{\begin{array}{ll}
	c|x+y| 	& \text{if}\quad x,y\in\{-2,-1,0,1,2\} \\
	0		& \text{otherwise,}
\end{array}\right.	
\]
where $c$ is a constant.
\begin{parts}
%--------------------
\part 
Find the value of $c$.
\begin{answer}
First we tabulate the values of $|x+y|$:
\[
\begin{array}{|cc|ccccc|}\hline
	&		&		&		& y		&		&		\\
	&		& -2		& -1		& 0		& 1		& 2 		\\ \hline
	& -2		& 4 		& 3		& 2 		& 1		& 0		\\
	& -1		& 3		& 2		& 1		& 0		& 1		\\
x	& 0		& 2		& 1		& 0		& 1		& 2		\\
	& 1		& 1		& 0		& 1		& 2		& 3		\\
	& 2		& 0		& 1		& 2		& 3		& 4		\\ \hline
\end{array}\]
Because the probabilities must sum to $1$, it follows that $c=1/40$. 
\par
Hence the joint PMF of $X$ and $Y$, along with their marginal PMFs, are as follows:
\[
\begin{array}{|cc|ccccc|c|}\hline
	&		&		&		& y		&		&		&			\\
	&		& -2		& -1		& 0		& 1		& 2 		&			\\ \hline
	& -2		& 4/40	& 3/40	& 2/40	& 1/40	& 0		&	10/40 	\\
	& -1		& 3/40	& 2/40	& 1/40	& 0		& 1/40	&	 7/40 	\\
x	& 0		& 2/40	& 1/40	& 0		& 1/40	& 2/40	&	 6/40 	\\
	& 1		& 1/40	& 0		& 1/40	& 2/40	& 3/40	&	 7/40 	\\
	& 2		& 0		& 1/40	& 2/40	& 3/40	& 4/40	&	10/40 	\\ \hline
	&		& 10/40	& 7/40	& 6/40	& 7/40	& 4/40	& \\ \hline
\end{array}
\]
\end{answer}
%--------------------
\part 
Find $\prob(X=0,Y=-2)$.
\begin{answer}
$\prob(X = 0 \text{ and } Y = -2) = f(0, -2) = 2/40 = 1/20$.
\end{answer}
%--------------------
\part 
Find $\prob(X=2)$.
\begin{answer}
$\prob(X=2) = f_X(2) = 10/40 = 1/4$
\end{answer}
%--------------------
\part 
Find $\prob(|X-Y|\leq 1)$.
\begin{answer}
\begin{align*}
\prob(|X-Y|\leq 1) 
	& = \prob(-1\leq X-Y\leq 1) \\
	& = \prob(X-Y = -1,0\text{ or } 1) \\
	& = \prob\big((X = Y - 1)\,{\cup}\,(X = Y)\,{\cup}\,(X = Y + 1)\big) \\
	& = 8/40 + 12/40 + 8/40 \\
	& = 7/10
\end{align*}
\end{answer}
%--------------------
\end{parts}

\end{questions}
\end{exercise}
%======================================================================
\endinput
%======================================================================

%----------------------------------------------------------------------

%======================================================================
\endinput
%======================================================================
