% !TEX root = main.tex
%======================================================================
\chapter{Quantiles}\label{chap:quantiles}
%======================================================================

Let $X$ be a continuous random variable, and let $F$ denote its CDF.
\bit
\it Recall that CDFs are \emph{increasing} functions: $x<y \ \Rightarrow\ F(x)\leq F(y)$.
\it In this lecture, we will assume that $F$ is \emph{strictly increasing}: $x<y \ \Rightarrow\ F(x)<F(y)$.
\it This condition ensures that the inverse function $F^{-1}:[0,1]\to\R$ exists.
\it The ideas presented here can be extended to discrete distributions, with some modifications.
\eit

%----------------------------------------------------------------------

\section{Median}
%----------------------------------------------------------------------
% location
The mean of a distribution can be thought of as its \emph{location}.
\bit
\it The expected \emph{squared deviation} of $X$ from the point $c$ is given by $\expe\big[(X-c)^2\big]$.
\it The mean of $X$ is the value of $c$ that minimises the expected squared deviaiton. 
\eit

\vspace{2ex}
Another location parameter is provided by a \emph{median} of a distribution.
\bit
\it The expected \emph{absolute deviation} of $X$ from the point $c$ is given by $\expe\big[|X-c|\big]$.
\it The median of $X$ is a value of $c$ that minimises the expected absolute deviaiton. 
\eit

% defn: median
\begin{definition}
A median of a distribution is a value $\eta\in\R$ such that
\[
\displaystyle F(\eta) = \frac{1}{2}.
\] 
\end{definition}

\begin{remark}
If $F$ is continuous and strictly increasing, the median is uniquely defined.
\end{remark}


%----------------------------------------------------------------------
\section{Quantiles}
%----------------------------------------------------------------------
The $q$-quantiles of a distribution divide the real line into $q$ `equal' parts:

% defn: q-quantiles
\begin{definition}
For $q\in\N$, the $q$-quantiles of a distribution are values $x_1 < x_2 < \ldots < x_{q-1}$ such that 
\[
F(x_k) = \frac{k}{q} \qquad\text{for}\quad k=1,2,\ldots,q-1
\]
\end{definition}

\begin{remark}
If $F$ is continuous and strictly increasing, $q$-quantiles are uniquely defined.
\end{remark}

In particular,
\bit
\it The $2$-quantile is the \emph{median} of $F$.
\it The $4$-quantiles are the called the \emph{quartiles} of $F$.
\it The $100$-quantiles are called the \emph{percentiles} (or \emph{percentage points}) of $F$.
\eit

%----------------------------------------------------------------------
\subsection{Quartiles}
%----------------------------------------------------------------------
\begin{definition}
%Let $X$ be a random variable with distribution function $F$.
\ben
\it 
The \emph{lower quartile} of $F$ is a number $x_L$ such that $\displaystyle F(x_L) = \frac{1}{4}$.
\it
%The \emph{median} of $F$ is a number $\eta$ such that $\displaystyle F(\eta) = \prob(X \leq \eta) = \frac{1}{2}$.
%\it
The \emph{upper quartile} of $F$ is a number $x_U$ such that $\displaystyle F(x_U) = \frac{3}{4}$.
\it 
The \emph{inter-quartile range} of $F$ is the difference $x_U-x_L$ between its upper and lower quartiles.
\een
\end{definition}

% location and scale
\begin{remark}
\bit
\it The median quantifies the \emph{location} of a distribution.
\it The inter-quartile range quantifies the \emph{size} or \emph{scale} of a distribution.
%\it[]
\it The lower quartile is the median of the lower half of the distribution.
\it The upper quartile is the median of the upper half of the distribution.
\eit
\end{remark}


% example: quartiles
\begin{example}
A continuous random variable $X$ has the following PDF:
$
f(x) = \begin{cases}
	3x^2/8		& \text{if }\ 0\leq x\leq 2, \\
	0		& \text{otherwise}.	
\end{cases}
$
\par
Find the median and the interquartile range of this distribution.
%Find
%\ben
%\it the value of $c$,
%\it the median, and 
%\it the interquartile range.
%\een
\end{example}

\begin{solution}
\ben
%\it % << (i)
%$\displaystyle\int_{-\infty}^{\infty} f(x)\,dx = c\int_0^2 x^2\,dx = c\left[\frac{x}{3}\right]_0^2 = \frac{8c}{3}$.\par
%Since $f(x)$ is a probability mass function, its integral of $f(x)$ over $\R$ is equal to $1$, so $c=3/8$.
\it % << (ii)
The median $\eta$ satisfies
\[
F(\eta) = \frac{3}{8}\int_0^{\eta} x^2\,dx = \frac{1}{2}
\quad\Rightarrow\quad
\frac{3}{8}\times\frac{\eta^3}{3} = \frac{1}{2}
\quad\Rightarrow\quad
\eta = 4^{1/3} = 1.5874.
\]
\it % << (iii)
The lower quartile $x_L$ satisfies
\[
F(x_L) = \frac{3}{8}\int_0^{x_L} x^2\,dx = \frac{1}{4}
\quad\Rightarrow\quad
\frac{3}{8}\times\frac{x_L^3}{3} = \frac{1}{4}
\quad\Rightarrow\quad
x_U = 2^{1/3} = 1.2599.
\]
The upper quartile $x_U$ satisfies
\[
F(x_U) = \frac{3}{8}\int_0^{x_U} x^2\,dx = \frac{3}{4}
\quad\Rightarrow\quad
\frac{3}{8}\times\frac{x_U^3}{3} = \frac{3}{4}
\quad\Rightarrow\quad
x_L = 6^{1/3} = 1.8171.
\]
Hence the inter-quartile range is equal to $x_U-x_L = 1.8171 - 1.2599 = 0.5572$.
\een
\end{solution}

%----------------------------------------------------------------------
%\newpage
\subsection{Percentiles}
%----------------------------------------------------------------------
\begin{definition}
The $k$th percentile of $F$ is a value $x_k$ such that
\[
F(x_k) = \frac{k}{100} \qquad\text{for}\quad k=1,2,\ldots,99
\]
In particular, 
\bit
\it The $25$th percentile is the lower quartile.
\it The $50$th percentile is the median.
\it The $75$th percentile is the upper quartile.
\eit
\end{definition}


%----------------------------------------------------------------------
\section{The quantile function}
%----------------------------------------------------------------------

% quantile function
\begin{definition}
Let $F$ be a continuous and stricty increasing CDF.
The \emph{quantile function} is the inverse of $F$:
\[
\begin{array}{cccl}
Q:	& [0,1]	& \longrightarrow	& \mathbb{R}	 [0,1] \\
	& p 		& \mapsto			& F^{-1}(p) 
\end{array}
\]
\end{definition}

For any $p\in[0,1]$, $Q(p)$ is the value of $x$ for which $F(x)=p$.
\ben
\it $Q(p)$ called the \emph{critical point} of the distribution at level $p$, and is often denoted by $x_p$.
\een

% upper and lower tails
For example, the critical points at levels $p=0.05$ and $p=0.95$ satisfy
\[\begin{array}{rlll}
F(x_{0.05}) & = \prob(X\leq x_{0.05}) & = 0.05. 		& \quad\text{(lower tail)} \\
1-F(x_{0.95}) & = \prob(X\geq x_{0.95}) & = 0.05. 	& \quad\text{(upper tail)}
\end{array}\]
%Note that $x_{0.05}$ and $x_{0.95}$ are respectively the $5$th and $95$th percentiles of $F$.

% typical and extreme events
\begin{remark}
\bit
\it The event $\{x_{0.05} < X < x_{0.95}\}$ occurs with probability $0.9$: this is a \emph{typical} event.
\it The event $\{X\leq x_{0.05}\}\cup\{X\geq x_{0.95}\}$ occurs with probability $0.1$" this is an \emph{extreme} event.
\it Confidence intervals and statistical hypothesis tests are constructed on this basis.
\eit
\end{remark}

% example: exponential distribution
\begin{example}
Let $X$ have (negative) exponential distribution with rate parameter $\lambda$. %This has distribution function
\[
F(x) = \begin{cases}
	1 - e^{-\lambda x}	& x > 0, \\
	0					& \text{otherwise.}
\end{cases}
\]
Derive an explicit expression for the quantile function of $F$.
\end{example}

\begin{solution}
Let $x_p = Q(p)$ where $p\in[0,1]$. Then
\begin{align*}
p = F(x_p) 
	& \ \Rightarrow\ p = 1 - e^{-\lambda x_p} \\
	& \ \Rightarrow\ e^{-\lambda x_p} = 1 - p \\
	& \ \Rightarrow\ -\lambda x_p = \log(1 - p) \\
	& \ \Rightarrow\ x_p = -\frac{\log(1 - p)}{\lambda}.
\end{align*}

In particular,
\[
x_{0.05} = \frac{\log(20/19)}{\lambda}, \quad
x_{0.25} = \frac{\log(4/3)}{\lambda}, \quad
x_{0.50} = \frac{\log(2)}{\lambda}, \quad
x_{0.75} = \frac{\log(4)}{\lambda}, \quad
x_{0.95} = \frac{\log(20)}{\lambda}.
\]
%\[
%\begin{array}{lll}
%%x_{0.05}		& = \displaystyle\frac{\log(20/19)}{\lambda} 		& \text{($5$th percentile).}\\
%%x_{0.05}		& = \displaystyle\frac{\log(20/19)}{\lambda} 		& \text{($5$th percentile).}\\[1ex]
%x_{0.25}		& = \displaystyle\frac{\log(4/3)}{\lambda} 		& \text{(lower quartile).}\\
%x_{0.50}		& = \displaystyle\frac{\log(2)}{\lambda} 			& \text{(median).}\\
%x_{0.75}		& = \displaystyle\frac{\log(4)}{\lambda} 			& \text{(upper quartile).}\\
%%x_{0.95}		& = \displaystyle\frac{\log(20)}{\lambda} 		& \text{($95$th percentile).}	
%\end{array}
%\]
\end{solution}
\normalsize

\newpage

% statistical tables
\begin{remark}[Statistical Tables]
\bit
\it
For many important distributions, it is not possible to derive explicit (closed-form) expressions for their quantile functions. 
\it
In such cases, critical points must be estimated using numerical approximation techniques. 
\it
For a number of standard distributions, tables of such critical points are available.
\it
These tables list critical points for various values of $p\in[0,1]$.
% are available.
\it 
In recent years, statistical tables have been supplanted by statistical software packages.
\eit
\end{remark}

%----------------------------------------------------------------------
\section{Exercises}
% !TEX root = main.tex
% ex19_quantiles.tex
\begin{exercise}
\begin{questions}
%----------------------------------------
%--------------------
% MODE AND MEDIAN
% RND 107
\question
Let $X$ be a continuous random variable. 
\bit
\it A \emph{mode} of $X$ is a number $a\in\R$ such that $f(a)\geq f(x)$ for all $x\in\R$. %(it may not be unique).
%\it A \emph{median} of $X$ is a number $m\in\R$ such that $F(m)=\prob(X\leq m) = 1/2$. % (it may not be unique).
\eit
Find a mode and a median of a random variable having the following PDF:
\[
f(x) = \begin{cases}
\displaystyle\frac{2}{3}\cos\left(x-\frac{\pi}{6}\right)	& \text{if }\ 0\leq x \leq \displaystyle\frac{2\pi}{3}, \\[2ex]
0											& \text{otherwise.}
\end{cases}
\]
\begin{answer}
\ben
\it % << mode
A mode is a value that maximises the PDF. We locate stationary points of $f(x)$ by finding its derivative and setting this to zero:
\[
f'(x) = -\frac{2}{3}\sin\left(x-\frac{\pi}{6}\right) = 0.
\]
For $x\in [0,2\pi/3]$, we have $\sin(x-\pi/6) = 0$ only when $x=\pi/6$, so $\frac{\pi}{6}$ is a mode of $X$. (Note that the mode is unique in this example, but this need not always be the case.)
\it % << mdeian
A median is a value that divides the probability distribution into two halves. To find the median, we first need to find the CDF of $X$:
\begin{align*}
F(x) = \int_{-\infty}^x f(u)\,du
	& = \frac{2}{3}\int_0^x \cos(u-\pi/6)\,du \\
	& = \frac{2}{3}\Big[\sin(u-\pi/6)\Big]_0^x \\
	& = \frac{2}{3}\Big(\sin(x-\pi/6) - \sin(-\pi/6)\Big) \\
	& = \frac{2}{3}\sin\left(x-\frac{\pi}{6}\right) - \frac{1}{3} \qquad\text{because $\sin(\pi/6)=1/2$.}
\end{align*}
The median satisfies 
\[
F(m) = \frac{2}{3}\sin\left(m-\frac{\pi}{6}\right) - \frac{1}{3} = \frac{1}{2}, 
\qquad\text{i.e.}\quad 
\sin\left(m-\frac{\pi}{6}\right) = \frac{1}{4}.
\]
Thus a median of $X$ is 
\[
m = \frac{\pi}{6} + \sin^{-1}\left(\frac{1}{4}\right) = \frac{\pi}{6} + 0.2527 = 0.7763.
\]
Note that the median is unique in this example, but this need not always be true. (For example, suppose $X$ has PDF $f(x)=1/2$ for $x\in[0,1]\cup[2,3]$, and zero otherwise. Then every $m\in[1,2]$ is a median for this distribution, i.e.\ any $m\in[1,2]$ separates the probability into two equal parts.)
\een
\end{answer}

%--------------------
% MEDIAN
% RND 4.1.5
\question
A continuous random variable $X$ has PDF
\[
f(x) = \begin{cases}
	ce^{-x}	& \text{if }\ x\geq 0,\\
	0		& \text{otherwise,}		
\end{cases}
\]
where $c$ is a constant. 
\ben
\it Find the value of $c$.
\it Find the CDF of $X$.
\it Find the median of the distribution.
\een
\begin{answer}
\ben
\it % << (i)
Since $f$ is a PDF we need that $\int_{-\infty}^{\infty} f(x)\,dx = 1$:
\[
\int_{-\infty}^{\infty} f(x)\,dx 
	= c\int_0^{\infty} e^{-x}\,dx 
	= c\big[-e^{-x}\big]_0^{\infty}
	= c, \qquad\text{so}\quad c = 1.
\]
(This is the exponential distribution with mean $1$.)
\it % << (ii)
The CDF of $X$ is
\[
F(x) = \begin{cases}
	\int_{-\infty}^{x} f(u)\,du = \int_0^x e^{-u}\,du = 1 - e^{-x}	& \text{if }\ x\geq 0, \\
	0	& \text{otherwise.}
\end{cases}
\]	
\it % << (iii)
The median $m$ satisfies $F(m)=1/2$, and
\[
F(m) = 1 - e^{-m} = \frac{1}{2} \quad\Rightarrow\quad m = \log 2 = 0.69315.
\]
\een
\end{answer}


%--------------------
% QUARTILES OF THE EXPONENTIAL DISTRIBUTION
% RND 4.1.9
\question
Suppose that $X$ has the exponential distribution with (rate) parameter $\lambda>0$. The PDF of $X$ is
\[
f(x) = \begin{cases}
	\lambda e^{-\lambda x} 	& \text{if }\ x\geq 0,\\
	0		  				& \text{otherwise.}		
\end{cases}
\]
Find the median and the inter-quartile range of this distribution.

\begin{answer}
The median $m$ satisfies $F(m)=1/2$, where $F(x)=1-e^{-\lambda x}$ is the CDF of $X$. Thus
\[
1 - e^{-\lambda m} = \frac{1}{2}
\quad\text{so}\quad
m = \frac{1}{\lambda}\log 2.
\]
Let $x_U$ and $x_L$ denote the upper and lower quartiles. Then
\bit
\it $F(x_L) = 1/4$, so $x_L = \displaystyle\frac{1}{\lambda}\log 4$, and
\it $F(x_U) = 3/4$, so $x_U = \displaystyle\frac{1}{\lambda}\log \frac{4}{3}$.
\eit
Thus the inter-quartile range is 
\[
x_U - x_L 
	= \frac{1}{\lambda}\log \frac{4}{3} - \frac{1}{\lambda}\log 4
	= \frac{1}{\lambda}\log 3.
\]
\end{answer}


\end{questions}
\end{exercise}

%======================================================================
\endinput
%======================================================================

%----------------------------------------------------------------------

%======================================================================
\endinput
%======================================================================
