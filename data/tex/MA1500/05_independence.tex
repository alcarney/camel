% !TEX root = main.tex
%======================================================================
\chapter{Independence}\label{chap:independence}
%======================================================================

%----------------------------------------------------------------------
\section{Independence}
%----------------------------------------------------------------------
If the probability that event $A$ occurs is \emph{not} affected by whether or not event $B$ occurs, then
$
\prob(A|B) = \prob(A).
$
In such cases, we say that events $A$ and $B$ are \emph{independent}:

% definition: independence
\begin{definition}
Two events $A$ and $B$ are said to be \emph{independent} if $\prob(A\cap B) = \prob(A)\prob(B)$.
\end{definition}

% lemma
\begin{lemma}
If $A$ and $B$ are independent, then $A$ and $B^c$ are also independent.
\end{lemma}
% proof
\begin{proof}
\[\begin{array}{ll}
\prob(A\cap B^c) 
%	= \prob(A\setminus B)
	= \prob(A) - \prob(A\cap B)
	& = \prob(A) - \prob(A)\prob(B) \quad\text{by independence,} \\
	& = \prob(A)(1-\prob(B))
	= \prob(A)\prob(B^c).
\end{array}\]
\end{proof}

\begin{example}
A fair die is rolled once. Let $A$ be the event that the outcome is an even number, and let $B$ be the event that the outcome is divisible by $3$. Are $A$ and $B$ independent?
\end{example}
\begin{solution}
\bit
\it $\prob(A) = \prob(\{2,4,6\}) = 1/2$, $\prob(B) = \prob(\{3,6\}) = 1/3$, $\prob(A\cap B) = \prob(\{6\}) = 1/6$.
\it Thus we see that $\prob(A\cap B) = \prob(A)\prob(B)$, so $A$ and $B$ are independent.
\eit
\end{solution}

%----------------------------------------------------------------------
\section{Pairwise independence and total indepencence}
%----------------------------------------------------------------------

\begin{definition}
A family of events $\{A_1,A_2,\ldots\}$ is said to be 
\ben
\it \emph{pairwise independent} if $\prob(A_i\cap A_j)=\prob(A_i)\prob(A_j)$ for all $i\neq j$.
\it \emph{totally independent} if, for every finite subset 
$\{B_1,B_2,\ldots,B_m\}\subseteq \{A_1,A_2,\ldots\}$, we have
\[
\prob(B_1\cap B_2\cap \ldots \cap B_m) = \prob(B_1)\prob(B_2)\cdots\prob(B_m).
\]
This can also be written as 
$\prob\left(\bigcap_{j=1}^m B_j\right) = \prod_{j=1}^m \prob(B_j).$
\een
\end{definition}
%
%\begin{definition}
%A collection of events $\{A_1,A_2,\ldots,A_n\}$ is called to be \emph{pairwise independent} if 
%\[
%\prob(A_i\cap A_j)=\prob(A_i)\prob(A_j)
%\]
%for every pair of distinct events $A_i$ and $A_j$ (i.e.\ with $i\neq j$), and \emph{totally independent} if 
%\[
%\prob(A_{i_1}\cap A_{i_2}\cap \ldots \cap A_{i_m}) = \prob(A_{i_1})\prob(A_{i_2})\cdots\prob(A_{i_m})
%\]
%for every possible sub-collection of events $\{A_{i_1},A_{i_2},\ldots,A_{i_m}\}\subseteq\{A_1,A_2,\ldots,A_n\}$.
%\end{definition}

%\newpage
% problem: de Mere's paradox
\begin{example}[de M\'{e}r\'{e}'s Paradox]
Show that you are more likely to obtain a six in 4 rolls of a single fair die, than to obtain a double-six in 24 rolls of two fair dice.
\end{example}
\begin{solution}
Assume that the rolls are totally independent of each other.
%\begin{align*}
\[\begin{array}{ll}
\prob(\text{at least one six in 4 rolls of a single die)}
	& = 1- \prob(\text{no sixes obtained in 4 rolls}) \\
	& = 1 - (5/6)^4 = 0.5177. \\
\prob(\text{at least one double six in 24 rolls of two dice}) 
	& = 1- \prob(\text{no double-sixes in 24 rolls}) \\
	& = 1 - (35/36)^{24}  = 0.4914. 
\end{array}\]
%\end{align*}
\end{solution}


%\newpage % <<

% example
\begin{example}
Consider a sample space $\Omega=\{1,2,3,4\}$ where each outcome is equally likely. Let $A=\{1,2\}$, $B=\{1,3\}$ and $C=\{1,4\}$. Show that $\{A,B,C\}$ is pairwise independent, but not totally independent.
\end{example}
% solution
\begin{solution}
\bit
\it $\prob(A)=1/2$ and $\prob(B)=1/2$
\it $\prob(A\cap B)=\prob(\{1\})=1/4$. 
\it Hence $\prob(A\cap B) = \prob(A)\prob(B)$, so $A$ and $B$ are independent. 
\it Similarly, $\prob(A\cap C) = \prob(A)\prob(C)$ and $\prob(B\cap C) = \prob(B)\prob(C)$.
\eit
Thus the set $\{A,B,C\}$ is pairwise independent. 
\bit
\it $\prob(A\cap B\cap C)=\prob(\{1\}) = 1/4$
\it However, $\prob(A)\prob(B)\prob(C)=1/8$. 
\it Hence $\prob(A\cap B\cap C)\neq \prob(A)\prob(B)\prob(C)$.
\eit
Thus the set $\{A,B,C\}$ is not totally independent.
\end{solution}

%----------------------------------------------------------------------
\section{Exercises}
%----------------------------------------------------------------------

%----------------------------------------
\begin{exercise}
\begin{questions}
%----------------------------------------
% cond prob
\question
A fair six-sided die is rolled repeatedly. How many times should it be rolled to ensure that the probability of getting a six is at least 0.8?

\begin{answer}
For convenience, let $p=1/6$ denote the probability that the die shows a six.
\par
Let $A_1$ be the event that we observe a six on the first roll, $A_2$ that we observe a six on the second roll, and so on. 
\par
Let $A$ be the event that we observe at least one six in $n$ rolls. The complementary event $A^c$ is the event that we do not observe a six on any of the $n$ rolls. This can be written as
\[
A^c = A_1^c\cap A_2^c\cap\ldots\cap A_n^c
\]
If we assume that the rolls are independent of each other, the set of events $\{A_1,A_2,\ldots,A_n\}$ is totally independent, and hence $\{A_1^c,A_2^c,\ldots,A_n^c\}$is also totally independent. The probability of getting at least one six in $n$ rolls is therefore given by 
\begin{align*}
\prob(A) 
	& = 1 - \prob(A^c) \\
	& = 1 - \prob(A_1^c\cap A_2^c\cap\ldots\cap A_n^c) \\
	& = 1 - \prob(A_1^c)\prob(A_2^c)\cap\ldots\cap A_n^c) \\
	& = 1 - [1-\prob(A_1)][1-\prob(A_2)]\cdots[1-\prob(A_n)] \\
	& = 1 - (1-p)^n \\
\end{align*}

Let $n$ denote the number of rolls. To find the required value of $n$, we need that $\prob(A)\geq 0.8$, i.e.\
\begin{align*}
1-(1-p)^n \geq 0.8	
	& \Rightarrow (1-p)^n \leq 0.2 \\
	& \Rightarrow n\log (1-p) \leq \log 0.2 \\
	& \Rightarrow n\geq \frac{\log 0.2}{\log(1-p)} = \frac{\log 0.2}{\log(5/6)} \approx 8.827 \\
\end{align*}

Therefore we should roll the die at least $n=9$ times.
\end{answer}

%----------------------------------------
% independence
\question
A multiple choice test has five questions, with each question having four alternative
choices. At least three questions must be answered correctly to pass the test.
If a candidate chooses her answers at random, what is the probability that she passes the test? State any assumptions you make.
\begin{answer}
Assumption: The choices are independent of each other. 
\par
Let $A_i$ be the event that exactly $i$ questions are answered correectly ($i=0,1,2,3,4,5$), and let $A$ be the event that the student passes the test. Then
\begin{align*}
\prob(A) 	
	& = \prob(A_3\cup A_4\cup A_5) \\
	& = \prob(A_3) + \prob(A_4) + \prob(A_5) \qquad\text{(because events $A_3$, $A_4$ and $A_5$ are disjoint)} \} \\
	& = 10(0.25)^3(0.75)^2 + 5(0.25)^4(0.75) + (0.25)^5 \\
	& = 0.1035.
\end{align*}
\end{answer}

%----------------------------------------
\question
Two fair dice are rolled. Show that the event that their sum is $7$ is independent of the score shown on the first die.
\begin{answer}
For every $j\in\{1,2,3,4,5,6\}$,
\bit
\it $\prob(\text{first die shows $j$ and sum is $7$}) = \frac{1}{36}$
\it $\prob(\text{first die shows $j$})\prob(\text{sum is $7$}) = \frac{1}{6}\times\frac{1}{6} = \frac{1}{36}$
\eit
Alternatively, let $A$ be the event that the sum is $7$ and let $B$ be the event that any score is shown on the first die. Then $\prob(A)=1/6$ and  $\prob(B)=1$, while $\prob(A\cap B)=1/6$, so $A$ and $B$ are independent.
\end{answer}

%----------------------------------------
\question
A fair die is rolled twice, each roll being independent of the other. Let $A$ be the event that the first roll shows $3$, let $B$ be the event that the second roll shows $4$, and let $C$ be the event that the total of the two rolls is $7$.
\begin{parts}
\part Define a suitable sample space, and identify the subsets corresponding to events $A$, $B$ and $C$.
\begin{answer}
\bit
\it $\Omega = \{(a,b):1\leq a,b\leq 6\}$
\it $A = \{(3,b):1\leq b\leq 6\}$
\it $B = \{(a,4):1\leq a\leq 6\}$
\it $C = \{(a,b):1\leq a,b\leq 6\mbox{ and }a+b=7\}$
\eit
The sample space $\Omega$ contains $36$ outcomes, and the events $A$, $B$ and $C$ each contains $6$ outcomes. Because each outcome is equally likely, $\prob(A)=\prob(B)=\prob(C)=1/6$.
\end{answer}

\part Show that $\{A,B,C\}$ is pairwise independent but not totally independent.
\begin{answer}
To show that the set $\{A,B,C\}$ is pairwise independent, we need to show that any two events chosen from the set are independent of each other. For $A$ and $B$, their intersection $A\cap B$ is the event $\{(3,4)\}$, consisting of the single outcome $(3,4)$. This means that $\prob(A\cap B)=1/36$ and therefore
\[
\prob(A\cap B) = \prob(A)\prob(B)
\]
so $A$ and $B$ are pairwise independent. Similarly, $A\cap C=\{(3,4)\}$ which implies that $\prob(A\cap C)=\prob(A)\prob(C)$, and $B\cap C=\{(3,4)\}$ implies that $\prob(B\cap C)=\prob(B)\prob(C)$.  Hence, the set $\{A,B,C\}$ is pairwise independent.  However, the intersection of all three events is also $\{(3,4)\}$ so $\prob(A\cap B\cap C)=1/36$ and hence
\[
\prob(A\cap B\cap C) \neq \prob(A)\prob(B)\prob(C)
\]
so the set $\{A,B,C\}$ is \emph{not} independent.
\end{answer}
\end{parts}

%----------------------------------------
\question
A coin has probability $p$ of showing heads. Let $q_n$ be the probability that in $n$ independent tosses, a head is observed an even number of times (for this question, take $0$ to be an even number). Using the partition theorem, show that 
\[
q_n = p(1-q_{n-1}) + (1-p)q_{n-1}\quad\text{for any}\quad n\geq 1.
\]
\begin{answer}
Let $n\geq 1$, let $A_1,\ldots,A_n$ be any sequence of tosses, and let $N(A_1,\ldots,A_n)$ be the number of heads in the sequence. Using the fact that $A_n$ is independent of the previous tosses $A_1,\ldots,A_{n-1}$, it follows that the number $N(A_1,\ldots,A_n)$ is even if and only if
\bit
\it $A_n$ is heads and $N(A_1,\ldots,A_{n-1})$ is odd, which occurs with probability $p(1-q_{n-1})$, or
\it $A_n$ is tails and $N(A_1,\ldots,A_{n-1})$ is even, which occurs with probability $(1-p)q_{n-1}$
\eit
Hence
\[
q_n = p(1-q_{n-1}) + (1-p)q_{n-1}
\]
as required.
\end{answer}

%----------------------------------------
\end{questions}
\end{exercise}
%----------------------------------------

%======================================================================
\endinput
%======================================================================
