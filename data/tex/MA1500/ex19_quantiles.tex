% !TEX root = main.tex
% ex19_quantiles.tex
\begin{exercise}
\begin{questions}
%----------------------------------------
%--------------------
% MODE AND MEDIAN
% RND 107
\question
Let $X$ be a continuous random variable. 
\bit
\it A \emph{mode} of $X$ is a number $a\in\R$ such that $f(a)\geq f(x)$ for all $x\in\R$. %(it may not be unique).
%\it A \emph{median} of $X$ is a number $m\in\R$ such that $F(m)=\prob(X\leq m) = 1/2$. % (it may not be unique).
\eit
Find a mode and a median of a random variable having the following PDF:
\[
f(x) = \begin{cases}
\displaystyle\frac{2}{3}\cos\left(x-\frac{\pi}{6}\right)	& \text{if }\ 0\leq x \leq \displaystyle\frac{2\pi}{3}, \\[2ex]
0											& \text{otherwise.}
\end{cases}
\]
\begin{answer}
\ben
\it % << mode
A mode is a value that maximises the PDF. We locate stationary points of $f(x)$ by finding its derivative and setting this to zero:
\[
f'(x) = -\frac{2}{3}\sin\left(x-\frac{\pi}{6}\right) = 0.
\]
For $x\in [0,2\pi/3]$, we have $\sin(x-\pi/6) = 0$ only when $x=\pi/6$, so $\frac{\pi}{6}$ is a mode of $X$. (Note that the mode is unique in this example, but this need not always be the case.)
\it % << mdeian
A median is a value that divides the probability distribution into two halves. To find the median, we first need to find the CDF of $X$:
\begin{align*}
F(x) = \int_{-\infty}^x f(u)\,du
	& = \frac{2}{3}\int_0^x \cos(u-\pi/6)\,du \\
	& = \frac{2}{3}\Big[\sin(u-\pi/6)\Big]_0^x \\
	& = \frac{2}{3}\Big(\sin(x-\pi/6) - \sin(-\pi/6)\Big) \\
	& = \frac{2}{3}\sin\left(x-\frac{\pi}{6}\right) - \frac{1}{3} \qquad\text{because $\sin(\pi/6)=1/2$.}
\end{align*}
The median satisfies 
\[
F(m) = \frac{2}{3}\sin\left(m-\frac{\pi}{6}\right) - \frac{1}{3} = \frac{1}{2}, 
\qquad\text{i.e.}\quad 
\sin\left(m-\frac{\pi}{6}\right) = \frac{1}{4}.
\]
Thus a median of $X$ is 
\[
m = \frac{\pi}{6} + \sin^{-1}\left(\frac{1}{4}\right) = \frac{\pi}{6} + 0.2527 = 0.7763.
\]
Note that the median is unique in this example, but this need not always be true. (For example, suppose $X$ has PDF $f(x)=1/2$ for $x\in[0,1]\cup[2,3]$, and zero otherwise. Then every $m\in[1,2]$ is a median for this distribution, i.e.\ any $m\in[1,2]$ separates the probability into two equal parts.)
\een
\end{answer}

%--------------------
% MEDIAN
% RND 4.1.5
\question
A continuous random variable $X$ has PDF
\[
f(x) = \begin{cases}
	ce^{-x}	& \text{if }\ x\geq 0,\\
	0		& \text{otherwise,}		
\end{cases}
\]
where $c$ is a constant. 
\ben
\it Find the value of $c$.
\it Find the CDF of $X$.
\it Find the median of the distribution.
\een
\begin{answer}
\ben
\it % << (i)
Since $f$ is a PDF we need that $\int_{-\infty}^{\infty} f(x)\,dx = 1$:
\[
\int_{-\infty}^{\infty} f(x)\,dx 
	= c\int_0^{\infty} e^{-x}\,dx 
	= c\big[-e^{-x}\big]_0^{\infty}
	= c, \qquad\text{so}\quad c = 1.
\]
(This is the exponential distribution with mean $1$.)
\it % << (ii)
The CDF of $X$ is
\[
F(x) = \begin{cases}
	\int_{-\infty}^{x} f(u)\,du = \int_0^x e^{-u}\,du = 1 - e^{-x}	& \text{if }\ x\geq 0, \\
	0	& \text{otherwise.}
\end{cases}
\]	
\it % << (iii)
The median $m$ satisfies $F(m)=1/2$, and
\[
F(m) = 1 - e^{-m} = \frac{1}{2} \quad\Rightarrow\quad m = \log 2 = 0.69315.
\]
\een
\end{answer}


%--------------------
% QUARTILES OF THE EXPONENTIAL DISTRIBUTION
% RND 4.1.9
\question
Suppose that $X$ has the exponential distribution with (rate) parameter $\lambda>0$. The PDF of $X$ is
\[
f(x) = \begin{cases}
	\lambda e^{-\lambda x} 	& \text{if }\ x\geq 0,\\
	0		  				& \text{otherwise.}		
\end{cases}
\]
Find the median and the inter-quartile range of this distribution.

\begin{answer}
The median $m$ satisfies $F(m)=1/2$, where $F(x)=1-e^{-\lambda x}$ is the CDF of $X$. Thus
\[
1 - e^{-\lambda m} = \frac{1}{2}
\quad\text{so}\quad
m = \frac{1}{\lambda}\log 2.
\]
Let $x_U$ and $x_L$ denote the upper and lower quartiles. Then
\bit
\it $F(x_L) = 1/4$, so $x_L = \displaystyle\frac{1}{\lambda}\log 4$, and
\it $F(x_U) = 3/4$, so $x_U = \displaystyle\frac{1}{\lambda}\log \frac{4}{3}$.
\eit
Thus the inter-quartile range is 
\[
x_U - x_L 
	= \frac{1}{\lambda}\log \frac{4}{3} - \frac{1}{\lambda}\log 4
	= \frac{1}{\lambda}\log 3.
\]
\end{answer}


\end{questions}
\end{exercise}

%======================================================================
\endinput
%======================================================================
