% !TEX root = main.tex
% ex10_moments.tex
\begin{exercise}
\begin{questions}
%----------------------------------------
\question
Let $X$ be a random variable with the following PMF, where $c$ is a constant:
\[
\begin{array}{|c|ccc|}\hline
x       		& -2		& 0		& 2		\\ \hline
\prob(X=x)	& c/6	& c/2 & c/3 	\\ \hline
\end{array}
\]
\begin{parts}
\part
Find the value of $c$.
\begin{answer}
\[
\frac{c}{6} + \frac{c}{2} + \frac{c}{3} = 1 \quad\Rightarrow\quad c=1.
\]
\end{answer}
\part Show that $\expe(X)=1/3$ and $\var(X)=17/9$.
\begin{answer}
The expected value and variance of $X$ are
\begin{align*}
\expe(X) 	& = \left(-2 \times\frac{1}{6}\right) + \left(0 \times\frac{1}{2}\right) + \left(2 \times\frac{1}{3}\right) = \frac{1}{3}, \\
\expe(X^2) 	& = \left(4 \times\frac{1}{6}\right) + \left(0 \times\frac{1}{2}\right) + \left(4 \times\frac{1}{3}\right) = 2, \\
\var(X)		& = \expe(X^2) - \expe(X)^2 = 2 - \frac{1}{9} = \frac{17}{9}.
\end{align*}
\end{answer}

\part Find $\expe(Y)$ and $\var(Y)$ when $Y=3X+4$.
\begin{answer}
By the properties of expectation and variance,
\begin{align*}
\expe(Y)		& = \expe(3X+4) = 3\expe(X)+4 = 5, \\
\var(Y)		& =  \var(3X+4) = 9\var(X) = 17.
\end{align*}
\end{answer}

\end{parts}

%----------------------------------------
\question
Let $X$ be a random variable with the following PMF, where $c$ is a constant:
\[
\begin{array}{|c|ccc|}\hline
x       		& 1		& 2		& 3		\\ \hline
\prob(X=x)	& c/8	& c/8 & c/4 	\\ \hline
\end{array}
\]
\begin{parts}
\part
Find the value of $c$.
\begin{answer}
\[
\frac{c}{8} + \frac{c}{8} + \frac{c}{4} = 1 \quad\Rightarrow\quad c=2.
\]
\end{answer}
\part
Find $\expe(X)$ and $\var(X)$.
\begin{answer}
\begin{align*}
\expe(X) 	& = \left(1 \times\frac{1}{4}\right) + \left(2 \times\frac{1}{4}\right) + \left(3 \times\frac{1}{2}\right) = \frac{9}{4}, \\
\expe(X^2) 	& = \left(1 \times\frac{1}{4}\right) + \left(4 \times\frac{1}{4}\right) + \left(9 \times\frac{1}{2}\right) = \frac{23}{4}, \\
\var(X)		& = \expe(X^2) - \expe(X)^2 = \frac{23}{4} - \frac{81}{16} = \frac{11}{16}.
\end{align*}
\end{answer}
\part
Let $Y=2X+3$. Show that $\expe(Y)=15/2$ and $\var(Y)=11/4$.
\begin{answer}
\begin{align*}
\expe(Y)		& = \expe(2X+3) = 2\expe(X)+3 = 15/2, \\
\var(Y)		& =  \var(2X+3) = 4\var(X) = 11/4.
\end{align*}
\end{answer}
\end{parts}

%----------------------------------------
\question
Let $X$ be a random variable with the following PMF, where $c$ is a constant:
\[
f(x) = \begin{cases}
	\frac{x}{8}	& \text{for }\ x=1,2, \\
	c 				& \text{for }\ x=3,4, \\
	\frac{1}{8}	& \text{for }\ x=5, \\
	0				& \text{otherwise.}
\end{cases}	
\]
\begin{parts}

\part Show that $c = 1/4$.
\begin{answer}
Because $f(x)$ is a PMF we need that $\sum_{x=1}^5 \prob(X=x) = 1$, so
\[
\frac{1}{8}+\frac{2}{8}+c+c+\frac{1}{8} = 1 \quad \Rightarrow \quad c = \frac{2}{8} = \frac{1}{4}.
\]
\end{answer}

\part Find the mean and variance of $X$.
\begin{answer}
\begin{align*}
\expe(X) 
			& = \sum_{x=1}^5 xf(x) \\
			& = \left(1\times\frac{1}{8}\right) + \left(2\times\frac{2}{8}\right) + \left(3\times\frac{2}{8}\right) + 
					\left(4\times\frac{2}{8}\right) + \left(5\times\frac{1}{8}\right) \\
			& = \frac{1+4+6+8+5}{8} = \frac{24}{8} = 3. \\
\expe(X^2)	
			& = \sum_{x=1}^5 x^2 f(x) \\
			& = \left(1\times\frac{1}{8}\right) + \left(4\times\frac{2}{8}\right) + \left(9\times\frac{2}{8}\right) + 
					\left(16\times\frac{2}{8}\right) + \left(25\times\frac{1}{8}\right) \\
			& = \frac{1+8+18+32+25}{8} = \frac{84}{8} = \frac{21}{2}. \\
\end{align*}
Hence, $\var(X) = \expe(X^2)-\expe(X)^2 = \frac{21}{2} - 9 = \frac{3}{2}$.
\end{answer}

\part Find the mean and variance of $Y=2X-1$.
\begin{answer}
Using the properties of expectation and variance,
\begin{align*}
\expe(Y)		& = \expe(2X-1) = 2\expe(X)-1 = 5, \\
\var(Y) 		& = \var(2X-1) = 4\var(X) = 6.
\end{align*}

Alternatively, 
\begin{align*}
\expe(Y) 
			& = \sum_{x=1}^5 (2x-1)f(x) \\
			& = \left(1\times\frac{1}{8}\right) + \left(3\times\frac{2}{8}\right) + \left(5\times\frac{2}{8}\right) + 
					\left(7\times\frac{2}{8}\right) + \left(9\times\frac{1}{8}\right) \\
			& = \frac{1+6+10+14+9}{8} = \frac{40}{8} = 5. \\
\end{align*}
Similarly,
\begin{align*}
\expe(Y^2)	
			& = \sum_{x=1}^5 (2x-1)^2f(x) \\
			& = \left(1\times\frac{1}{8}\right) + \left(9\times\frac{2}{8}\right) + \left(25\times\frac{2}{8}\right) + 
					\left(49\times\frac{2}{8}\right) + \left(81\times\frac{1}{8}\right) \\
			& = \frac{1+18+50+98+81}{8} = \frac{248}{8} = 31. \\
\end{align*}
Hence, $\var(Y) = \expe(Y^2)-\expe(Y)^2 = 31 - 25 = 6$.
\end{answer}
\end{parts}

%----------------------------------------
% mean & variance
% RND p116q9
\question
Let $X$ be a random variable with mean $\mu$ and variance $\sigma^2$. If $c$ is any real number, show that 
\[
\expe\big[(X-c)^2\big] = \sigma^2 + (\mu-c)^2
\]
and deduce that $\expe\big[(X-c)^2\big]$ is minimised when $c=\mu$.
\begin{answer}
\begin{align*}
\expe\big((X-c)^2\big)
	& = \expe\Big[\big[(X-\mu)+(\mu-c)\big]^2\Big] \\
	& = \expe\big[(X-\mu)^2 + 2(X-\mu)(\mu-c) + (\mu-c)^2\big] \\
	& = \expe\big[(X-\mu)^2\big] + 2(\mu-c)\expe(X-\mu) + (\mu-c)^2 \qquad\text{(by linearity)}\\
	& = \expe\big[(X-\mu)^2\big] + (\mu-c)^2\quad \text{because $\expe(X-\mu) = 0$.}
\end{align*}

Since $(\mu-c)^2\geq 0$, it follows that $\expe\big[(X-c)^2\big] \geq \expe\big[(X-\mu)^2\big]$, with equality if $c=\mu$. 

This shows that $\expe\big[(X-c)^2\big]$ is minimised when $c=\mu$.
\end{answer}


\end{questions}
\end{exercise}
%======================================================================
\endinput
%======================================================================
