% !TEX root = main.tex
%======================================================================
\chapter{The Uniform, Exponential and Normal Distributions}\label{chap:uniform_exponential_normal}
%======================================================================

%----------------------------------------------------------------------
\section{The uniform distribution}
%----------------------------------------------------------------------
\begin{center}
\begin{tabular}{ll}\hline
Notation			& $X\sim\text{Uniform}[a,b]$ \\
Parameter(s)		& $a,b\in\R$, with $a<b$ \\
Range			& $[a,b]\subset\R$ \\
PDF				& $f(x) = \displaystyle\frac{1}{b-a}$ \\[2ex]
CDF				& $F(x) = \displaystyle\frac{x-a}{b-a}$ \\[2ex] \hline
\end{tabular}
\end{center}

% mean and variance
\begin{lemma}
If $X\sim\text{Uniform}[a,b]$, then 
\[
\expe(X)= \frac{a+b}{2} \quad\text{and}\quad \var(X) = \frac{(b-a)^2}{12}.
\]
\end{lemma}

\begin{proof}
\hideoff
See Lecture~\ref{chap:cts}.
\hideon
\end{proof}

% example
\begin{example}
A string of length $L$ is cut at a point chosen uniformly at random along the string. Let $Y$ be the area of the rectangle, two of whose sides are formed by the two pieces of string. Find $\expe(Y)$, and the probability that $Y$ exceeds $5L^2/36$.
\end{example}
\begin{solution}
Let $X$ and $L-X$ denote the lengths of the two pieces. The area of the rectangle is $Y=X(L-X)$. Since $X$ has the uniform distribution on $[0,L]$, it has the following PDF:
\[
f(x) = \begin{cases}
	\displaystyle\frac{1}{L}	& \text{if }\ 0\leq x\leq L, \\[2ex]
	0			& \text{otherwise}.	
\end{cases}
\]
Let $g(x)=x(L-x)$. The expected value of $Y=g(X)$ is
\begin{align*}
\expe(Y)
	= \int_{-\infty}^{\infty} g(x)f(x)\,dx 
	= \int_0^L x(L-x)\frac{1}{L}\,dx 
	= \frac{1}{L}\left[\frac{Lx^2}{2}-\frac{x^3}{3}\right]_0^L 
	= \frac{L^2}{6}.
\end{align*}
The probability that $Y=g(X)$ exceeds $5L^2/36$ is
\begin{align*}
\prob\left(Y>\frac{5L^2}{36}\right)	
	& = \prob\left(LX - X^2 > \frac{5L^2}{36}\right) \\
	& = \prob\left(X^2 - LX + \frac{5L^2}{36} < 0 \right) \\
	& = \prob\left[\left(X-\frac{5L}{6}\right)\left(X-\frac{L}{6}\right) < 0\right]
\end{align*}
\bit
\it The expression $\displaystyle\left(X-\frac{5L}{6}\right)\left(X-\frac{L}{6}\right) < 0$ if and only if exactly one of the factors is negative. 
\it This occurs if and only if $\displaystyle\frac{L}{6} < X < \frac{5L}{6}$.
\eit
Hence,
\[
\prob\left(Y>\frac{5L^2}{36}\right)
	= \prob\left(\frac{L}{6} < X < \frac{5L}{6}\right)
	= \int_{L/6}^{5L/6} \frac{1}{L}\,dx = \frac{2}{3}.
\]	
\end{solution}

%----------------------------------------------------------------------
\section{The negative exponential distribution}
%----------------------------------------------------------------------
\begin{center}
\begin{tabular}{ll}\hline
Notation			& $X\sim\text{Exponential}(\lambda)$ \\
Parameter(s)		& $\lambda>0$ \quad (rate) \\
Range			& $[0,\infty)$ \\
PDF				& $f(x) = \lambda e^{-\lambda x}$ \\
CDF				& $F(x) = 1 - e^{-\lambda x}$ \\ \hline
\end{tabular}
\end{center}

% mean and variance
\begin{lemma}
If $X\sim\text{Exponential}(\lambda)$, then 
\[
\expe(X)= \frac{1}{\lambda} \quad\text{and}\quad \var(X) = \frac{1}{\lambda^2}.
\]
\end{lemma}

\begin{proof}
\[
\expe(X) = \int xf(x)\,dx = \lambda\int_0^{\infty} xe^{-\lambda x}\,dx
\]
Integrating by parts,
\begin{equation*}\label{eq:expe_expo}
\int_0^{\infty} xe^{-\lambda x}\,dx
	= \left[-\frac{xe^{-\lambda x}}{\lambda}\right]_0^{\infty} + \int_0^{\infty} \frac{e^{\lambda x}}{\lambda}\,dx
	= 0 + \frac{1}{\lambda}\int_0^{\infty} e^{-\lambda x}\,dx
	= \frac{1}{\lambda}\left[-\frac{e^{-\lambda x}}{\lambda}\right]_0^{\infty}
	= \frac{1}{\lambda^2}.% \tag{*}
\end{equation*}
Hence, 
\[
\expe(X) = \lambda\int_0^{\infty} xe^{-\lambda x}\,dx = \frac{1}{\lambda}.
\]
To compute the variance, 
\[
\expe(X^2) = \int x^2 f(x)\,dx = \lambda\int_0^{\infty} x^2 e^{-\lambda x}\,dx
\]
Integrating by parts,
\[
\int_0^{\infty} x^2 e^{-\lambda x}\,dx
	= \left[x^2\left(-\frac{e^{-\lambda x}}{\lambda}\right)\right]_0^{\infty} + \int_0^\infty 2x\left(\frac{e^{-\lambda x}}{\lambda}\right)\,dx 
	= \frac{2}{\lambda}\int_0^{\infty} xe^{-\lambda x}\,dx 
	= \frac{2}{\lambda^3}.% \qquad \text{by Eq.~(\ref{eq:expe_expo}).}
\]
Hence 
\[
\expe(X^2) = \lambda\int_0^{\infty} x^2 e^{-\lambda x}\,dx = \frac{2}{\lambda^2},
\]
and the variance is 
\[
\var(X) = \expe(X^2) - \expe(X)^2 = \frac{2}{\lambda^2} - \left(\frac{1}{\lambda}\right)^2 = \frac{1}{\lambda^2}.
\]
\end{proof}
% example
%\begin{example}
%Let $T$ be a random variable representing the time (in hours) before a new type of lightbulb fails. Suppose that $T$ has exponential distribution with parameter $\lambda=0.001$.
%\ben
%\it Find the mean and standard deviation of the lifetime of a ligtbulb.
%\it Calculate the probability that a lightbulb lasts for at least $500$ hours.
%\it Calculate the conditional probability that a lightbulb lasts for at least $500$ hours, given that it has lasted for $300$ hours.
%\een
%\end{example}

\begin{example}
The lifetime of a type of electrical fuse has exponential distribution with a mean of $4$ years. 
\ben
\it Find the probability that a fuse lasts for at least $2$ years.
\it Find the probability that a fuse lasts for at least $6$ years, given that it lasts for at least $4$ years.
\een
\end{example}

\begin{solution}
Let $X$ be the lifetime of a fuse. Since the mean is $4$, the (rate) parameter of the distribution is $\lambda=1/4$, so $X$ has the following CDF: 
\[
F(x) = \begin{cases}
	1 - e^{-x/4} 	& \text{if }\ x\geq 0, \\
	0				& \text{otherwise}.	
\end{cases}
\]
Thus $\prob(X\geq 2) = 1-\prob(X\leq 2) = 1-F(2) = 1 - (1 - e^{-2/4}) = e^{-1/2} = 0.6065$, and 
\[
\prob(X\geq 6\,|\,X\geq 4)
	= \frac{\prob(X\geq 6\text{ and } X\geq 4)}{\prob(X\geq 4)} 
	= \frac{\prob(X\geq 6)}{\prob(X\geq 4)}
	= \frac{e^{-6/4}}{e^{-4/4}} = e^{-1/2} = 0.6065 \quad\text{(again)}.
\]
This illustrates that the exponential distribution has the so-called `memoryless' property.
\end{solution}

The exponential distribution is related to the Poisson distribution: if the number of `arrivals' per unit time has Poisson distribution with mean $\lambda$, the inter-arrival times are independent random variable, each having exponential distribution with rate parameter $\lambda$. 

% example
\begin{example}
The number of customers entering a supermarket per minute has the $\text{Poisson}(10)$ distribution.% with mean 10. 
\ben
\it What is the CDF of the time between the arrival of successive customers?
\it What is the probability that a customer arrives between three and five seconds after the previous customer?
\een
\end{example}

\begin{solution}
\ben
\it % << (i)
Let $\lambda=10$ denote the mean number of customers arriving per minute, and let $T$ denote the time between successive arrivals. Then $T$ has exponential distribution with rate parameter $\lambda$, so its distribution function is
\[
F(t) = \begin{cases}
	1 - e^{-\lambda t}	& \text{if }\ t \geq 0 \\
	0					& \text{otherwise.}
\end{cases}
\]
\it % << (ii)
The probability that a customer arrives between 3 and 5 seconds after the previous customer is
\begin{align*}
\prob(3/60\leq T\leq 5/60) 
	& = F(5/60) - F(3/60) \\
	& = e^{-3\lambda/60} - e^{-5\lambda/60} \\
	& = e^{-1/2} - e^{-5/6} = 0.6065 - 0.4346 = 0.1719.
\end{align*}
\een
\end{solution}

%----------------------------------------------------------------------
\section{The normal distribution}
%----------------------------------------------------------------------
\begin{center}
\begin{tabular}{ll}\hline
Notation			& $X\sim\text{N}(\mu,\sigma^2)$ \\
Parameter(s)		& $\mu\in\R$, $\sigma^2\in\R^{+}$ \\
Range			& $\R$ \\
PDF				& $f(x) = \displaystyle\frac{1}{\sigma\sqrt{2\pi}} e^{-\frac{1}{2}\left(\frac{x-\mu}{\sigma}\right)^2}$ \\[2ex] \hline 
\end{tabular}
\end{center}

The normal distribution is based on the \emph{Gaussian integral}:
\[
\int_{-\infty}^{\infty} e^{-t^2}\,dt = \sqrt{\pi}
\]
For this reason, it is often called the \emph{Gaussian distribution}.

% standard normal distribution
\begin{definition}
The \emph{standard normal distribution} is the normal distribution with parameters $\mu=0$ and $\sigma^2=1$.
\end{definition}

Let $Z\sim\text{N}(0,1)$ be a standard normal variable.
\ben
\it The distribution function of $Z$ is usually denoted by $\Phi(z) = \prob(Z\leq z)$.
\it The density function of $Z$ is usually denoted by $\phi(z)$:
\[
\phi(z) = \frac{1}{\sqrt{2\pi}} e^{-z^2/2}
\]
\een

% mean and variance
\begin{lemma}
If $X\sim\text{Normal}(\mu,\sigma^2)$, then $\expe(X)=\mu$ and $\var(X)=\sigma^2$.
\end{lemma}

\begin{proof}
First consider $Z\sim\text{N}(0,1)$.
\begin{align*}
\expe(Z) = \int_{-\infty}^{\infty} z \phi(z) \,dz 
	& = \frac{1}{\sqrt{2\pi}}\int_{-\infty}^{\infty} z e^{-\frac{1}{2}z^2} \,dz \\
	& = \frac{1}{\sqrt{2\pi}}\int_{-\infty}^{\infty} \frac{d}{dz}(-e^{-\frac{1}{2}z^2}) \,dz \\
	& = \frac{1}{\sqrt{2\pi}}\left[ -e^{-\frac{1}{2}z^2} \right]_{-\infty}^{\infty} \\
	& = 0. 
\end{align*}
and
\begin{align*}
\expe(Z^2) 
	= \int_{-\infty}^{\infty} z^2 \phi(z)\,dz 
	& = \frac{1}{\sqrt{2\pi}}\int_{-\infty}^{\infty} z^2 e^{-z^2/2} \,dz \\
	& = \frac{1}{\sqrt{2\pi}}\int_{-\infty}^{\infty} z\frac{d}{dz}(-e^{-z^2/2}) \,dz \\
	& = \frac{1}{\sqrt{2\pi}}\left[ -ze^{-z^2/2} \right]_{-\infty}^{\infty} + \frac{1}{\sqrt{2\pi}}\int_{-\infty}^{\infty} e^{-z^2/2}\,dz \qquad\text{(integration by parts)}\\
	& = 0 + \frac{1}{\sqrt{\pi}}\int_{-\infty}^{\infty} e^{-t^2}\,dt \qquad\text{(using a change of variable: $t=z\sqrt{2}$)} \\
	& = 1,
\end{align*}	
so $\var(Z) = \expe(Z^2) - \expe(Z)^2 = 1$.

\vspace*{2ex}
For the general case, let $X=\mu+\sigma Z$ where $Z\sim N(0,1)$. The CDF of $X$ is 
\[
F(x) 
	= \prob(X\leq x) 
	= \prob\left(\mu + \sigma Z \leq x\right) 
	= \prob\left(Z\leq \frac{x-\mu}{\sigma}\right) 
	= \Phi\left(\frac{x-\mu}{\sigma}\right)
\]
and (using the chain rule), the density function of $X$ is
\begin{align*}
f(x) 
	& = \frac{d}{dx}F(x) \\
	& = \frac{d}{dx}\Phi\left(\frac{x-\mu}{\sigma}\right) \\
	& = \frac{1}{\sigma}\phi\left(\frac{x-\mu}{\sigma}\right) \\
	& = \frac{1}{\sigma\sqrt{2\pi}}\exp\left[{-\displaystyle\frac{1}{2}\left(\frac{x-\mu}{\sigma}\right)^2}\right]
\end{align*}
so $X\sim\text{N}(\mu,\sigma^2)$. By the linearity of expectation, 
\begin{align*}
\expe(X)	& = \expe(\mu+\sigma Z) = \mu + \sigma\expe(Z) = \mu, \\
\intertext{and by the properties of variance}
\var(X)	& = \var(\mu+\sigma Z) = \sigma^2\var(Z) = \sigma^2.
\end{align*}
\end{proof}
%Hence the normal distribution is parameterized by its mean $\mu$ and variance $\sigma^2$.

% example
\begin{example}
The height $X$ in metres of a randomly chosen adult male has a distribution which is approximately normal with mean $\mu$ and variance $\sigma^2$. If $10\%$ of males in the population are taller than $1.8$ metres, and $5\%$ are shorter than $1.6$ metres, find $\mu$ and $\sigma$. 

\vspace{2ex}
For $Z\sim\text{N}(0,1)$, you may assume that $\prob(Z\leq -1.645)	\approx 0.05$ and $\prob(Z\leq 1.280) \approx 0.90$.
%\begin{align*}
%\prob(Z\leq -1.645)	& \approx 0.05 \text{\quad and} \\
%\prob(Z\leq 1.280)	& \approx 0.90.
%\end{align*}
\bit
\it The values $z_{0.05} = -1.645$ and $z_{0.90} = 1.280$ are called the \emph{critical points} of the standard normal distribution at levels $0.05$ and $0.90$ respectively. 
\it Critical values can be looked up in statistical tables, or computed using statistical software packages.
\eit

%For $Z\sim\text{N}(0,1)$, you may assume that $\prob(Z\leq -1.645)\approx 0.05$ and $\prob(Z\leq 1.280)\approx 0.90$. 
%
%[The values $z_{0.05} = -1.645$ and $z_{0.90} = 1.280$ are the \emph{critical points} of the standard normal distribution at levels $0.05$ and $0.90$ respectively. These critical points would usually be looked up in statistical tables, or computed using a statistical software package.)]
\end{example}

\begin{solution}
\bit
\it Let $X\sim\text{N}(\mu,\sigma^2)$. 
\eit
From the question, we know that $\prob(X\leq 1.6) = 0.05$ and $\prob(X\leq 1.8)=0.9$.
\bit
\it Let $Z=\displaystyle\frac{X-\mu}{\sigma}$. 
\eit
%\bit
%\it 
Then $Z\sim\text{N}(0,1)$, with
$\prob\left(Z\leq \displaystyle\frac{1.6-\mu}{\sigma}\right) = 0.05$ and
$\prob\left(Z\leq \displaystyle\frac{1.8-\mu}{\sigma}\right) = 0.90$.
%\eit

\vspace{2ex}
From the critical points given in the question,
\[
\displaystyle\frac{1.6-\mu}{\sigma} = -1.645 \text{\quad and\quad} \displaystyle\frac{1.8-\mu}{\sigma} = +1.280.
\]
%Thus we obtain $\mu - 1.645\sigma = 1.6$ and $\mu + 1.280\sigma = 1.8$. 
Solving these equations, we obtain
\[
\mu=1.712 \text{\quad and\quad} \sigma=0.0684.
\]
%Solving these equations, we obtain $\mu=1.712$ and $\sigma=0.0684$.

The height of adult males therefore has mean $1.712$m and standard deviation $0.068$m.
\end{solution}


%----------------------------------------------------------------------
\newpage
\section{The chi-squared distribution$^{*}$}
%----------------------------------------------------------------------
During an experiment, we make a sequence of independent observations $X_1,X_2,\ldots,X_n$. An expert tells us that our observations should be normally distributed with mean $\mu$ and variance $\sigma^2$. How can we test to see whether our data fits this model?

Under the proposed model, the standardised variables $\Z_i = (X_i-\mu)/\sigma$ should have the standard normal distribution $\text{N}(0,1)$. To quantify the extent to which our data fits the model, we compute the total squared deviation between our standardised observations $Z_1,Z_2,\ldots,Z_n$ and their hypothetical mean, which is zero. This is called the \emph{chi-squared} statistic:
\[
\chi^2_n = \sum_{i=1}^n Z_i^2 = \sum_{i=1}^n \left(\frac{X_i-\mu}{\sigma}\right)^2
\]
If the expert is correct, $\chi^2_n$ will have the so-called \emph{chi-squared} distribution with $n$ degrees of freedom. Consequently, if the observed value of $\chi^2_n$ is far from the centre of this distribution (i.e.\ the observed value is in the upper or lower tail of the distribution), we might conclude that the observations are \emph{not} normally distributed.

Let $Z_1,Z_2,\ldots,Z_k$ be independent standard normal random variables, and let
\[
\displaystyle X=\sum_{i=1}^k Z_i^2.
\]
Then $X$ is said to have \emph{chi-squared distribution} with $k$ degrees of freedom.

\begin{center}
\begin{tabular}{ll}\hline
Notation			& $X\sim\text{Chi-squared}(k)$ \\
Parameter(s)		& $k \in \N$ \quad (degrees of freedom) \\
Range			& $\R^{+}$ \\
PDF				& $f(x) = \displaystyle\frac{x^{k/2-1} e^{-x/2}}{2^{k/2}\Gamma(k/2)} $ \\[2ex] \hline
\end{tabular}
\end{center}


% mean and variance
\begin{lemma}
If $X\sim\text{Chi-squared}(k)$, then $\expe(X)= k$ and $\var(X) = 2k$.
\end{lemma}

\begin{proof}
\hideoff
If $Z_i\sim\text{N}(0,1)$, then 
\[
\expe(Z_i^2) = \var(Z_i)+\expe(Z_i)^2 = 1.
\]
Thus by the linearity of expectation,
\[
\expe(X) = \expe(Z^2_1)+\expe(Z^2_2)+\ldots+\expe(Z^2_k) = k
\]
It is easy to show that $\expe(Z_i^4)=3$, so 
\[
\var(Z_i^2) = \expe(Z_i^4)-\expe(Z_i^2)^2 = 2.
\]
Hence, because the $Z_i$ are independent,
\[
\var(X) = \var(Z^2_1)+\var(Z^2_2)+\ldots+\var(Z^2_k) = 2k,
\]
as required.
\hideon
\end{proof}


%----------------------------------------------------------------------
\newpage
\section{Exercises}
% !TEX root = main.tex
% ex18_uniform_exponential_normal.tex
\begin{exercise}
\begin{questions}
%----------------------------------------
%--------------------
% UNIFORM
% RND 4.2.2
\question
The first bus of the day arrives at a certain stop at 7 a.m., and then at 10 minute intervals until the evening. A girl arrives at the stop at a time which is uniformly distributed between 7.15 a.m. and 7.45 a.m. Find the probability that she waits for a bus for
\ben
\it less than two minutes,
\it more than four minutes.
\een

\begin{answer}
\ben
\it % << (i)
She waits less than two minutes if she arrives during any of the time intervals $(7.18,7.20]$, $(7.28,7.30]$ and $(7.38,7.40]$. The total length of these intervals is $6$ minutes, and there are $30$ minutes between 7.15 a.m. and 7.45 a.m. Since her time of arrival at the bus stop is uniformly distributed between 7.15 a.m. and 7.45 a.m., the probability that she waits for less than two minutes is $6/30 = 0.2$.
\it % << (ii)
She waits more than four minutes if she arrives during any of the intervals $(7.15,7.16]$, $(7.20,7.26]$ and $(7.30,7.36]$. The total length of these intervals is $18$ minutes, so the probability that she waits for more than four minutes is $18/30 = 0.6$.
\een
\end{answer}

%--------------------
% ORDER STATISTICS
% GS 4.2.3
\question
Let $X_1$, $X_2$, $X_3$ and $X_4$ be independent and identically distributed continuous random variables. 
\ben
\it Show that $\prob(X_1 < X_2 < X_3 < X_4) = \frac{1}{24}$.
\it Find $\prob(X_1 > X_2 < X_3 < X_4)$.
\een

\begin{answer}
Since $X_1$, $X_2$, $X_3$ and $X_4$ are independent and identically distributed, every ordering is equally likely. 
\ben
\it There are $4!=24$ possible orderings, so the probability that $X_1 < X_2 < X_3 < X_4$ is equal to $\frac{1}{24}$.
\it There are three possible orderings corresponding to $X_1 > X_2 < X_3 < X_4$:
\[
X_2 < X_1 < X_3 < X_4,\quad X_2 < X_3 < X_1 < X_4 \quad\text{and}\quad X_2 < X_3 < X_4 < X_1.
\]
Each of these has probability $\frac{1}{24}$, so $\prob(X_1 > X_2 < X_3 < X_4) = \frac{3}{24}$.
\een

\end{answer}


%--------------------
% ARRIVALS
% RND 4.2.11
\question
The time $T$ between successive arrivals at a hospital has PDF
\[
f(t) = \begin{cases}
	\lambda e^{-\lambda t} 	& \text{if }\ t\geq 0,\\
	0		  				& \text{otherwise.}		
\end{cases}
\]
Exactly 100 inter-arrival times were measured, and the total was found to be 500 hours.
\ben
\it Estimate the value of $\lambda$.
\it Use this estimate of $\lambda$ to estimate the probabilities $\prob(T\leq 5)$ and $\prob(T\leq 10\,|\, T > 5)$.
\een
\begin{answer}
\ben
\it % << (i)
An estimate of the average inter-arrival time $\expe(T)$ is $\displaystyle\frac{500}{100} = 5$. Since $\expe(T)=\displaystyle\frac{1}{\lambda}$, we take $\displaystyle\frac{1}{5}$ as an estimate of $\lambda$. (This is the so-called \emph{method-of-moments} estimator.)
\it % << (ii)
\begin{align*}
\prob(0<T<5)
	& \approx \int_0^5\frac{1}{5}e^{-t/5}\,dt 
	= \big[-e^{-t/5}\big]_0^5 
	= 1 - e^{-1} 
	= 0.6321. \\
\prob(T<10\,|\,T>5)
	& \approx \frac{\prob(5<T<10)}{\prob(T>5)} 
	= \frac{\big[-e^{-t/5}\big]_5^{10}}{\big[-e^{-t/5}\big]_0^5}
	= 1 - e^{-1} 
	= 0.6321. 
\end{align*}
\een
\end{answer}

%--------------------
% SCALING
% RND 4.2.15
\question
A teacher set and marked an examination, and found that the distribution of marks were (approximately) normally distributed with mean $42$ and standard deviation $14$. The school's policy is to present scaled marks whose distribution is (approximately) normal with mean $50$ and standard deviation $15$. Find the (linear) transformation that the teacher should apply to the raw marks to accomplish this. What is the transformed mark corresponding to a raw mark of $40$?
\begin{answer}
If $X\sim\text{N}(\mu,\sigma^2)$ and $Y=aX+b$ where $a,b\in\R$ are constants, then $Y\sim\text{N}(a\mu+b,a^2\sigma^2)$. To see this,
\[
F(y) = \prob(Y\leq y) 
	 = \prob\left(X\leq\frac{y-b}{a}\right) 
	  = \frac{1}{\sigma\sqrt{2\pi}} \int_{-\infty}^{\frac{y-b}{a}} \exp\left(-\frac{1}{2}\left(\frac{x-\mu}{\sigma}\right)^2\right)\,dx.
\]
Changing the variable of integration from $x$ to $t = ax + b$, this becomes
\[
F(y)  = \frac{1}{|a|\sigma\sqrt{2\pi}} \int_{-\infty}^{y} \exp\left(-\frac{1}{2}\left(\frac{t-(a\mu+b)}{a\sigma}\right)^2\right)\,dt 
\]
which is the distribtion function of a $\text{N}(a\mu+b,a^2\sigma^2)$ random variable.

In this example, $X\sim\text{N}(42,14^2)$ and $Y\sim\text{N}(42a+b,14^2a^2)$, so we need to find $a$ and $b$ such that
\[
42a + b = 50 \quad\text{and}\quad 14^2a^2 = 15^2.
\]
Solving these equations, we obtain the solutions
\[
\left(a = \frac{15}{14},\ b=5\right) \quad\text{and}\quad \left(a = -\frac{15}{14},\ b=95\right). 
\]
The transformation $Y=-\displaystyle\frac{15}{14}X + 95$ transforms the low marks to high marks (and vice versa); the required transformation is 
\[
Y=\displaystyle\frac{15}{14}X + 5.
\]
If the raw mark is $X=40$, the transformed mark is $Y = 48$ (rounded to the nearest integer).
\end{answer}

%--------------------
% MAX AND MIN
% GS 4.2.2
\question
Let $X_1$ and $X_2$ be independent and identically distributed continuous random variables, let $F(x)$ denote their common CDF, and let $f(x)$ denote their common PDF. 
\ben
\it Show that the CDF and PDF of $V=\max\{X_1,X_2\}$ are $F_V(v)=F(v)^2$ and $f_V(v) = 2F(v)f(v)$ respectively.
\it Find the CDF and PDF of $U=\min\{X_1,X_2\}$.
\een
\begin{answer}
\ben
\it % << (i)
$V=\max\{X_1,X_2\}\leq v$ if and only if $X_1\leq v$ and $X_2\leq v$, so by independence,
\begin{align*}
F_V(v) 
	& = \prob(V\leq v) = \prob\big(\max\{X_1,X_2\}\leq v\big) \\
	& = \prob(X_1\leq v,X_2\leq v) \\
	& = \prob(X_1\leq v)\prob(X_2\leq v) \\
	& = F(v)^2.
\end{align*}
Taking the derivative with respect to $v$ (using the chain rule),
\[
f_V(v) = \frac{d}{dv}F_V(v) = \frac{d}{dv}\big(F(v)^2\big) =2F(v)f(v).
\]
\it % << (ii)
$U=\min\{X_1,X_2\}>u$ if and only if $X_1> v$ and $X_2> v$, so by independence,
\begin{align*}
F_U(u) 
	& = 1- \prob(U > u) = 1-\prob\big(\min\{X_1,X_2\} > u\big) \\
	& = 1-\prob(X_1 > u, X_2 > u) \\
	& = 1-\prob(X_1 > u)\prob(X_2 > u) \\
	& = 1-\big(1-F(u)\big)^2.
\end{align*}
%so $F_U(u) = \prob(U\leq u) = 1 - \prob(U > u) = 1 - \big(1-F(u)\big)^2$. 

Taking the derivative with respect to $u$,
\[
f_U(u) = \frac{d}{du}F_U(u) = \frac{d}{du}\Big(1-\big(1-F(u)\big)^2\Big) = 2\big(1-F(u)\big)f(u).
\]
\een
\end{answer}

%--------------------
% EXPECTATION IN TERMS OF COMPLEMENTARY CDF
\question
Let $X:\Omega\to\R $ be a continuous random variable with CDF $F(x)$, and suppose that $X$ is \emph{non-negative} in the sense that $X(\omega)\geq 0$ for all $\omega\in\Omega$. Show that 
\[
\expe(X) = \int_{0}^{\infty}\big(1-F(x)\big)\,dx
\]
\begin{answer}
\begin{align*}
\int_0^\infty 1 - F(x) \,dx 
	& = \int_{0}^{\infty} \prob(X > x)\,dx \\
	& = \int_{0}^{\infty}\left(\int_{x}^{\infty} f(t)\,dt\right)\,dx \\
	& = \int_{0}^{\infty}\left(\int_{0}^{t} f(t)\,dx\right)\,dt \\
	& = \int_{0}^{\infty} t f(t)\,dt \\
	& = \expe(X)
\end{align*}
\end{answer}

%--------------------
% MEMORYLESS (EXP)
\question 
Let $X$ have exponential distribution with (rate) parameter $\lambda$. The CDF of $X$ is as follows:
\[
F(x) = \begin{cases}
	1 - e^{-\lambda x}	& x \geq 0, \\
	0					& \text{otherwise.}
\end{cases}
\]
Show that the exponential distribution has the so-called \emph{memoryless} property: for any $t>0$,
\[
\prob(X\leq x + t\,|\,X > t) = \prob(X \leq x)\quad\text{for all}\quad x\geq 0.
\]
\begin{answer}
For all $x>0$,
\begin{align*}
\prob(X > x + t\,|\, X > t)
	& =  \frac{\prob(X > x + t\text{ and } X > t)}{\prob(X > t)} \\
	& =  \frac{\prob(X > x + t)}{\prob(X > t)} \\
	& =  \frac{e^{-\lambda(x + t)}}{e^{-\lambda t}} =  e^{-\lambda x} = \prob(X > x).
\end{align*}
\end{answer}


%%==========================================================================
%% INFINITE EXPECTATION
%\question 
%Consider the following game. First a number $X$ is chosen uniformly at random from $[0,1]$, then a sequence $Y_1,Y_2,\ldots$ of numbers are chosen independently and uniformly at random from $[0,1]$. Let $Y_n$ be the first number in the sequence for which $Y_n > X$. When this occurs, the game stops and the player is paid $(n-1)$ pounds. Show that the expected win is infinite.
%\begin{answer}
%Let $Z$ represent the amount won. Suppose first that $X$ takes the value $x\in[0,1]$.
%\begin{align*}
%\prob(Z=k|X=x)
%	& = \prob(Y_1\leq x, Y_2\leq x, \ldots, Y_k\leq x, Y_{k+1}>x) \\
%	& = \prob(Y_1\leq x)\prob(Y_2\leq x)\ldots \prob(Y_k\leq x)\prob(Y_{k+1}>x) \qquad\text{(by independence)}\\
%	& = x^k(1-x)
%\end{align*}
%The PDF of $X$ is $f(x)=1$ for $x\in[0,1]$ and zero otherwise. Hence, by the law of total probability,
%\begin{align*}
%\prob(Z=k)
%	& = \int_0^1 \prob(Z=k|X=x) f(x)\,dx \\
%	& = \int_0^1 x^k(1-x)\,dx \\
%	& = \left[\frac{1}{k+1}x^{k+1}-\frac{1}{k+2}x^{k+2}\right]_0^1 \\
%	& = \frac{1}{k+1} - \frac{1}{k+2} \\
%	& = \frac{1}{(k+1)(k+2)} \\
%\end{align*}
%For all $k\geq 1$ we have $k+1\leq 2k$ and $k+2\leq 3k$ so
%\[
%\frac{1}{(k+1)(k+2)} \geq \frac{1}{6k^2}
%\]
%Thus $\expe(Z)$ satisfies
%\[
%\expe(Z) = \sum_{k=0}^{\infty}k\left(\frac{1}{(k+1)(k+2)}\right) \geq \frac{1}{6}\sum_{k=1}^{\infty}\frac{1}{k} = \infty,
%\]
%so the expected win is infinite.
%\end{answer} 


\end{questions}
\end{exercise}

%======================================================================
\endinput
%======================================================================

%----------------------------------------------------------------------

%======================================================================
\endinput
%======================================================================
