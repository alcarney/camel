% !TEX root = main.tex
%======================================================================
\chapter{Normal Approximation}\label{chap:normal_approximation}
%======================================================================

%----------------------------------------------------------------------
\section{Approximation of discrete distributions by continuous distributions}
%----------------------------------------------------------------------
Recall the PMFs of the binomial and Poisson distributions:
\bit
\it If $X\sim\text{Binomial}(n,p)$ then 
\[
\prob(X=k)=\displaystyle\frac{n!}{(n-k)!k!} p^k(1-p)^{n-k} \text{\quad for $k=0,1,2,\ldots,n$}.
\]
%\bit
%\it $\expe(X)=np$ and $\var(X)=np(1-p)$.
%\eit
\it If $X\sim\text{Poisson}(\lambda)$ then 
\[
\prob(X=k)=\displaystyle\frac{\lambda^k}{k!}e^{-\lambda} \text{\quad for $k=0,1,2,\ldots$}.
\]
%\bit
%\it $\expe(X)=\lambda$ and $\var(X)=\lambda$.
%\eit
\eit

If $k$ is large, computing $k!$ can take a long time. To avoid this, we make use of the fact that, under certain conditions, the binomial and Poisson distributions can both be approximated by the normal distribution.

%Recall the following:
\underline{Binomial distribution}:
\bit
\it If $X\sim\text{Binomial}(n,p)$ then $\expe(X)=np$ and $\var(X)=np(1-p)$.
\it If $np$ and $n(1-p)$ are both sufficiently large,
\[
X\sim\text{N}\big(np,np(1-p)\big)\quad\text{approx.}
\]
\eit

\vspace*{2ex}
\underline{Poisson distribution}:
\bit
\it If $X\sim\text{Poisson}(\lambda)$ then $\expe(X)=\lambda$ and $\var(X)=\lambda$.
\it If $\lambda$ is sufficiently large,
\[
X\sim\text{N}(\lambda,\lambda)\quad\text{approx.}
\]
\eit

%----------------------------------------------------------------------
\subsection{The continuity correction}
%----------------------------------------------------------------------

% continuity correction
\begin{definition}[The continuity correction]
Let $X$ be a discrete random variable, taking values in the set $\{0,\pm 1,\pm 2,\ldots\}$. If the distribution of a continuous random variable $Y$ is taken as an approximation to the distribution of $X$, we set
\[
\prob(X=k) = \prob\left(k - \frac{1}{2} < Y < k + \frac{1}{2}\right).
\] 
\end{definition}

In particular,
\bit
\it $\prob(X < k) = \prob(Y\leq k-1/2)$ and $\prob(X\leq k) = \prob(Y\leq k+1/2)$,
\it $\prob(X > k) = \prob(Y\geq k+1/2)$ and $\prob(X\geq k) = \prob(Y\geq k-1/2)$,
\it $\prob(X = k) = \prob(k-1/2 < Y \leq k+1/2)$.
\eit

%----------------------------------------------------------------------
\section{Normal approximation of the binomial distribution}
%----------------------------------------------------------------------

% theorem: normal approx to binomial
\begin{theorem}
If $X\sim\text{Binomial}(n,p)$, then
\[
\prob(X=k)\to \frac{1}{\sqrt{2\pi np(1-p)}}\int_{k-1/2}^{k+1/2} \exp\left[-\frac{1}{2}\left(\frac{x-np}{\sqrt{np(1-p)}}\right)^2\right]\,dx 
\quad\text{as}\quad n\to\infty.
\]
\end{theorem}
\proofomitted

% example
\begin{example}
A fair coin is tossed $500$ times. What is the probability that $270$ heads are obtained?
\end{example}

\begin{solution}
Let $X$ be the number of heads obtained. Then $X\sim\text{Binomial}(500,270)$, so 
\[
\prob(X > 270) = \sum_{k=271}^{n}\binom{500}{k}\left(\frac{1}{2}\right)^{500}
\]
This is difficult to compute (the exact value is $0.0333$).

Using the normal approximation, the associated normal variable is $Y\sim\text{N}(250,125)$, so
\begin{align*}
\prob(X\geq 271)
	& \approx\prob(Y > 270.5) \\
	& = 1 - \prob(Y \leq 270.5) \\
	& = 1 - \prob\left(Z \leq \frac{270.5 - 250}{\sqrt{125}}\right) \qquad\text{where}\quad Z\sim\text{N}(0,1), \\
	& = 1 - \Phi(1.8336) \\
	& = 0.0334 \qquad\text{(from tables)}.
\end{align*}
\end{solution}


%----------------------------------------------------------------------
\section{Normal approximation of the Poisson distribution}
%----------------------------------------------------------------------

% theorem: normal approx to Poisson
\begin{theorem}
If $X\sim\text{Poisson}(\lambda)$, then
\[
\prob(X=k)\to \frac{1}{\sqrt{2\pi\lambda}}\int_{k-1/2}^{k+1/2} \exp\left[-\frac{1}{2}\left(\frac{x-\lambda}{\sqrt{\lambda}}\right)^2\right]\,dx 
\quad\text{as}\quad \lambda\to\infty.
\]
\end{theorem}
\proofomitted

% example
\begin{example}
A newsagent knows from past experience that the weekly demand for a certain magazine has Poisson distribution with mean $20$. How many copies of the magazine should the newsagent stock in order to satisfy the weekly demand with probability $0.95$?
\end{example}


\begin{solution}
Suppose the newsagent stocks $N$ copies of the magazine. Let $X\sim\text{Poisson}(20)$ represent the weekly demand. We need that 
\[
\prob(\text{demand satisfied}) = \prob(X\leq N) = \sum_{k=0}^{N}\frac{20^k}{k!}e^{-20} = 0.95.
\]
Using the normal approximation, the associated normal variable is $Y\sim\text{N}(20,20)$, so 
\begin{align*}
\prob(X\leq N) = 0.95
	& \quad\Rightarrow\quad \prob(Y \leq N + 0.5) = 0.95 \\
	& \quad\Rightarrow\quad \prob\left(Z \leq \frac{N + 0.5 - 20}{\sqrt{20}}\right) = 0.95 \qquad\text{where}\quad Z\sim\text{N}(0,1), \\
	& \quad\Rightarrow\quad \Phi\left(\frac{N -19.5}{\sqrt{20}}\right) = 0.95 \\
	& \quad\Rightarrow\quad \frac{N -19.5}{\sqrt{20}} = 1.645 \qquad\text{(from tables: $z_{0.95} = 1.645$)},\\
	& \quad\Rightarrow\quad N = 27 \qquad\text{(to the nearest integer)}.
\end{align*}
Thus the newsagent should stock 27 copies of the magazine.
\end{solution}



%----------------------------------------------------------------------
\section{Exercises}
% !TEX root = main.tex
% ex20_limit_theorems.tex
\begin{exercise}
\begin{questions}
%----------------------------------------
%--------------------
% NORMAL APPROX TO BINOMIAL
% RND 4.3.3
\question
$90\%$ of all items produced by a manufacturing process are satisfactory. Find an approximation for the probability that a sample of $250$ items contains exactly $25$ defective items. %State any assumptions you make.

%[\slshape Hint: Use the normal approximation to the binomial distribution.\normalfont]

\begin{answer}
Let $X$ be the number of defective items in the sample. Assume that the probability that any particular item is defective is equal to $0.1$, independently of the quality of the other items inspected. Then $X\sim\text{Binomial}(n,p)$ with $n=250$ and $p=0.1$.

Since $np=25$ and $np(1-p)=22.5$, we define the normal approximation to $X$ by the random variable $Y\sim\text{N}(25,22,5)$. Then
\begin{align*}
\prob(X=25)
	& \approx \prob(24.5\leq Y\leq 25.5) \\
	& = \prob(Y\leq 25.5) - \prob(Y\leq 24.5) \\
	& = \prob\left(Z\leq \frac{25.5-25}{\sqrt{22.5}}\right) - \prob\left(Z\leq \frac{254.5-25}{\sqrt{22.5}}\right) \qquad\text{where $Z\sim\text{N}(0,1)$,}\\
	& = \Phi\left(\frac{25.5-25}{\sqrt{22.5}}\right) - \Phi\left(\frac{254.5-25}{\sqrt{22.5}}\right) \\
	& = \Phi(0.1054) - \Phi(-0.1054) \\
	& = 2\Phi(0.1054) - 1		\qquad\text{(by symmetry)} \\
	& = 0.08395 \qquad\text{(from tables).}
\end{align*}
In fact, the exact value is $0.08382$ so the approximation is reasonably good. (As a rule of thumb, the approximation is acceptable provided $np>10$ and $n(1-p)>10$.) 
\end{answer}


%--------------------
% RND 4.3.10
\question
A casino buys a new die and rolls it $600$ times. Let $N$ denote the number of times a six occurs.
\ben
\it Find the probability that $N$ is between $90$ and $100$, assuming that the die is fair.
\it Find the value $c$ for which $\prob(100-c \leq N\leq 100+c) = 0.95$, assuming that the die is fair.
\it What might the casino conclude if a six occurred $N=120$ times?
\een 

\begin{answer}
Let $X$ be the number of times a six occurs. Then $X\sim\text{Binomial}(n,p)$ with $n=600$ and $p=1/6$ if the dice is fair, in which case the mean of $X$ is $np=100$, and its variance is $np(1-p)=250/3$. Let $Y\sim\text{N}(100, 250/3)$ be the normal approximation to $X$.
\ben
\it % << (i)
\begin{align*}
\prob(90\leq X\leq 100)
	& \approx \prob(89.5\leq Y\leq 100.5) \\
	& = \prob(Y\leq 100.5) - \prob(Y\leq 89.5) \\
	& = \prob\left(Z\leq \frac{100.5-100}{\sqrt{250/3}}\right) - \prob\left(Z\leq \frac{89.5-100}{\sqrt{250/3}}\right) \qquad\text{where $Z\sim\text{N}(0,1)$,}\\
	& = \Phi\left(\frac{100.5-100}{\sqrt{250/3}}\right) - \Phi\left(\frac{89.5-100}{\sqrt{250/3}}\right) \\
	& = \Phi(0.0548) - \Phi(-1.1502) \\
	& = 0.3968 \qquad\text{(from tables).}
\end{align*}
\it % << (ii)
We wish to find a positive integer $N$ such that 
\begin{align*}
& \prob(99.5-N\leq Y\leq 100.5+N) = 0.95, 
\intertext{i.e.\ such that}
& \Phi\left(\frac{N+0.5}{\sqrt{250/3}}\right) - \Phi\left(\frac{N+0.5}{\sqrt{250/3}}\right) = 0.95, \\
\intertext{or (by symmetry)}
& \Phi\left(\frac{N+0.5}{\sqrt{250/3}}\right) = 0.975. \\
\intertext{Using tables,}
& \frac{N+0.5}{\sqrt{250/3}} = 1.96, \quad\text{and hence}\quad N = 0.5 + 1.96\sqrt{250/3}.
\end{align*}
This has no integer solution, but the integer that most nearly satisfies it is $N=17$.\par
In fact, $\prob(83\leq X\leq 117)=0.945$.
\it % << (iii)
He might concluded that there is some evidence that the die is unfair, because if the die is fair, $120$ is an unexpectedly large number of sixes.
\een
\end{answer}

\end{questions}
\end{exercise}

%======================================================================
\endinput
%======================================================================

%----------------------------------------------------------------------

%======================================================================
\endinput
%======================================================================
