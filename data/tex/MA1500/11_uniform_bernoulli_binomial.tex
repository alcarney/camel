% !TEX root = main.tex
%======================================================================
\chapter{The Uniform, Bernoulli and Binomial Distributions}\label{chap:unif-bern-bino}
%======================================================================

In this lecture we look at some well-known distributions which take values in a finite subset of the non-negative integers.
%----------------------------------------------------------------------
\section{Uniform distribution}
%----------------------------------------------------------------------

The uniform distribution on $\{1,2,\ldots,n\}$ assigns an equal probability to each value, and as such underpins the classical model of probability (Lecture~\ref{chap:classical}).
 
\begin{center}
\begin{tabular}{ll}\hline
Notation		& $X\sim\text{Uniform}\{1,2,\ldots,n\}$ \\
Parameter(s)	& $n\in\mathbb{N}$ \\
Range			& $\{1,2,\ldots,n\}$ \\
PMF				& $f(k) = 1/n$ for all $k=1,2,\ldots,n$ \\ \hline
\end{tabular}
\end{center}

\begin{lemma}
If $X\sim\text{Uniform}\{1,2,\ldots,n\}$, then 
\[
\expe(X)= \frac{n+1}{2} \quad\text{and}\quad \var(X) = \frac{n^2-1}{12}.
\]
\end{lemma}

\begin{proof}
\begin{align*}
\expe(X) 
	& = \sum_{k=1}^n k\,f(k)
	= \frac{1}{n}\sum_{k=1}^n k 
	= \frac{1}{n}\left(\frac{n(n+1)}{2}\right)
	= \frac{n+1}{2}. \\
\expe(X^2) 
	& = \sum_{k=1}^n k^2\,f(k)
	= \frac{1}{n}\sum_{k=1}^n k^2 
	= \frac{1}{n}\left(\frac{n(n+1)(2n+1)}{6}\right)
	= \frac{(n+1)(2n+1)}{6}. \\
\var(X)
	& = \expe(X^2)-\expe(X)^2  
	= \frac{(n+1)(2n+1)}{6} - \frac{(n+1)^2}{4} 
	  = \frac{(n-1)(n+1)}{12}
	  = \frac{n^2-1}{12}.
\end{align*}
\end{proof}

\newpage
\begin{example}[Entropy]
Entropy is a way of quantifying the uncertainty associated with a random variable.

\smallskip
Let $X$ be a random variable taking values in $\{1,2,\ldots,n\}$, and let $f(k)=\prob(X=k)$ be its PMF. The \emph{entropy} of $X$ is defined by
\[
H(X) = -\sum_{k=1}^n f(k)\log f(k)
\]

If $X$ is a non-random, say $f(j)=1$ and $f(k)=0$ for all $k\neq j$, then
\[
H(X) = -\sum_{k=1}^n f(k)\log f(k) = - f(j)\log f(j) = -\log 1 = 0.
\]

If $X$ is uniformly distributed, so that \ $f(k)=1/n$ for all $k\in\{1,2,\ldots,n\}$, then
\[
H(X) = -\sum_{k=1}^n \frac{1}{n}\log\left(\frac{1}{n}\right) = -\log\left(\frac{1}{n}\right) = \log n.
\]

In general, for any random variable $X$ taking values in $\{1,2,\ldots,n\}$, it can be shown that 
\[
0\leq H(X)\leq \log n.
\]
Among all probability distributions on $\{1,2,\ldots,n\}$, the uniform distribution has \emph{maximum entropy}.
\end{example}

%----------------------------------------------------------------------
\section{Bernoulli distribution}
%----------------------------------------------------------------------
The Bernoulli distribution is the distribution of an indicator variable.

\begin{center}
\begin{tabular}{ll}\hline
Notation			& $X\sim\text{Bernoulli}(p)$ \\
Parameter(s)		& $p \in [0,1]$ \quad (probability of success) \\
Range			& $\{0,1\}$ \\
PMF				& $f(0) = 1-p$ and $f(1) = p$ \\ \hline
\end{tabular}
\end{center}

\begin{lemma}
If $X\sim\text{Bernoulli}(p)$, then 
\[
\expe(X)= p \quad\text{and}\quad \var(X) = p(1-p).
\]
\end{lemma}

\begin{proof}
\begin{align*}
\expe(X) 
	& 	= \sum_{k=0}^1 k\,f(k)
	 	= \big[0\times(1-p)\big] + \big[1\times p\big]
		= p. \\
\expe(X^2) 
	& 	= \sum_{k=0}^1 k^2\,f(k)
		= \big[0^2\times(1-p)\big] + \big[1^2\times p\big]
		= p. \\
\var(X)	
	& 	= \expe(X^2)-\expe(X)^2
		= p - p^2
		= p(1-p).
\end{align*}
\end{proof}

\newpage
% example
\begin{example}
Find the entropy of the $\text{Bernoulli}(p)$ distribution, and show that it reaches its maximum value when $p=1/2$.
\end{example}

\begin{solution}
Let $X\sim\text{Bernoulli}(p)$. Then $f(0)=1-p$ and $f(1)=p$, so
\[
H(X) = -\sum_{k=1}^n f(k)\log f(k) = -(1-p)\log(1-p) - p\log p.
\]
To find the maximum, we first differentiate with respect to p:
\begin{align*}
\frac{dH}{dp} 
	& = \log(1-p) -(1-p)\frac{1}{1-p}(-1) - \log p - p\,\frac{1}{p} \\
	& = \log(1-p) - \log p = \log\left(\frac{1-p}{p}\right).
\end{align*}
Setting this equal to zero,
\[
\log\left(\frac{1-p}{p}\right) = 0
\quad\Rightarrow\quad
\frac{1-p}{p} = 1
\quad\Rightarrow\quad
p = 1/2, \quad\text{as required.}
\]
\end{solution}


%----------------------------------------------------------------------
\section{Binomial distribution}
%----------------------------------------------------------------------

%The binomial distribution has two parameters, 
%\bit
%\it the \emph{number of trials} $n$, and 
%\it the \emph{probability of success} $p$. 
%\eit
%It is the distribution of the number of successes in a sequence of $n$ independent Bernoulli trials, where $p$ the probability of success in each trial.

The binomial distribution is the distribution of the number of successes in a sequence of independent Bernoulli trials.

\begin{center}
\begin{tabular}{ll}\hline
Notation			& $X\sim\text{Binomial}(n,p)$ \\
Parameter(s)		& $n \in\mathbb{N}$ \qquad (number of trials) \\
				& $p \in[0,1]$ \quad (probability of success) \\
Range			& $\{0,1,2,\ldots,n\}$ \\
PMF				& $f(k) = \displaystyle\binom{n}{k}p^k(1-p)^{n-k}$\\[2ex] \hline
\end{tabular}
\end{center}

\begin{lemma}\label{lem:binom-mean-var}
If $X\sim\text{Binomial}(n,p)$, then 
\[
\expe(X)= np \quad\text{and}\quad \var(X) = np(1-p).
\]
\end{lemma}

\begin{proof}
Consider a sequence of independent Bernoulli trials in which the probability of success is $p$. Let $X_i$ be the indicator variable of the event that success occurs on the $i$th trial, and let $X$ be the total number of successes in $n$ trials:
\[
X = \sum_{i=1}^n X_i \qquad\text{where}\quad X_i =
  \begin{cases}
   1 & \text{if success on the $i$th trial,} \\
   0 & \text{otherwise.}
  \end{cases}
\]
Then $X_i\sim\text{Bernoulli}(p)$ for each $i=1,2,3,\ldots$, and $X\sim\text{Binomial}(n,p)$. Now,
\[
\expe(X_i) = p \text{\quad and\quad} \var(X_i)=p(1-p),
\]
so by the linearity of expectation,
\begin{align*}
\expe(X) & = \expe(X_1)+\expe(X_2)+\ldots+\expe(X_n) = np, \\
\intertext{and because the trials are independent,}
\var(X) & = \var(X_1)+\var(X_2)+\ldots+\var(X_n) = np(1-p).
\end{align*}
\end{proof}

%We show that if $X\sim\text{Binomial}(n,p)$, then $\expe(X)=np$ and $\var(X)=np(1-p)$.
%
%\vspace*{2ex}
%Let $q=1-p$. Then the PMF of $X$ can be written as 
%\[
%f(k) = \binom{n}{k} p^k q^k,\qquad k=0,1,\ldots,n.
%\]
%The expected value is
%\begin{align*}
%\expe(X)
%	= \sum_{k=0}^{n} kf(k)
%	& = \sum_{k=0}^{n}\binom{n}{k} k p^{k}q^{n-k} \\
%	& = \sum_{k=1}^{n}\frac{n!}{(k-1)!(n-k)!} p^{k} q^{n-k} \\
%	& = np\sum_{k=1}^{n}\frac{(n-1)!}{(k-1)!(n-k)!} p^{k-1} q^{n-k}.
%\end{align*}
%
%\textbf{Claim}: $\displaystyle\sum_{k=1}^{n}\frac{(n-1)!}{(k-1)!(n-k)!} p^{k-1} q^{n-k} = 1$.
%
%Let $j=k-1$. Then
%\begin{align*}
%\sum_{k=1}^{n}\frac{(n-1)!}{(k-1)!(n-k)!} p^{k-1} q^{n-k}
%	& = \sum_{j=0}^{n-1}\frac{(n-1)!}{j!(n-j-1)!} p^{j} q^{n-j-1} \\
%	& = \sum_{j=0}^{n-1}\binom{n-1}{j} p^{j} q^{n-j-1} \\
%	& = (p+q)^{n-1} = 1 \qquad\text{(by the binomial theorem).}
%%	& = 1.%\qquad\text{(since $p+q=1$).}
%\end{align*}
%Hence $\expe(X)=np$.
%To compute the variance, we use the identity 
%\[
%\expe(X^2) = \expe\big[X(X-1]\big]+\expe(X).
%\]
%%First,
%\begin{align*}
%\expe[X(X - 1)]
%	= \sum_{k=0}^{n} k(k-1)f(k) 
%	& = \sum_{k=0}^{n} \binom{n}{k} k(k-1)p^{k} q^{n-k} \\
%	& = \sum_{k=2}^{n}\frac{n!}{(k-2)!(n-k)!} p^{k} q^{n-k}  \\
%	& = n(n-1)p^{2} \sum_{k=2}^{n}\frac{(n-2)!}{(k-2)!(n-k)!} p^{k-2} q^{n-k}.
%\end{align*}
%
%\textbf{Claim}: $\displaystyle\sum_{k=2}^{n}\frac{(n-2)!}{(k-2)!(n-k)!} p^{k-2} q^{n-k} = 1$.
%
%\vspace*{2ex}
%Let $j=k-2$. Then
%\begin{align*}
%\sum_{k=2}^{n}\frac{(n-2)!}{(k-2)!(n-k)!} p^{k-2} q^{n-k} 
%	& = \sum_{j=0}^{n-2}\frac{(n-2)!}{j!(n-2-j)!} p^{j} q^{n-2-j} \\
%	& = \sum_{j=0}^{n-2}\binom{n-2}{j} p^{j} q^{n-2-j} \\
%	& = (p+q)^{n-2} = 1\qquad\text{(by the binomial theorem).}
%%	& = n(n-1)p^2 \qquad\text{(since $p+q=1$).}
%\end{align*}
%
%%\begin{align*}
%%\expe(X(X - 1))
%%	& = n(n-1)p^{2} \sum_{j=0}^{n-2}\frac{(n-2)!}{j!(n-2-j)!} p^{j} q^{n-2-j} \qquad\text{where $j=k-2$,} \\
%%	& = n(n-1)p^{2} \sum_{j=0}^{n-2}\binom{n-2}{j} p^{j} q^{n-2-j} \\
%%	& = n(n-1)p^{2}(p+q)^{n-2}\qquad\text{(by the binomial theorem)} \\
%%	& = n(n-1)p^2 \qquad\text{(since $p+q=1$).}
%%\end{align*}
%Thus $\expe[X(X - 1)]=n(n-1)p^2$, so
%\[
%\expe(X^2) 
%	= \expe[X(X-1)] + \expe(X)
%	= n(n-1)p^2 + np,
%\]
%and hence
%\begin{align*}
%\var(X)	
%	& = \expe(X^2)-\expe(X)^2 \\
%	& = n(n-1)p^2 + np - n^2p^2 \\
%	& = np(1-p).
%\end{align*}

\newpage
% example
\begin{example}
A multiple choice test consists of ten questions, each with a choice of three different answers. A student decides to choose the answers at random. Find the mean and variance of the number of correct answers.
\end{example}

\begin{solution}
Let $X$ be the number of correct answers. Then $X\sim\text{Binomial}(n,p)$ where
\bit
\it $n=10$  (the total number of questions), and
\it $p=1/3$ (the probability of a correct answer).
\eit
Thus
\[
\expe(X)=np=\frac{10}{3}\quad\text{and}\quad \var(X)=np(1-p)=\frac{20}{9}.
\]
\end{solution}

%% skewness and kurtosis
%\subsubsection*{Skewness and kurtosis}
%\ben
%\it The skewness of $X\sim\text{Binomial}(n,p)$ is
%\[
%\gamma_1 = \frac{1-2p}{\sqrt{np(1-p)}} = \begin{cases}
%	< 0 	& \text{if } p > \frac{1}{2} \\
%	= 0 	& \text{if } p = \frac{1}{2} \\
%	> 0 	& \text{if } p < \frac{1}{2} \\
%\end{cases}	
%\]
%\it The kurtosis of $X\sim\text{Binomial}(n,p)$ is
%\[
%\gamma_2 = \frac{1-6p(1-p)}{np(1-p)} = \begin{cases}
%	< 0 				& \text{if } \left|p -\frac{1}{2}\right| < \frac{1}{2\sqrt{3}} \\
%	> 0 				& \text{if } \left|p -\frac{1}{2}\right| > \frac{1}{2\sqrt{3}} \\
%\end{cases}
%\]
%\een
%
%\bit
%\it $X$ has \emph{negative kurtosis} (low and wide peak) when p is close to $1/2$.
%\it $X$ has \emph{positive kurtosis} (tall and narrow peak) when $p$ is close to $0$ of $1$.
%\eit
%
%\begin{remark}
%\bit
%\it The kurtosis $\gamma_2$ reaches its minimum value when $p=\frac{1}{2}$. 
%\it For any fixed $p\in(0,1)$, $\gamma_2\to 0$ as $n\to\infty$.
%\it The kurtosis of the \emph{normal distribution} is $0$.
%\eit
%\end{remark}
%
%The \emph{de\,Moivre\,--\,Laplace} theorem states that the binomial distribution converges to a normal distribution as the number of trials $n\to\infty$.
%\bit
%%\it The binomial distribution converges to a normal distribution as $n\to\infty$.
%\it This is a special case of the \emph{central limit theorem} (CLT).
%\eit


%----------------------------------------------------------------------
\section{Exercises}
% !TEX root = main.tex
% ex11_uniform_bernoulli_binomial.tex
\begin{exercise}
\begin{questions}
%----------------------------------------
% RND p97q5
\question
A pair of fair dice is rolled six times. What is the probability of getting a total of seven
\ben
\it twice,
\it at least once,
\it more than three times?
\een
\begin{answer}
Let $X$ be the number of times that a total of 7 is obtained. 
\bit
\it Then $X\sim\text{Binomial}(n,p)$ with $n=6$ and $p=\frac{6}{36}=\frac{1}{6}$.
\eit
\ben
\it $\prob(X=2) 
	= \binom{6}{2}\left(\frac{1}{6}\right)^2 \left(\frac{5}{6}\right)^4 
	= 0.2009$.
\it $\prob(X\geq 1) 
	= 1 - \prob(X=0) 
	= 1 - \left(\frac{5}{6}\right)^6 
	= 0.6651$.
\it $\prob(X>3)
	= \prob(X=4) + \prob(X=5) +\prob(X=6)
	= 30\left(\frac{1}{6}\right)^4 \left(\frac{5}{6}\right)^2
		+ 6\left(\frac{1}{6}\right)^5 \left(\frac{5}{6}\right) 
		+ \left(\frac{1}{6}\right)^6 
	= 0.0087$.
\een
\end{answer}

%----------------------------------------
% RND p97q4 (modified)
\question
A biased coin, for which the probability of getting a head is $1/4$, is tossed $10$ times. What are the probabilities of observing
\ben
\it exactly two heads,
\it fewer than two heads,
\it more than two heads,
\it at most two heads, 
\it at least two heads?
\een
\begin{answer}
Let $X$ be the number of heads obtained. Then $X\sim\text{Binomial}(10,0.15)$. 
\ben
\it $\prob(X=2) 
	= \binom{10}{2}\left(\frac{1}{4}\right)^2 \left(\frac{3}{4}\right)^8 
	= 0.2816$.
\it $\prob(X\leq 1) 
	= \prob(X=0) + \prob(X=1) 
	= \left(\frac{3}{4}\right)^{10} + 10\left(\frac{1}{4}\right)^1 \left(\frac{3}{4}\right)^9 
	= 0.2441$.
\it $\prob(X>2)
	= 1 = \prob(X\leq 2)
	= 1 - \prob(X=0) + \prob(X=1) +\prob(X=2)
	= 0.4744$.
\it $\prob(X\leq 2)
	= \prob(X=0) + \prob(X=1) +\prob(X=2)
	= 0.5260$.
\it $\prob(X\geq 2)
	= 1 = \prob(X<2)
	= 1 - \prob(X=0) + \prob(X=1)
	= 0.7560$.
\een
\end{answer}
%----------------------------------------

%----------------------------------------
% binomial
% RND p98q6a
\question
The probability that a production line produces a faulty item is $0.1$. Are you more likely to find at most one faulty item in a sample of $10$ items, or at most two faulty items in a sample of $20$ items?

\begin{answer}
\bit
\it Let $X$ be the number of faulty items in a sample of size $10$. Then $X\sim\text{Binomial}(10,0.1)$.
\it Let $Y$ be the number of faulty items in a sample of size $20$. Then $Y\sim\text{Binomial}(20,0.1)$.
\eit
\begin{align*}
\prob(\text{at most one faulty item in 10})
	& = \prob(X=0) + \prob(X=1) \\
	& = (0.9)^{10} + 10(0.1)(0.9)^9 = 0.7361. \\
\prob(\text{at most two faulty items in 20})
	& = \prob(Y=0) + \prob(Y=1) + \prob(Y=2) \\
	& = (0.9)^{20} + 20(0.1)(0.9)^9 + 190(0.1)^2(0.9)^{18} = 0.6769.
\end{align*}
Thus we are more likely to observe at most one faulty item in a sample of $10$ items.
\end{answer}

%----------------------------------------
\question % GS 2.7.7
Airlines find that customers who reserve a seat fail to turn up with probability $0.1$. To avoid having empty seats, EasyJet always sell 10 tickets for their 9-seater aeroplane, while Ryanair always sell 20 tickets for their 18-seater aeroplane. Which of the two airlines is most often overbooked?
\begin{answer}
Let $X$ and $Y$ denote the (random) number of people on an EasyJet and Ryanair flight respectively. Then $X\sim\text{Binomial}(10,0.9)$ and $Y\sim\text{Binomial}(20,0.9)$, so
\begin{align*}
\prob(X=k)	& = \binom{10}{k}\left(\frac{9}{10}\right)^k\left(1-\frac{9}{10}\right)^{10-k} \\
\prob(Y=k)	& = \binom{20}{k}\left(\frac{9}{10}\right)^k\left(1-\frac{9}{10}\right)^{20-k}
\end{align*}
Thus
\begin{align*}
\prob(\text{EasyJet flight is overbooked}) 
	& = \prob(X=10) = \left(\frac{9}{10}\right)^{10} = 0.3487 \\
\prob(\text{Ryanair flight is overbooked}) 
	& = \prob(Y=19)+\prob(Y=20) \\
	& = 20\left(\frac{9}{10}\right)^{19}\left(\frac{1}{10}\right) + \left(\frac{9}{10}\right)^{20} = 0.3917
\end{align*}	  
so Ryanair is overbooked more often than EasyJet.
\end{answer}

%----------------------------------------
% binomial (backwards)
% RND p99q8
\question
A random variable $X$ has binomial distribution with mean $1.5$ and variance $1.275$. Find the probability that $X$ is at most $2$.

\begin{answer}
If $X\sim\text{Binomial}(n,p)$ then $\expe(X)=np$ and $\var(X)=np(1-p)$. Solving the equations $np=1.5$ and $np(1-p)=1.275$ in terms of $n$ and $p$, we obtain $n=10$ and $p=0.15$. Thus
\begin{align*}
\prob(X\leq 2) = \sum_{k=0}^2 \prob(X=k)
	& = \sum_{k=0}^2 \binom{10}{k}(0.15)^k (0.85)^{10-k} \\
	& = (0.85)^{10} + 10(0.15)(0.85)^9 + 45(0.15)^2(0.85)^8 \\
	& = 0.8202.
\end{align*}
\end{answer}

%----------------------------------------
% mean & variance
% GS 3.11.14
\question
Let $X_1,X_2,\ldots,X_n$ be independent random variables, with $X_i\sim\text{Bernoulli}(p_i)$ for $i=1,2,\ldots,n$. Show that the mean and variance of their sum $X=X_1+X_2+\ldots+X_n$ are given by
\[
\expe(X) = \displaystyle\sum_{i=1}^n p_i \quad\text{and}\quad \var(X) = \displaystyle\sum_{i=1}^n p_i(1-p_i) \quad\text{respectively.}
\]
\begin{answer}
Since $X_i\sim\text{Bernoulli}(p_i)$ we have $\expe(X_i)=p_i$ and $\var(X_i)=p_i(1-p_i)$. 

By the linearity of expectation,
\begin{align*}
\expe(X)	& = \expe\left(\sum_{i=1}^n X_i\right) = \sum_{i=1}^n \expe(X_i) = \sum_{i=1}^n p_i, \\
\intertext{and because the $X_i$ are independent,}
\var(X)	& = \var\left(\sum_{i=1}^n X_i\right) = \sum_{i=1}^n \var(X_i) = \sum_{i=1}^n p_i(1-p_i),
\end{align*}
as required.
\end{answer}

%----------------------------------------
% binomial (function)
% RND p99q10
\question
Suppose that $n$ independent Bernoulli trials are carried out, each having probability of success $p$. Let the number of successes and failures obtained in these trials be denoted by $X$ and $Y$ respectively. Find the PMF of $Z=X-Y$, and show that $\expe(Z)=n(2p-1)$. 
\par
[Hint: use the fact that $Y=n-X$.]
\begin{answer}
Since $Y=n-X$ we have that $Z = X-(n-X) = 2X-n$, and since $X$ takes values in the set $\{0,1,2,\ldots,n\}$, it follows that $Z$ takes values in the set $\{-n,-n+2,-n+4,\ldots,n-4,n-2,n\}$. 

Thus the PMF of $Z$ is
\begin{align*}
\prob(Z=k) 
	& = \prob(2X-n=k) \\
	& = \prob\left(X = \frac{1}{2}(n+k)\right) \\
	& = \begin{cases}
	\displaystyle\binom{n}{\frac{1}{2}(n+k)} p^{\frac{1}{2}(n-k)}(1-p)^{\frac{1}{2}(n-k)}	&\text{for}\ \ k\in \{-n,-n+2,\ldots,n-2,n\} \\[2ex]
	0	& \text{otherwise.}
\end{cases}	
\end{align*}
The expected value of $Z$ can be easily computed using the linearity of expectation:
\[
\expe(Z) = \expe(2X-n) = 2\expe(X)-n = 2np - n = n(2p-1),
\]	
as required. 
\end{answer}

\end{questions}
\end{exercise}
%======================================================================
\endinput
%======================================================================

%----------------------------------------------------------------------

\newpage
%----------------------------------------------------------------------
\section*{Long proof of Lemma~\ref{lem:binom-mean-var}$^{*}$}
%----------------------------------------------------------------------

\underline{To show}: if $X\sim\text{Binomial}(n,p)$, then $\expe(X)=np$ and $\var(X)=np(1-p)$.

\vspace*{2ex}
Let $q=1-p$. The PMF of $X$ can be written as 
\[
f(k) = \binom{n}{k} p^k q^k,\qquad k=0,1,\ldots,n.
\]
The expected value is
\begin{align*}
\expe(X)
	= \sum_{k=0}^{n} kf(k)
	= \sum_{k=0}^{n}\binom{n}{k} k p^{k}q^{n-k} 
	& = \sum_{k=1}^{n}\frac{n!}{(k-1)!(n-k)!} p^{k} q^{n-k} \\
	& = np\sum_{k=1}^{n}\frac{(n-1)!}{(k-1)!(n-k)!} p^{k-1} q^{n-k}.
\end{align*}

%\textbf{To show}: \[\displaystyle\sum_{k=1}^{n}\frac{(n-1)!}{(k-1)!(n-k)!} p^{k-1} q^{n-k} = 1.\]
%
Letting $j=k-1$,
\begin{align*}
\sum_{k=1}^{n}\frac{(n-1)!}{(k-1)!(n-k)!} p^{k-1} q^{n-k}
	= \sum_{j=0}^{n-1}\frac{(n-1)!}{j!(n-j-1)!} p^{j} q^{n-j-1} 
	& = \sum_{j=0}^{n-1}\binom{n-1}{j} p^{j} q^{n-j-1} \\
	& = (p+q)^{n-1} = 1 \qquad\text{(by the binomial theorem).}
%	& = 1.%\qquad\text{(since $p+q=1$).}
\end{align*}
Hence $\expe(X)=np$.
To compute the variance, we use the identity 
\[
\expe(X^2) = \expe\big[X(X-1)\big]+\expe(X).
\]
First,
\begin{align*}
\expe[X(X - 1)]
	= \sum_{k=0}^{n} k(k-1)f(k) 
	& = \sum_{k=0}^{n} \binom{n}{k} k(k-1)p^{k} q^{n-k} \\
	& = \sum_{k=2}^{n}\frac{n!}{(k-2)!(n-k)!} p^{k} q^{n-k}  \\
	& = n(n-1)p^{2} \sum_{k=2}^{n}\frac{(n-2)!}{(k-2)!(n-k)!} p^{k-2} q^{n-k}.
\end{align*}

%\textbf{Claim}: $\displaystyle\sum_{k=2}^{n}\frac{(n-2)!}{(k-2)!(n-k)!} p^{k-2} q^{n-k} = 1$.

\vspace*{2ex}
Letting $j=k-2$,
\begin{align*}
\sum_{k=2}^{n}\frac{(n-2)!}{(k-2)!(n-k)!} p^{k-2} q^{n-k} 
	= \sum_{j=0}^{n-2}\frac{(n-2)!}{j!(n-2-j)!} p^{j} q^{n-2-j}
	& = \sum_{j=0}^{n-2}\binom{n-2}{j} p^{j} q^{n-2-j} \\
	& = (p+q)^{n-2} = 1\qquad\text{(by the binomial theorem).}
%	& = n(n-1)p^2 \qquad\text{(since $p+q=1$).}
\end{align*}

%\begin{align*}
%\expe(X(X - 1))
%	& = n(n-1)p^{2} \sum_{j=0}^{n-2}\frac{(n-2)!}{j!(n-2-j)!} p^{j} q^{n-2-j} \qquad\text{where $j=k-2$,} \\
%	& = n(n-1)p^{2} \sum_{j=0}^{n-2}\binom{n-2}{j} p^{j} q^{n-2-j} \\
%	& = n(n-1)p^{2}(p+q)^{n-2}\qquad\text{(by the binomial theorem)} \\
%	& = n(n-1)p^2 \qquad\text{(since $p+q=1$).}
%\end{align*}
Hence $\expe[X(X - 1)]=n(n-1)p^2$. This yields
\[
\expe(X^2) 
	= \expe[X(X-1)] + \expe(X)
	= n(n-1)p^2 + np,
\]
so the variance of $X$ is 
\begin{align*}
\var(X)	
	& = \expe(X^2)-\expe(X)^2 \\
	& = n(n-1)p^2 + np - n^2p^2 \\
	& = np(1-p).
\end{align*}
as required.

%======================================================================
\endinput
%======================================================================
