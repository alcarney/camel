% !TEX root = main.tex
% ex20_limit_theorems.tex
\begin{exercise}
\begin{questions}
%----------------------------------------
%--------------------
% NORMAL APPROX TO BINOMIAL
% RND 4.3.3
\question
$90\%$ of all items produced by a manufacturing process are satisfactory. Find an approximation for the probability that a sample of $250$ items contains exactly $25$ defective items. %State any assumptions you make.

%[\slshape Hint: Use the normal approximation to the binomial distribution.\normalfont]

\begin{answer}
Let $X$ be the number of defective items in the sample. Assume that the probability that any particular item is defective is equal to $0.1$, independently of the quality of the other items inspected. Then $X\sim\text{Binomial}(n,p)$ with $n=250$ and $p=0.1$.

Since $np=25$ and $np(1-p)=22.5$, we define the normal approximation to $X$ by the random variable $Y\sim\text{N}(25,22,5)$. Then
\begin{align*}
\prob(X=25)
	& \approx \prob(24.5\leq Y\leq 25.5) \\
	& = \prob(Y\leq 25.5) - \prob(Y\leq 24.5) \\
	& = \prob\left(Z\leq \frac{25.5-25}{\sqrt{22.5}}\right) - \prob\left(Z\leq \frac{254.5-25}{\sqrt{22.5}}\right) \qquad\text{where $Z\sim\text{N}(0,1)$,}\\
	& = \Phi\left(\frac{25.5-25}{\sqrt{22.5}}\right) - \Phi\left(\frac{254.5-25}{\sqrt{22.5}}\right) \\
	& = \Phi(0.1054) - \Phi(-0.1054) \\
	& = 2\Phi(0.1054) - 1		\qquad\text{(by symmetry)} \\
	& = 0.08395 \qquad\text{(from tables).}
\end{align*}
In fact, the exact value is $0.08382$ so the approximation is reasonably good. (As a rule of thumb, the approximation is acceptable provided $np>10$ and $n(1-p)>10$.) 
\end{answer}


%--------------------
% RND 4.3.10
\question
A casino buys a new die and rolls it $600$ times. Let $N$ denote the number of times a six occurs.
\ben
\it Find the probability that $N$ is between $90$ and $100$, assuming that the die is fair.
\it Find the value $c$ for which $\prob(100-c \leq N\leq 100+c) = 0.95$, assuming that the die is fair.
\it What might the casino conclude if a six occurred $N=120$ times?
\een 

\begin{answer}
Let $X$ be the number of times a six occurs. Then $X\sim\text{Binomial}(n,p)$ with $n=600$ and $p=1/6$ if the dice is fair, in which case the mean of $X$ is $np=100$, and its variance is $np(1-p)=250/3$. Let $Y\sim\text{N}(100, 250/3)$ be the normal approximation to $X$.
\ben
\it % << (i)
\begin{align*}
\prob(90\leq X\leq 100)
	& \approx \prob(89.5\leq Y\leq 100.5) \\
	& = \prob(Y\leq 100.5) - \prob(Y\leq 89.5) \\
	& = \prob\left(Z\leq \frac{100.5-100}{\sqrt{250/3}}\right) - \prob\left(Z\leq \frac{89.5-100}{\sqrt{250/3}}\right) \qquad\text{where $Z\sim\text{N}(0,1)$,}\\
	& = \Phi\left(\frac{100.5-100}{\sqrt{250/3}}\right) - \Phi\left(\frac{89.5-100}{\sqrt{250/3}}\right) \\
	& = \Phi(0.0548) - \Phi(-1.1502) \\
	& = 0.3968 \qquad\text{(from tables).}
\end{align*}
\it % << (ii)
We wish to find a positive integer $N$ such that 
\begin{align*}
& \prob(99.5-N\leq Y\leq 100.5+N) = 0.95, 
\intertext{i.e.\ such that}
& \Phi\left(\frac{N+0.5}{\sqrt{250/3}}\right) - \Phi\left(\frac{N+0.5}{\sqrt{250/3}}\right) = 0.95, \\
\intertext{or (by symmetry)}
& \Phi\left(\frac{N+0.5}{\sqrt{250/3}}\right) = 0.975. \\
\intertext{Using tables,}
& \frac{N+0.5}{\sqrt{250/3}} = 1.96, \quad\text{and hence}\quad N = 0.5 + 1.96\sqrt{250/3}.
\end{align*}
This has no integer solution, but the integer that most nearly satisfies it is $N=17$.\par
In fact, $\prob(83\leq X\leq 117)=0.945$.
\it % << (iii)
He might concluded that there is some evidence that the die is unfair, because if the die is fair, $120$ is an unexpectedly large number of sixes.
\een
\end{answer}

\end{questions}
\end{exercise}

%======================================================================
\endinput
%======================================================================
