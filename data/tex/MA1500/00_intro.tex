% !TEX root = main.tex
%======================================================================
\chapter{Introduction}\label{chap:intro}
%======================================================================

%\addtolength{\tabcolsep}{1ex}
%\renewcommand{\arraystretch}{1.3}

%----------------------------------------------------------------------
\section{Timetable}
%----------------------------------------------------------------------

\subsection*{Lectures}

\begin{table}[h]
\centering
\begin{tabular}{|llll|}\hline
Day			& Time			& Location	& \\ \hline
Tuesdays	& 12.10 - 13.00	& E/0.15 	& \\
Thursdays	& 13.10 - 14.00	& E/0.15	& \\
Fridays		& 10.00 - 10.50	& Phys B LT		& (odd weeks only) \\ \hline
\end{tabular}
\end{table}

\subsection*{Tutorials}
Tutorials will take place during even-numbered weeks.
\bit
\it One or more problems will be tackeld by groups of 4-5 students.
\it Each group will be invited to present its answers on the whiteboard.
\eit


%----------------------------------------------------------------------
\section{Preparing for lectures}
%----------------------------------------------------------------------

The syllabus can be divided into four parts:

\begin{tabular}{ll}
Weeks 1--4:	    & Elementary probability theory \\
Weeks 5--6:	    & Random variables and distributions \\
Weeks 7--8:	    & Joint distributions \\
Weeks 9--10:	& Continuous distributions 
\end{tabular}

\bigskip
Reading material will be provided in advance of the lectures. This will consist of
\ben
\it definitions,
\it theorems and proofs,
\it illustrative examples, and
\it exercises.
\een
You are expected to read the relevant notes \emph{before} the lecture, and try some of the exercises.

\newpage

During each lecture, we will
\ben
\it clarify definitions, theorems and proofs (where necessary),
\it work on exercises, and
\it assess each other's work.
\een

Points to note:
\bit
\it Printed ``sample answers'' to exercises will \textbf{not} be provided.
\it Answers will be provided via \textbf{screencast}, but strictly on request.
\eit

%----------------------------------------------------------------------
\section{Assessment}
%----------------------------------------------------------------------

\subsection*{Formative assessment}
Answers to exercises can be submitted at any time for assessment and feedback.

\subsection*{Summative assessment}
Homework accounts for 15\% of the total marks for the module.
\begin{table}[h]
\centering
\begin{tabular}{|llll|}\hline
			& Hand-out      & Hand-in       & Hand-back \\ \hline
Homework 1  & Thu 22 Oct    & Thu 30 Oct    & Thu 06 Nov \\
Homework 2  & Thu 06 Nov    & Thu 13 Nov    & Thu 20 Nov \\
Homework 3  & Thu 20 Nov    & Thu 27 Nov    & Thu 04 Dec \\ \hline 
\end{tabular}
\end{table}

\bit
\it Each homework is worth 5\% of the total marks.
\it Homework should be handed in at the main office before 12:00 on the hand-in date.
\eit

\subsection*{Assessment criteria}

\fbox{\begin{minipage}{\linewidth}
\centering\vspace*{1ex}
Submitted work should be \textbf{clear}, \textbf{concise} and \textbf{correct}.
\vspace*{1ex}
\end{minipage}}


%----------------------------------------------------------------------
\endinput
%----------------------------------------------------------------------
