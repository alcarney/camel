% !TEX root = main.tex
% ex17_continuous.tex
\begin{exercise}
\begin{questions}
%----------------------------------------
\question
Let $A_1,A_2,\ldots$ be a countable family of sets. Prove the following versions of De Morgan's laws:
\begin{parts}
% part (i)
\part $\big(\bigcup_{i=1}^{\infty} A_i\big)^c = \bigcap_{i=1}^{\infty} A_i^c$.
\begin{answer}
Let $a\in\big(\bigcup_{i=1}^\infty A_i\big)^c$. Then $a\notin\bigcup_{i=1}^\infty A_i$, so $a\in A_i^c$ for all $A_i$, and hence 
\[
\left(\bigcup_{i=1}^\infty A_i\right)^c \subseteq \bigcap_{i=1}^\infty A_i^c.
\] 
Let $a\in\bigcap_{i=1}^\infty A_i^c$. Then $a\notin A_i$ for all $A_i$, so $a\notin\bigcup_{i=1}^\infty A_i$, and hence
\[
\bigcap_{i=1}^\infty A_i^c \subseteq \left(\bigcup_{i=1}^\infty A_i\right)^c.
\]
Thus it follows that $\big(\bigcup_{i=1}^{\infty} A_i\big)^c = \bigcap_{i=1}^{\infty} A_i^c$, as required.
\end{answer}
% part (ii)
\part $\big(\bigcap_{i=1}^{\infty} A_i\big)^c = \bigcup_{i=1}^{\infty} A_i^c$.
\begin{answer}
Applying part (1) to the sets $A_1^c, A_2^c, \ldots$,
\[
\left(\bigcup_{i=1}^\infty A_i^c\right)^c = \bigcap_{i=1}^\infty \big(A_i^c\big)^c = \bigcap_{i=1}^\infty A_i.
\]
Taking the complement of both sides,
\[
\left(\bigcap_{i=1}^\infty A_i\right)^c = \bigcap_{i=1}^\infty A_i^c,
\]
as required.
\end{answer}
\end{parts}
% final part
Hence, or otherwise, show that $\sigma$-fields are closed under countable intersections.
\begin{answer}
Let $\mathcal{F}$ be a $\sigma$-field and let $A_1,A_2,\ldots\in\mathcal{F}$. Since $\mathcal{F}$ is closed under complementation and countable unions,
\[
\bigcap_{n=1}^{\infty} A_n = \left(\bigcup_{n=1}^{\infty} A^c_n\right)^c \in\mathcal{F}.
\] 
as required.
\end{answer}

%--------------------
% ELEMENTARY
% RND p103 red
\question
A continuous random variable $X$ has PDF
\[
f(x) = \begin{cases}
	cx(1-x)	& \text{if }\ 0\leq x\leq 1,\\
	0		& \text{otherwise.}		
\end{cases}
\]
where $c$ is a constant.
\begin{parts}
\part % << (i)
Find the value of $c$.
\begin{answer}
 Since $f(x)$ is a PDF, we must have that $\int_{-\infty}^{\infty} f(x)\,dx=1$:
\[
\int_{-\infty}^{\infty} f(x)\,dx 
	= c\int_{0}^{1} x-x^2 \,dx 
	= c\left[\frac{x^2}{2}-\frac{x^3}{3}\right]_{0}^{1}
	= c\left(\frac{1}{2}-\frac{1}{3}\right)
	= \frac{c}{6}, \qquad\text{so}\quad c = 6.
\]
\end{answer}
\part % << (ii)
Find the CDF of $X$.
\begin{answer}
For $0\leq x\leq 1$,
\begin{align*}
F(x) = \prob(X\leq x) 
	& = \int_{-\infty}^{x} f(u)\,du  \\
	& = 6\int_{0}^{x} u-u^2 \,du \\
	& = 6\left[\frac{u^2}{2}-\frac{u^3}{3}\right]_{0}^{x} \\
	& = 6\left(\frac{x^2}{2}-\frac{x^3}{3}\right) 
	 = x^2(3-2x).
\end{align*}
The CDF of $X$ is therefore
\[
F(x) = \prob(X\leq x) = \begin{cases}
	0 			& x < 0, \\
	x^2(3-2x)	& 0\leq x\leq 1, \\
	1 			& x > 0. \\
\end{cases}
\]
\end{answer}
\part % << (iii)
Show that $\prob\left(\frac{1}{3}\leq X\leq\frac{2}{3}\right) = \frac{13}{27}$.
\begin{answer}
Using the CDF,
\begin{align*}
\prob\left(\frac{1}{3}\leq X\leq\frac{2}{3}\right)
	& = \prob\left(X\leq\frac{2}{3}\right) - \prob\left(X < \frac{1}{3}\right) \\
	& = \prob\left(X\leq\frac{2}{3}\right) - \prob\left(X \leq \frac{1}{3}\right) + \prob\left(X = \frac{1}{3}\right) \\
	& = F\left(\frac{2}{3}\right) - F\left(\frac{1}{3}\right) + 0 \\
%	& = \left(\frac{2}{3}\right)^2\left(3-\frac{4}{3}\right) - \left(\frac{1}{3}\right)^2\left(3-\frac{2}{3}\right) \\
	& = \left(\frac{4}{9}\times\frac{5}{3}\right) - \left(\frac{1}{9}\times\frac{7}{3}\right) 
	  = \frac{20}{27} - \frac{7}{27}
	  = \frac{13}{27}.
\end{align*}
Alternatively, using the PDF $f(x)$ we get 
\[
\prob\left(\frac{1}{3}\leq X\leq\frac{2}{3}\right)
	= 6\int_{1/3}^{2/3} x(1-x)\,dx
	= \Big[3x^2 - 2x^3\Big]_{1/3}^{2/3}
	= \frac{13}{27}.
\]
\end{answer}
\part Find the expected value and variance of $X$.
\begin{answer}
The mean and variance are computed as follows:
\begin{align*}
\expe(X) 
	& = \int_{-\infty}^{\infty} x\,f(x)\,dx 
	  = 6\int_0^1 x^2(1-x)\,dx 
	  = 6\left[\frac{x^3}{3}-\frac{x^4}{4}\right]_{0}^{1}
	  = \frac{1}{2}. \\[3ex]
\expe(X^2) 
	& = \int_{-\infty}^{\infty} x^2\,f(x)\,dx 
	  = 6\int_0^1 x^3(1-x)\,dx 
	  = 6\left[\frac{x^4}{4}-\frac{x^5}{5}\right]_{0}^{1}
	  = \frac{6}{20}. \\[3ex]
\var(X)
	& = \expe(X^2)-\expe(X)^2 = \frac{6}{20} - \left(\frac{1}{2}\right)^2	= \frac{1}{20}.
\end{align*}
\end{answer}
\end{parts}

%==========================================================================
% ELEMENTARY
\question
Let $X$ be a random variable with PDF $f(x)$ and range $[-1,1]$ (meaning that $f(x)=0$ for all $|x|>1$). Find the mean and variance of $X$ in each of the following cases:
\begin{parts}
\part % << (i)
$f(x) = (3/4)(1-x^2)$.
\begin{answer}
$\expe(X)=0$, $\var(X)=1/5$.
\end{answer}
\part % << (ii)
 $f(x) = (\pi/4)\cos(\pi x/2)$.
\begin{answer}
$\expe(X)=0$, $\var(X)=(\pi^2-8)/\pi^2$.
\end{answer}
\part % << (iii)
 $f(x) = (x+1)/2$.
\begin{answer}
$\expe(X)=1/3$, $\var(X)=2/9$.
\end{answer}
\part % << (iv)
 $f(x) = (3/8)(x + 1)^2$.
\begin{answer}
$\expe(X)=1/2$, $\var(X)=3/20$.
\end{answer}
\end{parts}

%For part (i), 
%\[
%\expe(X)	
%	= \frac{3}{4}\int_{-1}^1 x(1-x^2)\,dx
%	= \frac{3}{4}\int_{-1}^1 x - x^3 \,dx
%	= \frac{3}{4}\left[\frac{x^2}{2}-\frac{x^4}{4}\right]_{-1}^1
%	= 0
%\]
%
%(Note that because $f(x)$ is an even function and the integral is over a symmetric interval, the mean must be zero.) The second moment is
%\[
%\expe(X^2)	
%	= \frac{3}{4}\int_{-1}^1 x^2(1-x^2)\,dx
%	= \frac{3}{4}\int_{-1}^1 x^2 - x^4 \,dx 
%	= \frac{3}{4}\left[\frac{x^3}{3}-\frac{x^5}{5}\right]_{-1}^1
%	= \frac{1}{5}
%\]
%so $\var(X)=1/5$. The results for parts (ii), (iii) and (iv) follow similarly.
%\end{answer}



%--------------------
% ELEMENTARY
% RND 4.1.1
\question
Let $X$ be the amount of petrol (in thousands of litres) sold per week in a certain garage. The PDF of $X$ is as follows:
\[
f(x) = \begin{cases}
	cx^3(9-x^2)	& \text{if }\ 0\leq x\leq 3,\\
	0			& \text{otherwise,}		
\end{cases}
\]
where $c$ is a constant. 
\begin{parts}
\part Find the value of $c$.
\part Show that $\expe(X)=72/35$.
\part Compute the standard deviation of $X$.
\end{parts}
\begin{answer}
Since $f$ is a PDF, we must have $\int_{-\infty}^{\infty} f(x)\,dx = 1$:
\[
\int_{-\infty}^{\infty} f(x)\,dx 
	= c\int_0^3 x^3(9-x^2)\,dx 
	= c\left[\frac{9x^4}{4}-\frac{x^6}{6}\right]_0^3
	= \frac{243c}{4}, \qquad\text{so}\quad c = \frac{4}{243}.
\]
\begin{align*}
\expe(X) 
	& = \frac{4}{243}\int_0^3 x(9x^3-x^5)\,dx 
	= \frac{4}{243}\left[\frac{9x^5}{5}-\frac{x^7}{7}\right]_0^3
	= \frac{72}{35} \\
\expe(X^2) 
	& = \frac{4}{243}\int_0^3 9x^5-x^7\,dx 
	= \frac{4}{243}\left[\frac{9x^6}{6}-\frac{x^8}{8}\right]_0^3
	= \frac{9}{2} \\
\var(X) 
	& = \expe(X^2)-\expe(X)^2 
	= \frac{9}{2} - \left(\frac{72}{35}\right)^2 
	= 0.2682. \\
\intertext{and}	
\text{s.d.}(X)
	& = \sqrt{\var(X)} = 0.5178\quad\text{(or $517.8$ litres).}
\end{align*}
\end{answer}



%--------------------
% CAUCHY
% RND 4.1.12
\question
Let $X$ be a random variable having the \emph{Cauchy} distribution, whose PDF is as follows:
\[
f(x) = 	\frac{1}{\pi(1+x^2)}\quad\qquad (x\in\R).
\]
\ben
\it Show that $f(x)$ is indeed a PDF, and sketch the curve $y=f(x)$.
\it Find the CDF of $X$.
\it Find $\prob(-1\leq X\leq 1)$.
\een

\begin{answer}
\ben
\it % << (i)
To show that $f(x)$ is a PDF, we must show that $f(x)\geq 0$ for all $x\in\R$ and $\int_{-\infty}^{\infty} f(x)\,dx = 1$. It is clear that the first of these conditions is satisfied. The second condition also holds because
\[
\int_{-\infty}^{\infty} f(x)\,dx 
	= \int_{-\infty}^{\infty} \frac{1}{\pi(1+x^2)}\,dx 
	= \frac{1}{\pi}\big[\tan^{-1} x\big]_{-\infty}^{\infty}
	= \frac{1}{\pi}\left(\frac{\pi}{2}+\frac{\pi}{2}\right)
	= 1.
\]
\it % << (ii)
The CDF of $X$ is
\[
F(x) 
	= \int_{-\infty}^{x} \frac{1}{\pi(1+u^2)}\,du 
	= \frac{1}{\pi}\big[\tan^{-1} u\big]_{-\infty}^{x}
	= \frac{1}{\pi}\left(\tan^{-1} x + \frac{\pi}{2}\right).
\]	
\it % << (iii)
\[
\prob(-1\leq X\leq 1) = F(1) - F(-1) = \frac{1}{\pi}\left(\tan^{-1}(1) + \frac{\pi}{2}\right) - \frac{1}{\pi}\left(\tan^{-1}(-1) + \frac{\pi}{2}\right) = \frac{1}{2}.
\]
\een
\end{answer}

%--------------------
% RND 4.1.13
\question
The operational lifetime (in days) of a battery-operated toy can be modelled by a continuous random variable $X$ with the following PDF:
\[
f(x) = \begin{cases}
	\displaystyle\frac{cx(50+x)}{5000} 	& \text{if }\ 0\leq x\leq 50,\\[1ex]
	\qquad c						& \text{if }\ 50< x\leq 100,\\[1ex]
	\qquad 0		  				& \text{otherwise.}		
\end{cases}
\]
\ben
\it Evaluate $c$, and find the mean operational lifetime of the toy.
\it If the purchase price of the toy is $\pounds 5$ and battery costs are $2$p per day, find the average cost-per-day.
\een
\begin{answer}
\ben
\it % << (i)
Since $f(x)$ is a PDF, we must have that $\int_{-\infty}^{\infty} f(x)\,dx = 1$:
\[
\int_{-\infty}^{\infty} f(x)\,dx 
	= c\int_0^{50} \frac{x(50-x)}{5000}\,dx + c\int_{50}^{100}\,dx 
	= \frac{425c}{6}\qquad\text{so}\quad c = \frac{6}{425}.	
\]
The expected value of $X$ is therefore
\[
\expe(X) 
	= \int_{-\infty}^{\infty} xf(x)\,dx
	= \frac{6}{425}\int_0^{50} \frac{x^2(50-x)}{5000}\,dx + \frac{6}{425}\int_{50}^{100} x\,dx 
	= \frac{1075}{17}
	= 63.235\text{ days.}
\]
\it % << (ii)
Let $X$ be the lifetime of the toy, and let $Y$ be the overall cost per day. Then $Y = 2 + 500/X$ and by the linearity of expectation, $\expe(Y) = 2 + 500\expe(1/X)$. Now,
\[
\expe\left(\frac{1}{X}\right)
	= \int_{-\infty}^{\infty} \frac{1}{x}f(x)\,dx
	= \frac{6}{425}\int_0^{50} \frac{(50-x)}{5000}\,dx + \frac{6}{425}\int_{50}^{100}\frac{1}{x}\,dx 
	= \frac{9}{850} + \frac{6}{425}\log 2
\]
Thus
\[
\expe(Y) = 2 + 500\left(\frac{9}{850} + \frac{6}{425}\log 2\right) = 12.19\text{ pence per day.}
\]
\een
\end{answer}

%--------------------
% LINEARITY OF EXPECTATION
% RND 4.1.6
\question
Let $X$ be continuous random variable, and let $a,b\in\R$.
\begin{parts}
\part Show that $\expe(aX+b) = a\expe(X)+b$.
\begin{answer}
For any well-behaved function $g:\R\to\R$ we have $\displaystyle\expe(X) = \int_{-\infty}^{\infty} g(x)f(x)\,dx$.
\begin{align*}
\expe(aX+b) 
	& = \int_{-\infty}^{\infty} (ax+b)f(x)\,dx 
	= a\int_{-\infty}^{\infty} xf(x)\,dx + b\int_{-\infty}^{\infty} f(x)\,dx
	= a\expe(X) + b 
\end{align*}
\end{answer}
\part Show that $\var(aX+b) = a^2\var(X)$.
\begin{answer}
For brevity, let $\mu = \expe(X)$. 
\begin{align*}
\var(aX+b)
	& = \int_{-\infty}^{\infty} \big[(ax+b)-(a\mu+b)\big]^2\,dx 
	= a^2\int_{-\infty}^{\infty} \big(x-\mu)^2\,dx 
	= a^2\var(X) 
\end{align*}
\end{answer}
\end{parts}

\end{questions}
\end{exercise}


%======================================================================
\endinput
%======================================================================
